% e. woollett
%
%   2009,2010,2011
% file dirac2.tex  Dirac package version 2
% ch. 12 Dirac Algebra and Quantum Electrodynamics
% edit with Notepad++, then load into LED for latexing
\documentclass[11pt]{article}
\usepackage[dvips,top=1.5cm,left=1.5cm,right=1.5cm,foot=1cm,bottom=1.5cm]{geometry}
\usepackage{times,amsmath,amsbsy,graphicx,fancyvrb,url}
\usepackage[usenames]{color}
\usepackage{slashed}
%\definecolor{MyDarkBlue}{rgb}{0,0.08,0.45}
\definecolor{mdb}{rgb}{0.1,0,0.55}
\newcommand{\tcdb}{\textcolor{mdb}}
\newcommand{\tcbr}{\textcolor{BrickRed}}
\newcommand{\tcb}{\textcolor{blue}}
\newcommand{\tcr}{\textcolor{red}}
\urldef\tedhome\url{ http://www.csulb.edu/~woollett/  }
\urldef\tedmail\url{ woollett@charter.net}
%1.  this is for maxima code: red framed bold, footnotesize 
\DefineVerbatimEnvironment%
   {myVerbatim}%
   {Verbatim}%
   {fontfamily=courier,fontseries=b,fontsize=\footnotesize ,frame=single,rulecolor=\color{BrickRed}}
\DefineVerbatimEnvironment%
   {myVerbatim1}%
   {Verbatim}%
   {fontfamily=courier,fontseries=b,fontsize=\scriptsize ,frame=single,rulecolor=\color{BrickRed}}
%2.  this is for blue framed bold 
\DefineVerbatimEnvironment%
   {myVerbatim2}%
   {Verbatim}%
   {fontfamily=courier,fontseries=b,frame=single,rulecolor=\color{blue}}
\DefineVerbatimEnvironment%
   {myVerbatim2s}%
   {Verbatim}%
   {fontfamily=courier,fontseries=b,fontsize=\small,frame=single,rulecolor=\color{blue}}
\DefineVerbatimEnvironment%
   {myVerbatim2f}%
   {Verbatim}%
   {fontfamily=courier,fontseries=b,fontsize=\footnotesize,frame=single,rulecolor=\color{blue}}
% 3.  this is for black framed  bold
\DefineVerbatimEnvironment%
   {myVerbatim3}%
   {Verbatim}%
   {fontfamily= courier, fontseries=b, frame=single}
% 4.  this is for no frame bold
\DefineVerbatimEnvironment%
   {myVerbatim4}%
   {Verbatim}%
   {fontfamily=courier, fontseries=b,fontsize=\small}
% 6.  for defaults use usual verbatim
%\newcommand{\Feyn}[1]{#1\kern-0.45em/}
\newcommand{\Feyn}[1]{#1\kern-0.5em/}
\newcommand{\mv}{\Verb[fontfamily=courier,fontseries=b]}
\newcommand{\mvs}{\Verb[fontfamily=courier,fontseries=b,fontsize=\small]}
\newcommand{\mvf}{\Verb[fontfamily=courier,fontseries=b,fontsize=\footnotesize]}

\renewcommand{\thefootnote}{\ensuremath{\fnsymbol{footnote}}}
%%%%%%%%%%%%%%%%%%%%%%%%%%%%%%%%%%%%%%%%%%%%%%%%%%%%%%%%%%%%%%%%%%%%%%%%
%   title page
%%%%%%%%%%%%%%%%%%%%%%%%%%%%%%%%%%%%%%%%%%%%%%%%%%%%%%%%%%%%%%%%%%%%%%

\title{Maxima by Example:\\ Ch. 12, Dirac Algebra and Quantum Electrodynamics
             \thanks{This version uses \textbf{Maxima 5.23.2} \;
			 Check \; \textbf{ \tedhome } \; for the latest version of these notes. Send comments and
			 suggestions to \textbf{\tedmail} } }

%\author{Edwin L. Woollett \thanks{Dept. of Physics and Astron, Calif. State University at Long Beach,\\
%                    author's email: \tedmail} }
\author{Edwin L. Woollett}
\date{\today}
%\date{}
%%%%%%%%%%%%%%%%%%%%%%%%%%%%%%%%%%%%%%%%%
%          document
%%%%%%%%%%%%%%%%%%%%%%%%%%%%%%%%%%%%%%%%%%
\begin{document}
%\SMALL
%\small
\maketitle
\tableofcontents
\numberwithin{equation}{section}
\newpage
%\normalsize
\subsubsection*{Preface}
\begin{myVerbatim2s} 
COPYING AND DISTRIBUTION POLICY    
This document is part of a series of notes titled
"Maxima by Example" and is made available
via the author's webpage http://www.csulb.edu/~woollett/
to aid new users of the Maxima computer algebra system.	
	
NON-PROFIT PRINTING AND DISTRIBUTION IS PERMITTED.
	
You may make copies of this document and distribute them
to others as long as you charge no more than the costs of printing.	

These notes (with some modifications) will be published in book form
eventually via Lulu.com in an arrangement which will continue
to allow unlimited free download of the pdf files as well as the option
of ordering a low cost paperbound version of these notes.
\end{myVerbatim2s}	
\smallskip
\noindent \tcbr{Feedback from readers is the best way for this series of notes
  to become more helpful to new users of Maxima}.
\tcdb{\emph{All} comments and suggestions for improvements will be appreciated and
  carefully considered}.
\smallskip
\begin{myVerbatim2s}
LOADING FILES
The defaults allow you to use the brief version load(fft) to load in the
Maxima file fft.lisp.
To load in your own file, such as qxxx.mac 
using the brief version load(qxxx), you either need to place 
qxxx.mac in one of the folders Maxima searches by default, or
else put a line like:
  
file_search_maxima : append(["c:/work2/###.{mac,mc}"],file_search_maxima )$
  
in your personal startup file maxima-init.mac (see later in this chapter
for more information about this).

Otherwise you need to provide a complete path in double quotes,
as in load("c:/work2/qxxx.mac"),
 
We always use the brief load version in our examples, which are generated 
using the XMaxima graphics interface on a Windows XP computer, and copied
into a fancy verbatim environment in a latex file which uses the fancyvrb
and color packages.
\end{myVerbatim2s} 
\smallskip
\begin{myVerbatim2s}
Maxima.sourceforge.net. Maxima, a Computer Algebra System. Version 5.23.2
 (2011). http://maxima.sourceforge.net/
\end{myVerbatim2s}   
\large
The author would like to thank Maxima developer Barton Willis for his
  initial suggestion to use the Maxima code file \mv|simplifying.lisp| and his
  patient help in understanding how to make use of that code.
\normalsize
The new code has also been simplified (as compared with the rather long-winded
  code in version 1) by using a set of functions shared by Barton Willis which
  are used in making decision branches.
These functions are
\begin{myVerbatim2}
  mplusp, mtimesp, mnctimesp, mexptp, mncexptp.
\end{myVerbatim2}
(The ``nc'' additions refer to ``non-commutative'').
General expansions of multiple term arguments have been simplified
 by using Barton Willis' methods which combine \mv|lambda| and \mv|xreduce|.
\newpage
\setcounter{section}{12}
\pagestyle{headings}
\subsection{References}
The following works are useful sources.
Some abbreviations (which will be used in the following sections) are defined here.
\begin{itemize}
\item Aitchison, I.J.R., Relativistic Quantum Mechanics, Barnes and Noble,1972
\item A/H: Aitchison,I.J.R, and Hey, A.J.G., Gauge Theories in Particle
  Physics, Adam Hilger, 1989
\item B/D: Bjorken, James D. and Drell, Sidney D, Relativistic Quantum
  Mechanics, McGraw Hill, 1964
\item BLP: Berestetskii, V. B, Lifshitz, E. M., and Pitaevskii, L. P.,
  Quantum Electrodynamics, Course of Theoretical Physics, Vol. 4, 2nd. Ed.
  Pergamon Press, 1982
\item  Berestetskii, V. B, Lifshitz, E. M., and Pitaevskii, L. P.,
  Relativistic Quantum Theory, Part 1, Course of Theoretical Physics, Vol. 4, 
  Pergamon Press, 1971
\item De Wit, B. and Smith, J., Field Theory in Particle Physics, 
          North-Holland, 1986
\item G/R:  Greiner, Walter, and Reinhardt, Joachim, Quantum Electrodynamics,
   fourth ed., Springer, 2009
\item  Griffiths, Introduction to Elementary Particles, Harper and Row,
       1987
\item H/M: Halzen, Francis and Martin, Alan D., Quarks and Leptons,
         John Wiley, 1984
\item I/Z: Itzykson, Claude and Zuber, Jean-Bernard, Quantum Field Theory, 
        McGraw-Hill, 1980
\item Jauch, J.M., and Rohrlich, F., The Theory of Photons and Electrons,
    Second Expanded Edition, Springer-Verlag, 1976
\item Kaku, Michio, Quantum Field Theory, Oxford Univ. Press, 1993
\item Maggiore, Michele, A Modern Introduction to Quantum Field Theory,
    Oxford Univ. Press, 2005
\item P/S: Peskin, Michael E. and Schroeder, Daniel V., An Introduction
  to Quantum Field Theory, Addison-Wesley, 1995
\item Quigg, Chris, Gauge Theories of the Strong, Weak, and Electromagnetic
  Interactions, Benjamin/Cummings, 1983
\item Renton, Peter, Electroweak Interactions, Cambridge Univ. Press, 1990
\item Schwinger, Julian, Particles, Sources, and Fields, Vol. 1,
          Addison-Wesley, 1970
\item Sterman, George, An Introduction to Quantum Field Theory,
        Cambridge Univ. Press, 1993
\item Weinberg, Steven, The Quantum Theory of Fields, Vol.I,
          Cambridge Univ. Press, 1995
\end{itemize}
\subsection{High Energy Physics Notation Conventions}
We use the same conventions as P/S (Peskin and Schroeder) and
  Maggiore, {\large $\mathbf{\boldsymbol{\epsilon} ^{0 1 2 3}} = + 1$}, 
  the electromagnetic units are Heaviside-Lorentz, and natural units
  are used, so {\large $\mathbf{e^2 = 4 \boldsymbol{\pi \alpha}}$}.\\
  
\noindent The only exception is that we interpret the Feynman rules as
 an expression for the quantity {\large $-i\,M$},
 following the conventions of Griffiths, and Halzen/Martin.\\
 
\noindent The diagonal elements of the metric
  tensor are {\large $(1,-1,-1,-1)$}, and the helicity (chiral) representation
  used for explicit matrix methods with the Dirac gamma matrices is
  (in $2\,\times \,2$ block form)
\Large
\begin{equation}
 \gamma^{0} = \begin{pmatrix} 0 & 1 \\ 1 & 0 \end{pmatrix}, \quad
 \gamma^{j} = \begin{pmatrix} 0 & \sigma^{j} \\ -\sigma^{j} & 0 \end{pmatrix}, \quad
  \gamma^{5} = \begin{pmatrix} -1 & 0 \\ 0 & 1 \end{pmatrix},
\end{equation}
\normalsize
  in which {\large $\sigma^{j}$} for {\large $ j = 1, 2, 3$} are the $2\,\times \,2$ Pauli matrices
\Large
\begin{equation}
 \sigma^{1} = \begin{pmatrix} 0 & 1 \\ 1 & 0 \end{pmatrix}, \quad
 \sigma^{2} = \begin{pmatrix} 0 & -i \\ i & 0 \end{pmatrix}, \quad
  \sigma^{3} = \begin{pmatrix} 1 & 0 \\ 0 & -1 \end{pmatrix}.
\end{equation}
\normalsize
The above form for {\large $\boldsymbol{\gamma^{5}}$} is equivalent to
\Large
\begin{equation}
 \boldsymbol{\gamma^{5}} = \boldsymbol{ i\, \gamma^{0}\,\gamma^{1}\,\gamma^{2}\,\gamma^{3}}.
\end{equation}
\normalsize
The lepton Dirac spinor is the four element column matrix {\large $u(p,\sigma)$}  
    where {\large $\sigma = +1$} picks out positive helicity {\large $h = +1/2$}
  and {\large $\sigma = -1$} picks out negative helicity {\large $h = -1/2$}.
The corresponding barred particle spinor is a four element row matrix
\Large
\begin{equation}
\bar{u}(p,\sigma) = u^{\dagger}\,\gamma^{0}
\end{equation}
\normalsize  
  and the particle Dirac spinor normalization  is
\Large
\begin{equation}
 \bar{u}(p,\sigma)\,u(p,\sigma^{\prime}) = 2\,m\,\delta_{\sigma\,\sigma^{\prime}}, \quad
  \sigma, \, \sigma^{\prime} = \pm 1,
\end{equation}
\normalsize  
and we also have
\Large
\begin{equation}
 \sum_{\sigma} \,u(p,\sigma)\, \bar{u}(p,\sigma) = \Feyn{p} + m,
\end{equation}
\normalsize
in which a $4 \times 4$ unit matrix is understood to be multiplying
  the scalar mass symbol {\large $m$}, and
  {\large $\Feyn{p} = p_{\mu}\,\gamma^{\mu}$} with use of the
  summation convention.\\

\noindent The antilepton Dirac spinor is the four element column matrix {\large $v(p,\sigma)$}
  in which {\large $\sigma = +1$} picks out the positive helicity {\large $h = +1/2$} antiparticle
  state and {\large $\sigma = -1$} picks out negative helicity {\large $h = -1/2$}
  antiparticle state, is defined (with P/S phase choice) via
\Large
\begin{equation}
 v(p,\sigma) = - \gamma^{5}\,u (p,-\sigma)
\end{equation}
\normalsize  
The corresponding barred antiparticle spinor is a four element row matrix
\Large
\begin{equation}
\bar{v}(p,\sigma) = v^{\dagger}\,\gamma^{0}
\end{equation}
\normalsize  
and the normalization is
\Large
\begin{equation}
 \bar{v}(p,\sigma)\,v(p,\sigma^{\prime}) 