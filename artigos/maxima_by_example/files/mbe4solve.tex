% e. woollett
%   march 08
% file solve1.tex
% things to add:
% to_poly_solve and to_poly_solver
% nth roots of - 1
% fateman's example in solving.pdf
% matrix eqn for three equations and four unknowns
% sent to solve: see work3,maxima tricks3,solve-matrix-eqns.txt
% saving only real roots
%> finding yield a lot of imaginary numbers.  Is there perhaps a way
%> to make maxima only give the real roots of a cubic, or any other
%> polynomial for that matter?  They really just get in the way when
%> dealing with real processes.
%
%realroots, realonly.
%
%(%i1) realroots (x^3-3*x^2+4*x-2 = 0);
%(%o1)                              [x = 1]
%
%(%i2) algsys([x^3-3*x^2+4*x-2 = 0], [x]);
%(%o2)               [[x = %i + 1], [x = 1 - %i], [x = 1]]
%
%(%i3) block([realonly:true], algsys([x^3-3*x^2+4*x-2 = 0], [x]));
%(%o3)                             [[x = 1]] 
% more: see work3 file maxima tricks 3 roots-cubic-deleting-imag.txt
%(%i3) solve (x^3 - 6*x + 2), rectform, ratsimp;
%(%o3) [x = -sqrt(2)*sqrt(3)*sin((atan(sqrt(7))-%pi)/3)
%          -sqrt(2)*cos((atan(sqrt(7))-%pi)/3),
%        x = sqrt(2)*sqrt(3)*sin((atan(sqrt(7))-%pi)/3)
%          -sqrt(2)*cos((atan(sqrt(7))-%pi)/3),
%        x = 2*sqrt(2)*cos((atan(sqrt(7))-%pi)/3)]
%So I propose something like this:
%real_roots_3 (e) := block ([rsol : []],
%    for r in ratsimp(rectform(solve(e))) do (
%        if freeof(%i,r) then
%            rsol : cons(r,rsol)
%    ),
%    rsol
%)$
%(%i1) real_roots_3 (x^3 - 6*x + 2);
%(%o1) [x = 2*sqrt(2)*cos((atan(sqrt(7))-%pi)/3),
%        x = sqrt(2)*sqrt(3)*sin((atan(sqrt(7))-%pi)/3)
%          -sqrt(2)*cos((atan(sqrt(7))-%pi)/3),
%        x = -sqrt(2)*sqrt(3)*sin((atan(sqrt(7))-%pi)/3)
%          -sqrt(2)*cos((atan(sqrt(7))-%pi)/3)]
%(%i2) real_roots_3 (x^3-3*x^2+4*x-2);
%(%o2) [x = 1]

% solving iequalities barton: solve_rat_ineq.mac.
%(%i4) load("solve_rat_ineq.mac")$
% (%i5) solve_rat_ineq(x^2+2*x-3>=0);
% (%o5) [[x<=-3],[x>=1]]
%(%i4) load('topoly_solver')$ 
%(%i5) to_poly_solve(abs(x)=2,x);
%(%o5) [[x=2],[x=-2]]
% topoly_solver.mac, topoly.lisp solve methods: see work3, solve-system.txt
% finding complex roots symbolically barton
% roots of equations containing %i
% unsolved prob: sqrt(-43), num val (3 + 4*%i)^(1/3)
% solving float problems: keepfloat:true screws up linsolve
% and is ignored by solve: wrap solve in float and
%turn off ratprint:
%(%i1)  display2d:false$
%(%i2) [globalsolve,programmode];
%(%o2) [false,true]
%(%i3) block([ratprint:false],
%          linsolve([0.9*x + 0.3*y = 2.1,0.3*x -0.5*y = 5.4],[x,y] ) );
%(%o3) [x = 89/18,y = -47/6]
%(%i4) block([ratprint:false,keepfloat:true],
%          linsolve([0.9*x + 0.3*y = 2.1,0.3*x -0.5*y = 5.4],[x,y] ) );
%Inconsistent equations:  [2]
% -- an error.  To debug this try debugmode(true);
%(%i5) block([keepfloat:true ],
%        solve([0.9*x + 0.3*y = 2.1,0.3*x -0.5*y = 5.4],[x,y] ) );
%`rat' replaced -2.1 by -21/10 = -2.1

%`rat' replaced 0.9 by 9/10 = 0.9

%`rat' replaced 0.3 by 3/10 = 0.3

%`rat' replaced -5.4 by -27/5 = -5.4

%`rat' replaced 0.3 by 3/10 = 0.3

%`rat' replaced -0.5 by -1/2 = -0.5
%(%o5) [[x = 89/18,y = -47/6]]
%(%i6) 

%working float problems:
%wrap solve in float and turn off ratprint

%(%i6) fsolve(eqns,vars) := 
%         block([ratprint:false], float(solve(eqns,vars)) );
%(%o6) fsolve(eqns,vars):=block([ratprint:false],float(solve(eqns,vars)))
%(%i7) fsolve(0.9*a - a*x,x);
%(%o7) [x = 0.9]
%(%i8) fsolve(0.8*x^2 + 2.3*x - 4.3, x);
%(%o8) [x = -4.1653941053494,x = 1.2903941053494]
%(%i9) ratprint;
%(%o9) true
%
% edit with Notepad++, then load into LED for latexing
\documentclass[12pt]{article}
\usepackage[dvips,top=1.5cm,left=1.5cm,right=1.5cm,foot=1cm,bottom=1.5cm]{geometry}
\usepackage{times}
\usepackage{amsmath}
\usepackage{amsbsy}
\usepackage{graphicx} 
\usepackage{url}
\urldef\tedhome\url{ http://www.csulb.edu/~woollett/  }
\urldef\tedmail\url{ woollett@charter.net}

%%%%%%%%%%%%%%%%%%%%%%%%%%%%%%%%%%%%%%%%%%%%%%%%%%%%%%%%%%%%%%%%%%%%%%%%
%   title page
%%%%%%%%%%%%%%%%%%%%%%%%%%%%%%%%%%%%%%%%%%%%%%%%%%%%%%%%%%%%%%%%%%%%%%
\title{Maxima by Example: Ch.4: Solving Equations \thanks{This version uses Maxima 5.17.1. This is a live
            document. Check \;  \tedhome \; for the latest version of these notes. Send comments and
			 suggestions to \tedmail } }


\author{Edwin L. Woollett}
\date{\today}
%%%%%%%%%%%%%%%%%%%%%%%%%%%%%%%%%%%%%%%%%
%          document
%%%%%%%%%%%%%%%%%%%%%%%%%%%%%%%%%%%%%%%%%%
\begin{document}
\small
\maketitle
\tableofcontents
\numberwithin{equation}{section}
\newpage

\setcounter{section}{3}
\normalsize
COPYING AND DISTRIBUTION POLICY\\
\\    
	
    
    This document is part of a series of notes titled
    "Maxima by Example" and	is made available\\
	via the author's webpage 
	\verb|http://www.csulb.edu/~woollett/ | to aid
	new users of the Maxima computer algebra system.\\
	
	\smallskip
	NON-PROFIT PRINTING AND DISTRIBUTION IS PERMITTED.\\
	You may make copies of this document and distribute them to others
	as long as you charge no more than the costs of printing.\\
	
	\smallskip
	These notes (with some modifications) will be published in book form
	eventually via Lulu.com in an arrangement which will continue
	to allow unlimited free download of the pdf files as well as the option
	of ordering a low cost paperbound version of these notes.
\newpage

\section{Solving Equations}
Maxima has several functions which can be used for solving sets of algebraic equations and
  for finding the roots of an expression.
These are described in the Maxima manual, Sec. 21, and listed under "Contents" under "Equations".\\

This chapter gives examples of the following Maxima functions:
\begin{itemize}
\item \textbf{solve} solves a system of simultaneous linear or nonlinear polynomial equations
  for the specified variable(s) and returns a list of the solutions.
\item  \textbf{linsolve} solves a system of simultaneous linear equations for the specified
   variables and returns a list of the solutions.
\item   \textbf{find\_root} uses a combination of binary search and Newton-Raphson methods
   for univariate functions and will find a root when provided with an interval containing at least
   one root.
\item   \textbf{allroots} finds all the real and complex roots of a real univariate polynomial.
\item   \textbf{realroots} finds all of the real roots of a univariate polynomial
   within a specified tolerance.
\item  \textbf{eliminate} eliminates variables from a set of equations or expressions.
\item \textbf{linsolve\_by\_lu} solves a system of linear algebraic equations by the matrix
   method known as "LU decomposition", and provides a Maxima method to work with a set
   of linear equations in terms of the matrix of coefficients.
\item \textbf{newton}, naive univariate Newton-Raphson, and \textbf{mnewton}, multivariate Newton-Raphson,
   can deal with nonlinear function(s).
\end{itemize}
We also encourage the use of two dimensional plots to approximately locate solutions.\\
   
This chapter does not yet include "Solving  Recurrence Relations", and 
  "Solving One Hundred Equations".

\subsection{One Equation or Expression: Symbolic Solution or Roots}

\subsubsection{The Maxima Function solve}
Maxima's ability to solve equations is limited, but progress is being made in this area.
The Maxima manual has an extensive entry for the important function \textbf{solve},
  which you can view in Maxima with the input \verb|? solve| (no semicolon) followed
    by (Enter), or the equivalent command: \verb|describe(solve)$|.
The input \verb|example(solve)$| will show you the manual examples without the manual
  syntax material.
We will present some examples of the use of \textbf{solve} and not try to cover "everything".\\
\textbf{solve} tries to find exact solutions.
If \textbf{solve}\verb|(f(x),x)| cannot find an exact solution, \textbf{solve} tries to 
  return a simplified version of the original problem.
  Sometimes the "simplified" version can be useful:
\small
\begin{verbatim}
(%i1) f(x);
(%o1)                                f(x)
(%i2) solve( f(x)^2-1 , x );
(%o2)                       [f(x) = - 1, f(x) = 1] 
\end{verbatim}
\normalsize
Since Maxima's idea of what is "simpler" may not agree with your own, often
  the returned version is of no use.  
\newpage
  
The Maxima manual \textbf{solve} syntax discussion relevant to solving one equation is:
\small
\begin{quote}
Function: \textbf{solve}\verb|(expr, x) |\\
Function: \textbf{solve}\verb|(expr) |\\
Solves the algebraic equation \verb|expr| for the variable \verb|x| and returns a
  list of solution equations in \verb|x|.
If \verb|expr| is not an equation, the equation \verb|expr = 0| is assumed in
  its place.
\verb|x| may be a function (e.g. \verb|f(x)|), or other non-atomic expression
  except a sum or product.
\verb|x| may be omitted if \verb|expr| contains only one variable.
\verb|expr| may be a rational expression, and may contain trigonometric functions,
 exponentials, etc.\\
\verb|breakup| if \verb|false| will cause \verb|solve| to express the solutions
   of cubic or quartic equations as single expressions rather than as made up
   of several common subexpressions which is the default. \\
\verb|multiplicities| will be set to a list of the multiplicities of the
   individual solutions returned by \verb|solve|, \verb|realroots|, or \verb|allroots|.\\
Try \verb|apropos (solve)| for the switches which affect \verb|solve|.
\verb|describe| may then by used on the individual switch names if their purpose is not clear.
\end{quote}
\normalsize
It is important to recognise that the first argument to \textbf{solve} is either
an equation such as \verb|f(x) = g(x)| (or \verb|h(x) = 0|), or simply \verb|h(x)|; in
  the latter case, \textbf{solve} understands that you mean the equation \verb|h(x) = 0|,
and the problem is to find the "roots" of \verb|h(x)|, ie., values of \verb|x| such that 
   the equation \verb|h(x) = 0| is satisfied.\\
   
Here we follow the manual suggestion about using \textbf{apropos} and \textbf{describe}:
\small
\begin{verbatim}
(%i1) apropos(solve);
(%o1) [solve, solvedecomposes, solveexplicit, solvefactors, solvenullwarn, 
                          solveradcan, solvetrigwarn, solve_inconsistent_error]
(%i2) describe(solveradcan)$
 -- Option variable: solveradcan
     Default value: `false'

     When `solveradcan' is `true', `solve' calls `radcan' which makes
     `solve' slower but will allow certain problems containing
     exponentials and logarithms to be solved.
(%i3) describe(solvetrigwarn)$
 -- Option variable: solvetrigwarn
     Default value: `true'

     When `solvetrigwarn' is `true', `solve' may print a message saying
     that it is using inverse trigonometric functions to solve the
     equation, and thereby losing solutions.
\end{verbatim}
\normalsize
\subsubsection{solve with Expressions or Functions \& the multiplicities List}
Let's start with a simple example where the expected answers are obvious and
  check the behavior of \textbf{solve}.
In particular we want to check \textbf{solve}'s behavior with both an expression
  and a function (defined via \verb|:=|).
We also want to check how the system list \textbf{multiplicities} is created
  and maintained.
We include the use of \textbf{realroots} and \textbf{allroots} in this comparison,
  even though we will not have to use these latter two functions for a while.
 \small
 \begin{verbatim}
(%i1) multiplicities;
(%o1)                             not_set_yet
(%i2) ex1 : x^2 - 2*x + 1;
                                  2
(%o2)                            x  - 2 x + 1
\end{verbatim}
\newpage
\begin{verbatim}
(%i3) factor(ex1);
                                          2
(%o3)                              (x - 1)
(%i4) g(x) := x^2 - 2*x + 1$
(%i5) g(y);
                                  2
(%o5)                            y  - 2 y + 1
(%i6) solve(ex1);
(%o6)                               [x = 1]
(%i7) multiplicities;
(%o7)                                 [2]
(%i8) solve(g(y));
(%o8)                               [y = 1]
(%i9) multiplicities;
(%o9)                                 [2]
(%i10) realroots(ex1);
(%o10)                              [x = 1]
(%i11) multiplicities;
(%o11)                                [2]
(%i12) allroots(ex1);
(%o12)                        [x = 1.0, x = 1.0]
(%i13) multiplicities;
(%o13)                                [2]
\end{verbatim}
\normalsize
We see that we can use either an expression or a function with \textbf{solve}, and you can 
  check that this also applies to \textbf{realroots} and \textbf{allroots}.
It is not clear from our use of \textbf{allroots} above how \textbf{allroots} affects
  \textbf{multiplicities}, although, as we will see later, the manual does not assert
  any connection, and we would not expect there to be a connection because \textbf{allroots}
  returns multiple roots explicitly in \verb|%o12|.
Just to make sure, let's restart Maxima and use only \textbf{allroots}:
\small
\begin{verbatim}
(%i1) multiplicities;
(%o1)                             not_set_yet
(%i2) allroots(x^2 - 2*x + 1);
(%o2)                         [x = 1.0, x = 1.0]
(%i3) multiplicities;
(%o3)                             not_set_yet
 \end{verbatim}
 \normalsize
As we expected, \textbf{allroots} does not affect \textbf{multiplicities}; only \textbf{solve}
  and \textbf{realroots} set its value. 
\subsubsection{General Quadratic Equation or Function}
To get our feet wet, lets turn on the machinery with a general quadratic equation 
  or expression.
 There are some differences if you employ an expression rather than a function
   defined with \verb|:=|.
Each method has some advantages and some disadvantages.
Let's first use the function argument, rather than an expression argument.
We will later show how the calculation is different if an expression is used.
We will step through the process of verifying the solutions and end up with
  a "do loop" which will check all the solutions. We will use a function \verb|f(x)|
  which depends parametrically on \verb|(a,b,c)| as the first argument to \textbf{solve}, and
  first see what happens if we don't identify the unknown: how smart is Maxima??
\small
\begin{verbatim} 
(%i1) f(x) := a*x^2 + b*x + c$
\end{verbatim}
\newpage
\begin{verbatim}
(%i2) f(y);
                                   2
(%o2)                           a y  + b y + c
(%i3) sol : solve( f(x) );
More unknowns than equations - `solve'
Unknowns given :  
[a, x, b, c]
Equations given:  
    2
[a x  + b x + c]
 -- an error.  To debug this try debugmode(true);
\end{verbatim}
\normalsize
We see that Maxima cannot read our mind! We must tell Maxima which of the four symbols is
  to be considered the "unknown". From Maxima's point of view (actually the point of view
  of the person who wrote the code), one equation cannot determine four unknowns, so we must
  supply the information about which of the four variables is to be considered the unknown.
\small
\begin{verbatim}
(%i4) sol : solve( f(x),x );
                         2                         2
                   sqrt(b  - 4 a c) + b      sqrt(b  - 4 a c) - b
(%o4)       [x = - --------------------, x = --------------------]
                           2 a                       2 a
\end{verbatim}
\normalsize
We see that \textbf{solve} returns the expected list of two possible symbolic solutions.

\subsubsection{Checking Solutions with subst or ev and a "Do Loop"}
Let's check the first solution:
\small
\begin{verbatim}
(%i5) s1 : sol[1];
                                      2
                                sqrt(b  - 4 a c) + b
(%o5)                     x = - --------------------
                                        2 a
\end{verbatim}
\normalsize
Now we can use the \verb|subst( x = x1, f(x) )| form of the \textbf{subst} function syntax.
\small
\begin{verbatim}
(%i6) r1 : subst(s1, f(x) );
                   2              2            2
            (sqrt(b  - 4 a c) + b)    b (sqrt(b  - 4 a c) + b)
(%o6)       ----------------------- - ------------------------ + c
                      4 a                       2 a
(%i7) expand(r1);
(%o7)                                  0
\end{verbatim}
\normalsize
Now that we understand what steps lead to the desired "0", we automate the process
  using a do loop:
\small
\begin{verbatim}
(%i8) for i:1 thru 2 do disp( expand( subst( sol[i], f(x) ) ) )$
                                       0
                                       0
\end{verbatim}
\normalsize
For each of the two solutions (for \verb|x|) found by Maxima, the given expression
  evaluates to zero, verifying the roots of the expression.
\newpage
Since the result (here) of using \verb|ev( f(x), sol[i])| is the same as
  using \verb|subst( sol[i], f(x) )|, we can use \textbf{ev} instead:

\small
\begin{verbatim}
(%i9) for i:1 thru 2 do disp( expand( ev(f(x), sol[i]) ))$
                                       0
                                       0
\end{verbatim}
\normalsize

  
  
\subsubsection{The One Argument Form of solve}  
The simple one-argument form of \textbf{solve} can be used if all but one of the
  symbols in the expression is already "bound".
\small
\begin{verbatim}
(%i10) solve(3*x -2);
                                         2
(%o10)                               [x = -]
                                         3
(%i11) (a:1, b:2, c:3)$
(%i12) [a,b,c];
(%o12)                             [1, 2, 3]
(%i13) solve(a*x^2 + b*x + c);
(%o13)            [x = - sqrt(2) %i - 1, x = sqrt(2) %i - 1]
(%i14) [a,b,c] : [4,5,6];
(%o14)                             [4, 5, 6]
(%i15) solve(a*x^2 + b*x + c);
                        sqrt(71) %i + 5      sqrt(71) %i - 5
(%o15)           [x = - ---------------, x = ---------------]
                               8                    8
(%i16) [a,b,c] : [4,5,6];
(%o16)                              [4, 5, 6]
\end{verbatim}
\normalsize
%\newpage
\subsubsection{Using disp, display, and print}
We have seen above examples of using \textbf{disp}, which can be used to print out
  the values of symbols or text, and \textbf{display}, which can be used to print
  out the name of the symbol and its value in the form of an equation:\\ "x = value".\\
Here is the do loop check of the roots of the quadratic found above using \textbf{print}
  instead of \textbf{disp}.\\
However, we need to be careful, because we are using a function \verb|f(x)| rather
  than an expression.
We have just assigned the values of \verb|a|, \verb|b|, and \verb|c|, and we
  want \verb|f(x)| to have arbitrary values of these parameters.
\small
\begin{verbatim}
(%i17) [a,b,c];
(%o17)                             [4, 5, 6]
(%i18) f(x);
                                   2
(%o18)                          4 x  + 5 x + 6
(%i19) kill(a,b,c);
(%o19)                               done
(%i20) [a,b,c];
(%o20)                             [a, b, c]
(%i21) f(x);
                                   2
(%o21)                          a x  + b x + c
\end{verbatim}
\newpage
\begin{verbatim}
(%i22) sol;
                         2                         2
                   sqrt(b  - 4 a c) + b      sqrt(b  - 4 a c) - b
(%o22)      [x = - --------------------, x = --------------------]
                           2 a                       2 a
(%i23) for i:1 thru 2 do print("expr = ", expand( subst(sol[i],f(x) ) ) )$
expr =  0 
expr =  0 
\end{verbatim}
\normalsize
Here we use \textbf{disp} to display a title for the do loop:
\small
\begin{verbatim}
(%i24) ( disp("check roots"), for i thru 2 do
               print("expr = ", expand( subst( sol[i],f(x) ) ) )  )$
                                  check roots

expr =  0 
expr =  0 
\end{verbatim}
\normalsize
The only tricky thing about this kind of code is getting the parentheses to balance.
Note that that \textbf{expand(...)} is inside \textbf{print}, so the syntax
  used is \textbf{do print(... )}, ie., a "one job do ".
The outside parentheses allow the syntax \verb|( job1, job2 )|.
Note also that the default start of the do loop index is "1", so we can use an abbreviated
 syntax that does not have the \verb|i:1| beginning.

 
 \subsubsection{Checking Solutions using map}
One advantage of using a function \verb|f(x)| defined via \verb|:=| as the first
  argument to \textbf{solve} is that it is fairly easy to check the roots
  by using the \textbf{map} function.
We want to use the syntax \verb|map( f, solnlist}|, where \verb|solnlist| is
  a list of the roots (not a list of replacement rules).
To get the solution list we can again use \textbf{map} with the syntax
  \verb|map(rhs, sol}|.
\small
\begin{verbatim}
(%i25) solnlist : map( rhs, sol );
                         2                     2
                   sqrt(b  - 4 a c) + b  sqrt(b  - 4 a c) - b
(%o25)          [- --------------------, --------------------]
                           2 a                   2 a
(%i26) map( f, solnlist );
               2              2            2
        (sqrt(b  - 4 a c) + b)    b (sqrt(b  - 4 a c) + b)
(%o26) [----------------------- - ------------------------ + c, 
                  4 a                       2 a
                               2              2            2
                        (sqrt(b  - 4 a c) - b)    b (sqrt(b  - 4 a c) - b)
                        ----------------------- + ------------------------ + c]
                                  4 a                       2 a
(%i27) expand(%);
(%o27)                              [0, 0]
(%i28) expand( map(f, map(rhs, sol) ) );
(%o28)                              [0, 0]
\end{verbatim}
\normalsize
  The last input \verb|%i27| shows a compact method which avoids having to name
   the "solnlist" and which also avoids having to look at the intermediate output.
When you see someone's example written in a compact form like this,
  you should realize that the "someone" probably tried out the progression of
  steps one step at a time (just like we did) to see the correct route, and once
  the path to the result has been found, reduced the result to the minimum
  number of steps and names.
Often, one does not know in advance which progression of steps will succeed,
  and one must experiment before finding the "true path".
You should "take apart" the compact code, by reading from the inside out (ie., from 
  right to left), and also try getting the result one step at a time to get
  comfortable with the method and notation being used.

\subsubsection{Psuedo-PostFix Code: \%\%}
An alternative "psuedo-postfix" (ppf) notation can be used which allows one to read
  the line from left to right, following the logical succession of procedures being
   used.
Although this  ppf notation costs more in keystrokes (an extra pair of outside
  parentheses, extra commas, and entry of double percent signs \verb|%%|), the 
  resulting code is usually easier for beginners to follow, and it is easier to
  mentally balance parentheses as well.
As an example, the previous double map check of the roots can be carried out as:
\small
\begin{verbatim}
(%i29) ( map(rhs,sol), map(f,%%), expand(%%) );
(%o29)                               [0, 0]
\end{verbatim}
\normalsize
Note the beginning and ending parentheses for the whole "line" of input, with
  the syntax:\\
  \verb|( job1, job2(%%), job3(%%),... )|.
The system variable \verb|%%| has the manual description (in part):
\small
\begin{quote}
System variable: \verb|%%|\\ 
In compound statements, namely \verb|block|, \verb|lambda|,
  or \verb|(s_1, ..., s_n)|, \verb|%%| is the value of the previous statement.
\end{quote}
\normalsize

  
\subsubsection{Using an Expression Rather than a Function with Solve}
Let's rework the general quadratic equation solution, including the checks of the
  solutions, using an expression \verb|ex| rather than a function \verb|f(x)|
  defined using \verb|:=|.
\small
\begin{verbatim}
(%i1) ex : a*x^2 + b*x + c$
(%i2) sol : solve( ex, x );
                         2                         2
                   sqrt(b  - 4 a c) + b      sqrt(b  - 4 a c) - b
(%o2)       [x = - --------------------, x = --------------------]
                           2 a                       2 a
(%i3) s1 : sol[1];
                                      2
                                sqrt(b  - 4 a c) + b
(%o3)                     x = - --------------------
                                        2 a
(%i4) r1 : subst(s1, ex );
                   2              2            2
            (sqrt(b  - 4 a c) + b)    b (sqrt(b  - 4 a c) + b)
(%o4)       ----------------------- - ------------------------ + c
                      4 a                       2 a
(%i5) expand(r1);
(%o5)                                  0
(%i6) for i:1 thru 2 do disp( expand( subst( sol[i], ex ) ) )$
                                       0
                                       0							   
\end{verbatim}
\normalsize
(We could have also used \textbf{ev} instead of \textbf{subst}.)
\newpage
Thus far, the methods have been similar.
If we now bind the values of \verb|(a,b,c)|, as we did in the middle of our solutions
  using \verb|f(x)|, what  is the difference?
\small
\begin{verbatim}
(%i7) [a,b,c] : [1,2,3]$
(%i8) [a,b,c];
(%o8)                              [1, 2, 3]
(%i9) ex;
                                   2
(%o9)                           a x  + b x + c
\end{verbatim}
\normalsize
We see that the symbol \verb|ex| remains bound to the same general expression.
The symbol \verb|ex| retains its original binding.
We can make use of the values given to \verb|(a,b,c)| with the expression \verb|ex|
  by using two single quotes, which forces an extra evaluation of the expression \verb|ex|
  by the Maxima engine, and which then makes use of the extra information about \verb|(a,b,c)|.
\small
\begin{verbatim}
(%i10) ''ex;
                                  2
(%o10)                           x  + 2 x + 3
(%i11) ex;
                                   2
(%o11)                          a x  + b x + c
\end{verbatim}
\normalsize
Forcing the extra evaluation in \verb|%10| does not change the binding of \verb|ex|.
Now let's try to check the solutions using \textbf{map}, as we did before.
To use \textbf{map} we need a function, rather than an expression to map on a solution list.
%\newpage
Let's try to define such a function \verb|f(x)| using the expression \verb|ex|.
\small
\begin{verbatim}
(%i12) f(x);
(%o12)                               f(x)
(%i13) f(x) := ex;
(%o13)                            f(x) := ex
(%i14) f(y);
                                   2
(%o14)                          a x  + b x + c
(%i15) f(x) := ''ex;
                                       2
(%o15)                      f(x) := a x  + b x + c
(%i16) f(y);
                                  2
(%o16)                           y  + 2 y + 3
(%i17) kill(a,b,c);
(%o17)                               done
(%i18) f(y);
                                   2
(%o18)                          a y  + b y + c
(%i19) solnlist : map(rhs,sol);
                         2                     2
                   sqrt(b  - 4 a c) + b  sqrt(b  - 4 a c) - b
(%o19)          [- --------------------, --------------------]
                           2 a                   2 a
\end{verbatim}
\newpage
\begin{verbatim}
(%i20) map(f,solnlist);
               2              2            2
        (sqrt(b  - 4 a c) + b)    b (sqrt(b  - 4 a c) + b)
(%o20) [----------------------- - ------------------------ + c, 
                  4 a                       2 a
                               2              2            2
                        (sqrt(b  - 4 a c) - b)    b (sqrt(b  - 4 a c) - b)
                        ----------------------- + ------------------------ + c]
                                  4 a                       2 a
(%i21) expand(%);
(%o21)                              [0, 0]
\end{verbatim}
\normalsize
Output \verb|%14| showed that the syntax \verb|f(x) := ex| did not succeed in defining
  the function we need.
The input \verb|f(x) := ''ex| suceeded in getting a true function of \verb|x|, but
  now the function \verb|f(x)| automatically makes use of the current binding of 
   \verb|(a,b,c)|, so we had to \textbf{kill} those values to get a function with
   arbitrary values of \verb|(a,b,c)|.
Having the function in hand, we again used the \textbf{map} function twice to check the
  solutions.
Now that we have discovered the "true path", we can restart Maxima and present the method as:
\small
\begin{verbatim}
(%i1) ex : a*x^2 + b*x + c$
(%i2) sol : solve( ex, x );
                         2                         2
                   sqrt(b  - 4 a c) + b      sqrt(b  - 4 a c) - b
(%o2)       [x = - --------------------, x = --------------------]
                           2 a                       2 a
(%i3) f(x) := ''ex$
(%i4) expand ( map(f, map(rhs, sol) ) ); 
(%o4)                               [0, 0]
\end{verbatim}
\normalsize
We can also use the (generally safer) syntax \verb|define( f(x),ex );| to
  obtain a true function of x:
\small
\begin{verbatim}
(%i5) define( f(x),ex );
                                       2
(%o5)                       f(x) := a x  + b x + c
(%i6) f(y);
                                   2
(%o6)                           a y  + b y + c
(%i7) expand ( map(f, map(rhs, sol) ) );
(%o7)                               [0, 0]
\end{verbatim}
\normalsize
  
%\newpage
We can also use the unnamed, anonymous function \textbf{lambda} to avoid 
  introducing needless names, like "\verb|f|":
\small
\begin{verbatim}
(%i8) expand( map(lambda([x],''ex), map(rhs,sol) ) );
(%o8)                               [0, 0]
\end{verbatim}
\normalsize

  
  
 \subsubsection{Escape Speed from the Earth}
In this section we solve a physics problem which involves a simple quadratic equation.
It is so simple that "doing it" on paper is faster than doing it with Maxima.
In fact, once you understand the plan of the calculation, you can come up with the final
  formula for the escape speed in your head.
However, we will present practical details of setup and evaluation which can be used with
  more messy problems, when you might want to use Maxima.\\
  
Let's use conservation of mechanical energy (kinetic plus potential) to first calculate the 
  initial radial speed a rocket must have near the surface of the earth to achieve a final
  required radial speed far from the earth (far enough away so we can neglect earth's gravitational
  pull).\\
Let the mass of the rocket be \verb|m|, the mass of the earth be \verb|M|, the radius of
  the earth be \verb|R|, a general radial distance from the center of the
  earth be \verb|r >= R|, a general radial rocket speed be \verb|v|, the maximum speed of
  the rocket near the surface of the earth be \verb|v0|, and the final radial speed of
  the rocket (as \verb|r| becomes infinite) be \verb|vf|.
  
At a general distance \verb|r| from the center of the earth the rocket has kinetic
  energy \verb|ke = m*v^2/2|, and gravitational energy \verb|pe = -G*M*m/r|, where
  \verb|G| is the gravitational constant:\\
  (\verb|G = 6.673 10^(-11) newton*meter^2/kg^2|).
\small
\begin{verbatim}
(%i1) energy : m*v^2/2 - G*M*m/r;
                                    2
                                 m v    m G M
(%o1)                            ---- - -----
                                  2       r
\end{verbatim}
\normalsize
The initial energy \verb|e0| corresponds to the energy the rocket has achieved at the
  moment of maximum radial speed: this will occur at a radius \verb|r| slightly larger
  than the radius of the earth \verb|R|, but negligible error to the required "lift-off speed" \verb|v0| 
  will be made by ignoring this difference in radius. (You can justify this as a good approximation
  by getting the answer when including this small difference, and comparing the percent difference
  in the answers.)
\small
\begin{verbatim}					  
(%i2) e0 : energy,v=v0,r=R;
                                     2
                                 m v0    m G M
(%o2)                            ----- - -----
                                   2       R
\end{verbatim}
\normalsize
As the rocket "rises", \verb|r| becomes larger, and the magnitude of the gravitational
energy becomes smaller.
The "final" energy \verb|efinal| will be the energy when the gravitational energy is so small
  that we can ignore it; in practice this will occur when the magnitude of the gravitational
  energy is much smaller than the magnitude of the inital gravitational energy.
The radial outward speed of the rocket then remains a constant value \verb|vf|.
\small
\begin{verbatim}
								   
(%i3) efinal : limit(energy,r,inf),v=vf;
                                         2
                                     m vf
(%o3)                                -----
                                       2
\end{verbatim}
\normalsize
If we neglect the loss of mechanical energy due to friction in leaving the earth's atmosphere,
  and also neglect other tiny effects like the gravitational interaction between the moon and
  the rocket, the sun and the rocket, etc, then we can approximately say that the total
  mechanical energy (as we have defined it) of the rocket is a constant, once chemical energy
  used to increase the rocket's speed is no longer a factor (which occurs at the moment
  of maximum radial speed).\\
We can then get one equation by approximately equating the mechanical energy of the
  rocket just after achieving maximum speed to the mechanical energy of the rocket when 
  \verb|r| is so large that we can ignore the instantaneous gravitational energy contribution.
\small
\begin{verbatim}									   
(%i4) v0soln : solve(efinal = e0,v0);
                           2 G M     2             2 G M     2
(%o4)         [v0 = - sqrt(----- + vf ), v0 = sqrt(----- + vf )]
                             R                       R
\end{verbatim}
\newpage
\begin{verbatim}
(%i5) v0soln : v0soln[2];
                                      2 G M     2
(%o5)                       v0 = sqrt(----- + vf )
                                        R
(%i6) v0;
(%o6)                                 v0
(%i7) v0 : rhs( v0soln );
                                    2 G M     2
(%o7)                          sqrt(----- + vf )
                                      R
\end{verbatim}
\normalsize
This provides the required "lift-off" speed near the surface of the earth to achieve 
  a given final radial speed \verb|vf|. 
We now want to find the "escape speed", the minimum value of the lift-off speed which
  will allow the rocket to escape the gravitational pull of the earth.
Any rocket which has a radial speed when \verb|r| is effectively infinite will succeed
  in escaping, no matter how small that radial speed is. 
The limiting initial speed is then gotten by taking the limit of \verb|v0| as \verb|vf|
  goes to zero.
\small
\begin{verbatim}								  
(%i8) vescape : ev( v0, vf = 0 );
                                            G M
(%o8)                          sqrt(2) sqrt(---)
                                             R
\end{verbatim}
\normalsize
%\newpage

Uing the mass of the earth, \verb|M = 5.974 10^(24) kg|, and the radius of the earth,\\
  \verb|R = 6.378 10^(6) meters|, we get an escape speed \verb|11,181 m/s = 11.18 km/s|.
\small
\begin{verbatim}
(%i09) ev( vescape, [G=6.673e-11,M=5.974e24,R=6.378e6] );
(%o09)                     7905.892670345354 sqrt(2)
(%i10) float(%);
(%o10)                         11180.62063706845
\end{verbatim}
\normalsize
We have rounded the answer to four significant figures, since that is the accuracy of
  the Earth data and gravitational constant we have used.  \\
%\newpage
If we need to use a set of physical or numerical constants throughout a session,
  we can define a list of "equalities", say \verb|clist|, and use as follows:
  \small
  \begin{verbatim}
(%i11) clist : [G=6.673e-11,M=5.974e24,R=6.378e6];
(%o11) [G = 6.6729999999999999E-11, M = 5.9740000000000004E+24, R = 6378000.0]
(%i12) ev( vescape, clist, float );
(%o12)                          11180.62063706845 
\end{verbatim}
\normalsize
Note that adding the option variable \textbf{float} as a switch (equivalent to
  \verb|float:true|), gets the \verb|sqrt(2)| in floating point.\\
  
Looking at all those digits is unnecessary; set \textbf{fpprintprec} to something
  reasonable (this only affects the numbers presented on the screen, not the
  accuracy of the calculation):
\small
\begin{verbatim}
(%i13) fpprintprec:8$
(%i14) ev( vescape, clist, float );
(%o14)                              11180.621
(%i15) clist;
(%o15)            [G = 6.673E-11, M = 5.974E+24, R = 6378000.0] 
  \end{verbatim}
  \normalsize
  
 
 
 
 
 
 \subsubsection{Cubic Equation or Expression}
Here is an example of using \textbf{solve} to "solve" a cubic equation, or, in the
  alternative language, find the roots of a cubic expression.
After checking the roots via the \textbf{map} function, we assign the
  values of the roots to the symbols\verb|(x1,x2,x3)|.
The cubic expression we choose is especially simple, with no arbitrary parameters,
  so we can use the one argument form of \textbf{solve}.  
\small
\begin{verbatim}
(%i1) ex : x^3 + x^2 + x$
(%i2) sol : solve(ex);
                      sqrt(3) %i + 1      sqrt(3) %i - 1
(%o2)          [x = - --------------, x = --------------, x = 0]
                            2                   2
(%i3) define( f(x), ex )$
(%i4) expand ( map(f, map(rhs, sol) ) );
(%o4)                              [0, 0, 0]
(%i5) [x1,x2,x3] : map(rhs,sol);
                        sqrt(3) %i + 1  sqrt(3) %i - 1
(%o5)                [- --------------, --------------, 0]
                              2               2
(%i6) x1;
                                 sqrt(3) %i + 1
(%o6)                          - --------------
                                       2
\end{verbatim}
\normalsize

\subsubsection{Trigonometric Equation or Expression}
Here is an exact solution using \textbf{solve}:
\small
\begin{verbatim}
(%i1) [fpprintprec:8,display2d:false]$
(%i2) ex : sin(x)^2 -2*sin(x) -3$
(%i3) sol : solve(ex);
`solve' is using arc-trig functions to get a solution.
Some solutions will be lost.
(%o3) [x = asin(3),x = -%pi/2]
(%i4) define( f(x), ex )$ 
(%i5) expand ( map(f, map(rhs, sol) ) );
(%o5) [0,0]
(%i6) numroots : float( map(rhs, sol) );
(%o6) [1.5707963-1.7627472*%i,-1.5707963]
\end{verbatim}
\normalsize
The first solution returned is the angle (in radians) whose sin is 3. 
For real \verb|x|, \verb|sin(x)| lies in the range \verb|-1 <=  sin(x) <= 1 |.
Thus we have found one real root.
But we have been warned that some solutions will be lost.
Because the given expression is a polynomial in \verb|sin(x)|, we can use \textbf{realroots}:
\small
\begin{verbatim}
(%i7) rr : realroots(ex);
(%o7) [sin(x) = -1,sin(x) = 3]
\end{verbatim}
\normalsize
However, by "realroots", \textbf{realroots} means that the numbers \verb|[-1,3]| are real!
\newpage
We can of course take the output of \textbf{realroots} and let \textbf{solve} go to
  work.
\small
\begin{verbatim}
(%i8) map(solve, rr);
`solve' is using arc-trig functions to get a solution.
Some solutions will be lost.

`solve' is using arc-trig functions to get a solution.
Some solutions will be lost.
(%o8) [[x = -%pi/2],[x = asin(3)]]
\end{verbatim}
\normalsize

We know that the numerical value of the expression \verb|ex3| repeats when \verb|x| is
  replaced by \verb|x + 2*%pi|, so there are an infinite number of real roots, 
  related to $-\pi/2$ by adding or subtracting $2 n \pi$, where $n$ is an integer.
%\newpage

We can make a simple plot of our expression to see the periodic behavior and the
  approximate location of the real roots.
  
\smallskip
%%\begin{figure}[!ht] 
\begin{figure} [h]
   \centerline{\includegraphics[width=5in]{root1.eps} }
	\caption{plot of ex3}
\end{figure}
  
We used the \textbf{plot2d} code:
\small
\begin{verbatim}
(%i18) plot2d([0.0,ex3],[x,-6,6],[y,-5,5] )$
\end{verbatim}
\normalsize



\subsubsection{Equation or Expression Containing Logarithmic Functions}
Here is an example submitted to the Maxima mailing list and a method of solution
  provided by Maxima developer Stavros Macrakis.
The problem is to find the roots of the following expression \verb|ex|:
\small
\begin{verbatim}
(%i1) [fpprintprec:8,display2d:false,ratprint:false]$
(%i2) ex : log(0.25*(2*x+5)) - 0.5*log(5*x - 5)$
\end{verbatim}
\normalsize
%\newpage
We first try \textbf{solve}, with the option variable \textbf{solveradcan} set
  equal to \textbf{true}.
Remember that the syntax\\
 \verb|func, optvar;| is equivalent to \verb|func, optvar:true;|.
\small
\begin{verbatim}
(%i3) sol : solve(ex,x),solveradcan;
(%o3) [log((2*x+5)/4) = log(5*x-5)/2]
\end{verbatim}
\normalsize
We see that \textbf{solve} tried to find a "simpler" form which it returned in \verb|%o2|.
The Maxima function \textbf{fullratsimp} has the manual description
\small
\begin{quote}
Function: \textbf{fullratsimp}\verb|(expr) |\\
\verb|fullratsimp| repeatedly applies \verb|ratsimp| followed by non-rational simplification
  to an expression until no further change occurs, and returns the result. \\
When non-rational expressions are involved, one call to \verb|ratsimp| followed as is
   usual by non-rational ("general") simplification may not be sufficient
   to return a simplified result.
Sometimes, more than one such call may be necessary.
\verb|fullratsimp| makes this process convenient. 
\end{quote}
\normalsize
The effect of \textbf{fullratsimp} in our case results in the decimals being replaced by exact
  fractions.
\small
\begin{verbatim}			
(%i4) ex : fullratsimp(ex);
(%o4) (2*log((2*x+5)/4)-log(5*x-5))/2
\end{verbatim}
\normalsize
The logarithms can be combined using the Maxima function \textbf{logcontract}.
This function was discussed in Chapter 2, Sect. 2.3.5.
A partial description is:
\small
\begin{quote}
Function: \textbf{logcontract}\verb|(expr) |\\
Recursively scans the expression \verb|expr|, transforming subexpressions
  of the form\\ \verb|a1*log(b1) + a2*log(b2) + c  | into \verb|log(ratsimp(b1^a1 * b2^a2)) + c | 
\begin{verbatim} 
(%i1) 2*(a*log(x) + 2*a*log(y))$
(%i2) logcontract(%);
                                 2  4
(%o2)                     a log(x  y )
\end{verbatim} 
\end{quote}
\normalsize
%\newpage
Here is the application to our problem:
\small
\begin{verbatim}
(%i5) ex : logcontract(ex);
(%o5) -log((80*x-80)/(4*x^2+20*x+25))/2
\end{verbatim}
\normalsize
Having combined the logarithms, we try out \textbf{solve} on this expression:
\small
\begin{verbatim}
(%i6) sol : solve(ex );
(%o6) [x = -(2*sqrt(30)-15)/2,x = (2*sqrt(30)+15)/2]
\end{verbatim}
\normalsize
We have a successful exact solution, but \textbf{solve} (in its present incarnation) needed
  some help.
We now use the \textbf{map} method to check the roots.
\small
\begin{verbatim}
(%i7) define( f(x), ex )$ 
(%i8) expand( map(f, map(rhs,sol) ) );
(%o8) [0,0]
\end{verbatim}
\normalsize
\subsubsection{Solving for the Nth Roots of ( - 1 )}
% this need editing
%  map 24, 2009
\begin{verbatim}
Maxima's solve function can return a solution in terms of z = (-1)^(1/n), 
which
is any (in general complex) number z such that z^n = -1.

Since complex roots must occur in pairs, if n is even (such as 6) all the 
roots
are complex, and rectform, realpart, and imagpart all pick out what I will 
call
the " k = 0 root", in the sense that the n roots are given by
  exp( %i * %pi * ( 1 + 2*k ) / n ), for k = 0, 1, 2, ..., (n - 1), for 
example.

If solve finds a solution in terms of (-1)^(1/n) for n odd, such as 
(-1)^(1/3),
maxima always simplifies this to (-1), the one real root, which is never the
"k = 0 " choice.

In other words for n odd the user has no choice, but for n even, the user
has a choice.  One such choice is to use that taken by rectform, realpart,
and imagpart.

The danger here, in working with a solution returned by solve which
includes unsimplified n'th roots of (-1), is that if you simplify first
using ratsimp, say, and  then use rectform, your final result may be
different than if you use rectform first, to "lock in" a specific choice.

This is all elementary, but can bite, and cause wasted time if one
does not recognise the danger.

Here is a simple example of this possible source of confusion:

(%i1) display2d:false$

(%i2) z (n) := (-1)^(1/n)$

(%i3) e : z(6) + 1/z(6);

(%o3) (-1)^(1/6)+1/(-1)^(1/6)

(%i4) ratsimp (rectform (e));

(%o4) sqrt(3)

(%i5) rectform (ratsimp (e));

(%o5) 0

(%i6) ratsimp (e);

(%o6) 0
================

and here is simple code for explorations

(%i7) rr (x) := ratsimp (rectform (x))$

(%i8) v(n,k) :=
    block( [fac],
      fac: ratsimp ( (1 + 2*k)/n),
           rr (exp ( %i*%pi*fac )))$

(%i9) vals (n) := for i:0 thru (n-1) do print (i," ",v(n,i))$

(%i10) z(3);
(%o10) -1
(%i11) vals(3)$
0   (sqrt(3)*%i+1)/2
1   -1
2   -(sqrt(3)*%i-1)/2

(%i12) z(4);
(%o12) (-1)^(1/4)

(%i13) rr(%);
(%o13) (sqrt(2)*%i+sqrt(2))/2

(%i14) vals(4)$
0   (sqrt(2)*%i+sqrt(2))/2
1   (%i-1)/sqrt(2)
2   -(sqrt(2)*%i+sqrt(2))/2
3   -(%i-1)/sqrt(2)

(%i15) z(5);
(%o15) -1

(%i16) vals(5)$
0   %i*sin(%pi/5)+cos(%pi/5)
1   %i*sin(3*%pi/5)+cos(3*%pi/5)
2   -1
3   cos(3*%pi/5)-%i*sin(3*%pi/5)
4   cos(%pi/5)-%i*sin(%pi/5)

(%i17) z(6);
(%o17) (-1)^(1/6)

(%i18) rr(%);
(%o18) (%i+sqrt(3))/2

(%i19) vals(6)$
0   (%i+sqrt(3))/2
1   %i
2   (%i-sqrt(3))/2
3   -(%i+sqrt(3))/2
4   -%i
5   -(%i-sqrt(3))/2

==================
Ted Woollett
p.s.  ratsimp is not bad and rectform is not bad ;)

===============================================
Message: 7
Date: Wed, 20 May 2009 17:21:16 -0700
From: Richard Fateman <fateman@cs.berkeley.edu>
Subject: Re: [Maxima] ratsimp bad, rectform good
To: Edwin Woollett <woollett@charter.net>
Cc: Richard Fateman <fateman@EECS.Berkeley.EDU>, maxima mailing list
<maxima@math.utexas.edu>
Message-ID: <4A149E7C.9030503@cs.berkeley.edu>
Content-Type: text/plain; charset=ISO-8859-1; format=flowed

Edwin Woollett wrote:
> On  19 May 2009, Richard Fateman  wrote:
> ---------------------------------------------------
> May be related to your starting expression being ambiguous. There are 6
> values for (-1)^(1/6).
> There are as many as 36 values for the expression.  (Though probably
> fewer distinct ones).
> -------------------------------------------------------
> Maxima's solve function can return a solution in terms of z = 
> (-1)^(1/n), which
> is any (in general complex) number z such that z^n = -1.
Actually, I think the expression (-1)^(1/6) in Maxima is not "any" 
number z such that z^6=-1.  It is a particular one.
Just as 3^(1/2) is  only the positive value.  This is a notational 
issue, and Maxima is not, generally, able to deal with multiple values.

however,
algebraic:true;
tellrat(a^6+1);
rat(a+1/a)  give the result a^5-a.

now there are 6 possible values for a in the complex plane.
a=exp(i pi/ 6) can be put in to a^5-a  to get sqrt(3).
a=exp(5 i pi/6) gets -sqrt(3)
a=exp(9 i pi/6) gets  0.

The danger you refer to below is caused by taking a particular value for 
the root  (instead of the collection of all the roots)
and giving it a name like (-1)^(1/6).   The fact that rectform locks in 
a specific root means that it may pick one that
is different, in related expressions.  e.g.  a+1/a   are the two a's the 
same??

Certainly one way to commit an error is to take one solution from solve 
and simplify it in isolation from the other solutions.

Example.

You don't need 6th roots.
Take the solution of a quadratic, which is the two roots
(-b+sqrt(b^2-4ac))/(2a)  and
(-b-sqrt(...))

Now you know that there are 2 sqrts of b^2-4ac   unless the b^2-4ac is 
zero. Assume it is not zero..
If you choose root r for the first formula and choose  root -r for the 
second formula, you no longer have a
complete solution set.

How to fix all this?  You can be rigorous in dealing with algebraic 
numbers (in which case you are requiring users of Maxima
to have taken a college course in Modern Algebra).  Or you can insist on 
multiple-values e.g. + and - sqrts.
Or you can punt.
Maxima, along with other CAS, punts.  Unless you use tellrat and friends.

RJF

to mailing list may 23, 2009:
solve and nth roots of ( -1 )

On  May 19, 2009, on thread "ratsimp bad,  rectform good",
  Edwin Woollett wrote:
    " Maxima's solve function can return a solution in terms of z = 
        (-1)^(1/n), which  is any (in general complex) number z   
        such that z^n = -1. "

On May 20, 2009, on the same thread, Richard Fateman wrote:
  "  Actually, I think the expression (-1)^(1/6) in Maxima is not "any" 
         number z such that z^6=-1.  It is a particular one.
        Just as 3^(1/2) is  only the positive value.  This is a notational 
        issue, and Maxima is not, generally, able to deal with
         multiple values."

Let's look at the 4th roots of ( - 1 ) as an example.

 solve ( a^4 + 1) returns a list of four answers, each
  with a  factor  ( - 1 )^(1/4).  Since the latter factor
 can be four possible complex numbers, one might
 think that maxima is returning more than four  answers.

For some reason, solve offers the user the choice of
replacing  ( - 1 )^(1/4)  (using ratsubst, say) throughout 
the list with a single choice and that will result in a list of
four complex numbers.

For any user choice, you get the same list of four
complex numbers, just in a different (cyclic) order.

Here is code for exploration of the case n = 4.

(%i1) display2d:false$

   usual definition of 4th roots of (-1), with k = 1, 2, 3, 4.

(%i2) v(k):= factor (ratsimp ( 
                       rectform (exp( %i*%pi*( -1 + 2*k)/4))))$

   display these  as a simplified list:

(%i3) q: fullmap ('rootscontract, makelist ( v(k), k, 1, 4));

(%o3) [ 2^-(1/2)*(%i+1),  (%i-1)/sqrt(2),
                  -2^-(1/2)*(%i+1),  -(%i-1)/sqrt(2) ]
 
  check roots:

(%i4) ratsimp ( q^4 );
(%o4) [-1,-1,-1,-1]

   Now ask solve for the 4th roots of (-1):

(%i5) s : factor (map ( 'rhs, solve ( a^4 + 1 )));

(%o5) [ (-1)^(1/4)*%i,   -(-1)^(1/4),   
                        -(-1)^(1/4)*%i,   (-1)^(1/4)   ]

   note presence of factor (-1)^(1/4) in each element

check solutions

(%i6) ratsimp(s^4);
(%o6) [-1,-1,-1,-1]

   define four particular solutions p[k] by replacing (-1)^(1/4) by
              q[k] defined above
 
(%i7) for k thru 4 do 
                  p[k] : factor (ratsubst ( q[k], (-1)^(1/4), s ))$

   show k,  p[k],  p[k]^4, and finally that p[k]
    is really the same as the list q defined above, but just
    in a different order.
  
(%i8) for k thru 4 do
        (print("   ", k,"  ",  p[k]),
           print("        ", ratsimp ( p[k]^4 ),
             "  ",  ratsimp ( sort ( p[k] ) - sort ( q ))))$

    1    [ (%i-1)/sqrt(2), -(%i+1)/sqrt(2),  
                    -(%i-1)/sqrt(2),  (%i+1)/sqrt(2) ] 
                   [-1,-1,-1,-1]    [0,0,0,0] 

    2    [ -(%i+1)/sqrt(2),  -(%i-1)/sqrt(2),  
                         (%i+1)/sqrt(2),  (%i-1)/sqrt(2) ] 
                    [-1,-1,-1,-1]    [0,0,0,0] 

    3    [ -(%i-1)/sqrt(2),   (%i+1)/sqrt(2), 
                       (%i-1)/sqrt(2),   -(%i+1)/sqrt(2)] 
                   [-1,-1,-1,-1]    [0,0,0,0] 

    4    [ (%i+1)/sqrt(2),   (%i-1)/sqrt(2),  
                       -(%i+1)/sqrt(2),   -(%i-1)/sqrt(2) ] 
                   [-1,-1,-1,-1]    [0,0,0,0] 
         
    Finally,  let pr be the list of 4th roots if we just rectform the
     list of solutions returned by solve, thus using maxima's default
     choice for ( -1 )^( 1/n ) for n even.
  
(%i9) pr : factor ( rectform (s) );

(%o9) [ (%i-1)/sqrt(2),   -(%i+1)/sqrt(2),  
                     -(%i-1)/sqrt(2),   (%i+1)/sqrt(2) ]

   check solution
(%i10) ratsimp ( pr^4 );
(%o10) [-1,-1,-1,-1]

   show it is just the particular solution p[1]:

(%i11) ratsimp ( pr - p[1] );
(%o11) [0,0,0,0]

=============================
A similar exploration can be made for any even n
with obvious changes.

Ted Woollett


added, 2nd mail:

show we can multiply by -1, %i, or -%i
and get same list of four complex numbers.

(%i12) pr1 : -pr;
(%o12) [-(%i-1)/sqrt(2),(%i+1)/sqrt(2),(%i-1)/sqrt(2),-(%i+1)/sqrt(2)]
(%i13) ratsimp( pr1^4 );
(%o13) [-1,-1,-1,-1]
(%i14) pr2 : ratsimp(-%i*pr);
(%o14) [(%i+1)/sqrt(2),(%i-1)/sqrt(2),-(%i+1)/sqrt(2),-(%i-1)/sqrt(2)]
(%i15) ratsimp( sort(pr2) - sort(pr1));
(%o15) [0,0,0,0]
(%i16) ratsimp( pr2^4 );
(%o16) [-1,-1,-1,-1]
(%i17) pr3 : ratsimp(%i*pr);
(%o17) [-(%i+1)/sqrt(2),-(%i-1)/sqrt(2),(%i+1)/sqrt(2),(%i-1)/sqrt(2)]
(%i18) ratsimp(pr3^4);
(%o18) [-1,-1,-1,-1]
(%i19) ratsimp( sort(pr3) - sort(pr1));
(%o19) [0,0,0,0]
(%i20) 
==============
from fateman may 23

Edwin Woollett wrote:

> On  May 19, 2009, on thread "ratsimp bad,  rectform good",
>   Edwin Woollett wrote:
>     " Maxima's solve function can return a solution in terms of z = 
>         (-1)^(1/n), which  is any (in general complex) number z   
>         such that z^n = -1. "
>
> On May 20, 2009, on the same thread, Richard Fateman wrote:
>   "  Actually, I think the expression (-1)^(1/6) in Maxima is not "any" 
>          number z such that z^6=-1.  It is a particular one.
>         Just as 3^(1/2) is  only the positive value.  This is a notational 
>         issue, and Maxima is not, generally, able to deal with
>          multiple values."
>
> Let's look at the 4th roots of ( - 1 ) as an example.
>
>  solve ( a^4 + 1) returns a list of four answers, each
>   with a  factor  ( - 1 )^(1/4).  Since the latter factor
>  can be four possible complex numbers, one might
>  think that maxima is returning more than four  answers.
>   

You might think this, but of course you would be wrong, since a 
polynomial of degree 4 has exactly 4 roots.
There is a fundamental challenge to the correctness of the result of 
solve if you separate the 4 answers and treat them as though each one 
can be further bifurcated into 4 separate results regardless of the 
other 3 answers that come from solve.

> For some reason, solve offers the user the choice of
> replacing  ( - 1 )^(1/4)  (using ratsubst, say) throughout 
> the list with a single choice and that will result in a list of
> four complex numbers.
>   
That seems like an OK thing to do, to me.
> For any user choice, you get the same list of four
> complex numbers, just in a different (cyclic) order.
>   
Exactly why it is OK to do it.

 I am not sure I understand what you are doing in the rest of the message. 

It is not adequate to determine that you have 4 distinct roots of a 
quartic by taking each of the roots, inserting it into the quartic, and 
simplifying to zero.
All that tells you is that you have 4 expressions, each of which might 
be equivalent to some root, but maybe not the 4 distinct ones.

  How could this be?  It is well known that 1/sqrt(2) and 2/sqrt(2) are 
equal, but not identical.  If you posed them as the two roots of the 
equation x^2-2=0,  what would your exploration say?

Let me try to be clear:  If you use a single expression that includes 
(-1)^(1/4),  you are asking for trouble.  If solve produces an 
expression that includes (-1)^(1/4), it probably also includes 3 more 
expressions with that form.  That's sort of OK, if you keep the 4 
expressions together as a set of solutions.

Or you can pick a particular solution (in the complex plane) and then 
instead of having the 4 solutions that circulate around, you have fixed 
them all in place. But then you should remove (-1)^(1/4) in all the 
roots in the solution set.


================

issue for new users is a safe path to the four unique
complex roots of (-1)^(1/4), in a solve return
list.

  z :  (-1)^(1/4) ;
  zr : rectform(z) ;
  soln : ratsubst( zr, z, sollist );
  
  


\end{verbatim}
  
\subsection{One Equation Numerical Solutions: allroots, realroots, find\_root}
We have already tried out the Maxima functions \textbf{realroots} and \textbf{allroots}.
The most important restriction for both of these numerical methods is that the expression
  or equation be a polynomial, as the manual explains:
\small
\begin{quote}
 -- Function: \textbf{realroots}\verb|(eqn, bound)|\\
 -- Function: \textbf{realroots}\verb|(expr)|\\
 -- Function: \textbf{realroots}\verb|(eqn)|\\ 
 Computes rational approximations of the real roots of the polynomial \verb|expr|
    or polynomial equation \verb|eqn| of one variable, to within a tolerance of \verb|bound|.
Coefficients of \verb|expr| or \verb|eqn| must be literal numbers;
   symbol constants such as \verb|%pi| are rejected.\\
\verb|realroots| assigns the multiplicities of the roots it finds to
     the global variable \verb|multiplicities|.\\
\verb|realroots| constructs a Sturm sequence to bracket each root, and
     then applies bisection to refine the approximations.
All coefficients are converted to rational equivalents before
     searching for roots, and computations are carried out by exact
     rational arithmetic.
Even if some coefficients are floating-point numbers, the results are
   rational (unless coerced to floats by the \verb|float| or \verb|numer| flags).\\
When \verb|bound| is less than 1, all integer roots are found exactly.
When \verb|bound| is unspecified, it is assumed equal to the global
     variable \verb|rootsepsilon| (default: \verb|10^(-7)|).\\
When the global variable \verb|programmode| is \verb|true| (default: \verb|true|),
 \verb|realroots| returns a list of the form\\ \verb|[x = <x_1>, x = <x_2>, ...]|.
When \verb|programmode| is \verb|false|, \verb|realroots| creates intermediate
     expression labels \verb|%t1|, \verb|%t2|, ..., assigns the results to them,
     and returns the list of labels.
\end{quote}
\normalsize
Here are the startup values of the option variables just mentioned:
\small
\begin{verbatim}
(%i1) fpprintprec:8$
(%i2) [multiplicities,rootsepsilon,programmode];
(%o2)                     [not_set_yet, 1.0E-7, true]
\end{verbatim}
\normalsize
%\newpage
The function \textbf{allroots} also accepts only polynomials, and finds
  numerical approximations to both real and complex roots:
\small
\begin{quote}
Function: \textbf{allroots}\verb|(expr) |\\
Function: \textbf{allroots}\verb|(eqn) |\\
Computes numerical approximations of the real and complex roots of the
   polynomial \verb|expr| or polynomial equation \verb|eqn| of one variable. \\
The flag \verb|polyfactor| when \verb|true| causes \verb|allroots| to factor the
   polynomial over the real numbers if the polynomial is real, or over the complex
   numbers, if the polynomial is complex (default setting of \verb|polyfactor| is \verb|false|). \\
\verb|allroots| may give inaccurate results in case of multiple roots.\\
If the polynomial is \verb|real|,
 \verb|allroots (%i*p))| may yield more accurate
   approximations than \verb|allroots(p)|, as \verb|allroots| invokes a different
   algorithm in that case. \\
\verb|allroots| rejects non-polynomials.
It requires that the numerator after rat'ing should be a polynomial, and
   it requires that the denominator be at most a complex number.
As a result of this, \verb|allroots| will always return an equivalent
  (but factored) expression, if \verb|polyfactor| is \verb|true|. 
\end{quote}
\normalsize
Here we test the default value of \textbf{polyfactor}:
\small
\begin{verbatim}
(%i3) polyfactor;
(%o3)                                false
\end{verbatim}
\normalsize
\subsubsection{Comparison of realroots with allroots}
Let's find the real and complex roots of a fifth order polynomial which
  \textbf{solve} cannot "solve", doesn't factor, and use both
  \textbf{realroots} and \textbf{allroots}.
\small
\begin{verbatim}
(%i4) ex : x^5 + x^4 -4*x^3 +2*x^2 -3*x -7$
(%i5) define( fex(x), ex )$ 
\end{verbatim}
\normalsize
We first use \textbf{realroots} to find the three real roots of the
  given polynomial, and substitute the roots back into the expression
  to see how close to zero we get.
\small
\begin{verbatim}
(%i6) rr : float( map(rhs, realroots(ex,1e-20) ) );
(%o6)                [- 2.7446324, - 0.880858, 1.7964505]
(%i7) frr : map( fex, rr );
(%o7)                     [0.0, - 4.4408921E-16, 0.0]
\end{verbatim}
\normalsize
\newpage
Next we find numerical approximations to the three real roots and the
  two (complex-conjugate) roots of the given polynomial, using
   \textbf{allroots}\verb|( ex )|  and substitute the obtained roots back into
   the expression to see how close to zero we get.
\small
\begin{verbatim}
(%i8) ar1 : map(rhs, allroots( ex ) );
(%o8) [1.1999598 %i + 0.41452, 0.41452 - 1.1999598 %i, - 0.880858, 1.7964505, 
                                                                   - 2.7446324]
(%i9) far1 : expand( map( fex, ar1 ) );
(%o9) [1.12132525E-14 %i + 4.4408921E-16, 4.4408921E-16 - 1.12132525E-14 %i, 
                            - 1.13242749E-14, 2.48689958E-14, - 2.84217094E-14]
\end{verbatim}
\normalsize
%\newpage
Finally, we repeat the process for the syntax \textbf{allroots}\verb|( %i* ex )|.
\small
\begin{verbatim}
(%i10) ar2 : map(rhs, allroots( %i*ex ) );
(%o10) [1.1999598 %i + 0.41452, - 3.60716392E-17 %i - 0.880858, 
0.41452 - 1.1999598 %i, 6.20555942E-15 %i + 1.7964505, 
- 6.54444294E-16 %i - 2.7446324]
(%i11) far2 : expand( map( fex, ar2 ) );
(%o11) [1.55431223E-15 %i - 1.77635684E-15, 5.61204464E-16 %i - 4.4408921E-16, 
1.61204383E-13 %i + 2.26041408E-13, 2.52718112E-13 %i - 1.91846539E-13, 
- 6.32553289E-14 %i - 3.97903932E-13]
(%i12) far2 - far1;
(%o12) [- 9.65894031E-15 %i - 2.22044605E-15, 
1.1774457E-14 %i - 8.8817842E-16, 1.61204383E-13 %i + 2.37365683E-13, 
2.52718112E-13 %i - 2.16715534E-13, - 6.32553289E-14 %i - 3.69482223E-13]
\end{verbatim}
\normalsize
The three real roots of the given fifth order polynomial are
  found more accurately by \textbf{realroots} than by either
  version of \textbf{allroots}.
We see that the three real roots of this fifth order polynomial are found
  more accurately by the syntax \textbf{allroots}\verb|(expr)|  (which was used
  to get \verb|ar1|), than by the syntax \textbf{allroots}\verb|(%i*expr)|,
  used to get \verb|ar2|.
We also see that the syntax \textbf{allroots}\verb|(%i*expr)| introduced
  a tiny complex piece to the dominant real part.
The two extra complex roots found by the first syntax (\verb|ar1|) are the complex
 conjugate of each other.  
The two extra complex roots found by the alternative syntax (\verb|ar2|) are also
  the complex conjugate of each other to within the default arithmetic accuracy
  being used.
\subsubsection{Intersection Points of Two Polynomials}
Where do the two curves $h(x)=x^3-8\, x^2 + 19\, x -12$ and
  $k(x)=\frac{1}{2}\, x^2 -x - \frac{1}{8}$ intersect?
We want approximate numerical values.
We can plot the two functions together and use the cursor to read off the
  values of \verb|x| for which the curves cross, and we can also find the
  roots numerically.
We first define the curves as expressions depending on \verb|x|, then
  define the difference of the expressions (\verb|rx|) to work with using \textbf{allroots}
  first to see if all the (three) roots are real, and then using \textbf{realroots}
  just for fun, and then checking the solutions.  
If we are just going to compare the results with a plot, we don't need
  any great accuracy, so we will use the default \textbf{realroots} precision.
\small
\begin{verbatim}
(%i1) fpprintprec : 8$
(%i2) hx : x^3 - 8*x^2 + 19*x - 12$
(%i3) kx : x^2/2 -x -1/8$
(%i4) rx : hx - kx;
                                     2
                             3   17 x           95
(%o4)                       x  - ----- + 20 x - --
                                   2            8
\end{verbatim}
\newpage
\begin{verbatim}
(%i5) factor(rx);
                              3       2
                           8 x  - 68 x  + 160 x - 95
(%o5)                      -------------------------
                                       8
(%i6) define( fr(x), rx )$ 
(%i7) allroots(rx);
(%o7)            [x = 0.904363, x = 2.6608754, x = 4.9347613]
(%i8) rr : float( realroots(rx) );
(%o8)            [x = 0.904363, x = 2.6608754, x = 4.9347613]
(%i9) rr : map( rhs, rr);
(%o9)                  [0.904363, 2.6608754, 4.9347613]
(%i10) map(fr, rr);
(%o10)          [2.04101367E-8, - 8.4406409E-8, 5.30320676E-8]
\end{verbatim}
\normalsize
We see that the numerical solutions are "zeros" of the cubic function to within
  the numerical accuracy \textbf{realroots} is using.
Just out of curiosity, what about exact solutions of this cubic polynomial?
Use of \textbf{solve} will generate a complicated looking expression involving
  roots and \verb|%i|.
Let's set \textbf{display2d} to \textbf{false} so the output doesn't take up so much
  room on the screen.
\small
\begin{verbatim}
(%i11) display2d : false$
(%i12) sx : solve(rx);
(%o12) [x = (-sqrt(3)*%i/2-1/2)*(3^-(3/2)*sqrt(16585)*%i/16+151/432)^(1/3)
          +49*(sqrt(3)*%i/2-1/2)/(36*(3^-(3/2)*sqrt(16585)*%i/16+151/432)
                                     ^(1/3))+17/6,
                  etc, etc ]
(%i13) sx1 : map(rhs, sx);
(%o13) [(-sqrt(3)*%i/2-1/2)*(3^-(3/2)*sqrt(16585)*%i/16+151/432)^(1/3)
         +49*(sqrt(3)*%i/2-1/2)/(36*(3^-(3/2)*sqrt(16585)*%i/16+151/432)
                                    ^(1/3))+17/6,
          etc, etc ]
\end{verbatim}
\normalsize
The list \verb|sx1| holds the exact roots of the cubic polynomial which
  \textbf{solve} found.
We see that the form returned has explicit factors of \verb|%i|.
We already know that the roots of this polynomial are purely real.
How can we get the exact roots into a form where it is "obvious" that
  the roots are real?
The Maxima expert Alexey Beshenov (via the Maxima mailing list) suggested
  using \textbf{rectform}, followed by \textbf{trigsimp}.
Using \textbf{rectform} gets rid of the factors of \verb|%i|, and
  \textbf{trigsimp} does some trig simplification. 
\small
\begin{verbatim}
(%i14) sx2 : rectform(sx1);
(%o14) [49*(3*sqrt(3)*sin(atan(432*3^-(3/2)*sqrt(16585)/(151*16))/3)/7
         -3*cos(atan(432*3^-(3/2)*sqrt(16585)/(151*16))/3)/7)/36
         +7*sqrt(3)*sin(atan(432*3^-(3/2)*sqrt(16585)/(151*16))/3)/12
         +%i*(-7*sin(atan(432*3^-(3/2)*sqrt(16585)/(151*16))/3)/12
         +49*(3*sin(atan(432*3^-(3/2)*sqrt(16585)/(151*16))/3)/7
         +3*sqrt(3)*cos(atan(432*3^-(3/2)*sqrt(16585)/(151*16))/3)/7)/36
		 -7*sqrt(3)*cos(atan(432*3^-(3/2)*sqrt(16585)/(151*16))/3)/12)
         -7*cos(atan(432*3^-(3/2)*sqrt(16585)/(151*16))/3)/12+17/6,
          etc,etc   ]
(%i15) sx3 : trigsimp(sx2);
(%o15) [ (7*sqrt(3)*sin(atan(9*sqrt(16585)/(151*sqrt(3)))/3)
         -7*cos(atan(9*sqrt(16585)/(151*sqrt(3)))/3)+17)/6,
        -(7*sqrt(3)*sin(atan(9*sqrt(16585)/(151*sqrt(3)))/3)
         +7*cos(atan(9*sqrt(16585)/(151*sqrt(3)))/3)-17)/6,
		 (14*cos(atan(9*sqrt(16585)/(151*sqrt(3)))/3)+17)/6 ]
\end{verbatim}
\normalsize
The quantity \verb|sx3| is a list of the "simplified" exact roots of
  the cubic.
Using \textbf{float} we ask for the numerical values:
\small
\begin{verbatim}
(%i16) sx4 : float(sx3);
(%o16) [2.6608754,0.904363,4.9347613]
\end{verbatim}
\normalsize
We see that the numerical values agree, although the order of
  the roots is different.
Next we enquire whether or not the "exact roots", when substituted back into the cubic,
  result in "exact zeroes".
Mapping the cubic onto the list of roots doesn't automatically simplify to
  a list of three zeroes, as we would like, although applying \textbf{float}
  suggests the analytic roots are correct.
The combination\\
 \textbf{trigsimp( expand}\verb|( [fr(root1), fr(root2), fr(root3)] ) )|
  still does not provide the algebraic and trig simplification needed, but we
  finally get \verb|[0.0, 0.0, 0.0]| when applying \textbf{float}.
\small
\begin{verbatim}
(%i17) float( map(fr, sx3) );
(%o17) [-3.55271368E-15,1.88737914E-15,-1.42108547E-14]
(%i18) float( expand( map(fr, sx3) ) );
(%o18) [6.66133815E-16,-2.44249065E-15,0.0]
(%i19) float( trigsimp( expand( map(fr, sx3) ) ) );
(%o19) [0.0,0.0,0.0]
\end{verbatim}
\normalsize
Let's next plot the three functions, using the expressions hx, kx, and rx (the difference).

\smallskip
%%\begin{figure}[!ht] 
\begin{figure} [ht]
   \centerline{\includegraphics[width=5in]{cubic1.eps} }
	\caption{Intersection Points are Zeroes of rx}
\end{figure}

%\newpage

Here is code you can use to make something close to the above plot.
\small
\begin{verbatim}
(%i20) plot2d([hx,kx,rx],[x,0,5],
        [style, [lines,2,1], [lines,2,2], [lines,2,0] ],
           [legend, "hx", "kx", "rx=hx - rx"],
        [gnuplot_preamble, " set xzeroaxis lw 2 "])$
\end{verbatim}
\normalsize
When you place the cursor over the places on the \verb|x| axis where the expression
  \verb|rx| is zero, you can read off the coordinates in the lower left corner of the plot
  window.
The \verb|x| coordinate (the first) is the desired root.  
 \newpage

 
 \subsubsection{Transcendental Equations and Roots: \; find\_root}
A transcendental equation is an equation containing a transcendental function.
Examples of such equations are $x=e^{x}$ and $x = \sin(x)$.
The logarithm and the exponential function are examples of transcendental functions.
We will include the trigonometric functions, i.e., sine, cosine, tangent,
  cotangent, secant, and cosecant in this category of functions.
(A function which is not transcendental is said to be algebraic.
Examples of algebraic functions are rational functions and the square root function.)

\smallskip
To find the roots of transcendental expressions, for example, we can first make a
  plot of the expression, and then use \textbf{find\_root} knowing roughly where
  \textbf{find\_root} should start looking.
The Maxima manual provides a lot of details, beginning with:
\small
\begin{quote}
Function: \textbf{find\_root}\verb|(expr, x, a, b) |\\
Function: \textbf{find\_root}\verb|(f, a, b) |\\
Option variable: \textbf{find\_root\_error }\\
Option variable: \textbf{find\_root\_abs }\\
Option variable: \textbf{find\_root\_rel }\\
Finds a root of the expression \verb|expr| or the function \verb|f| over the
   closed interval \verb|[a, b]|.
The expression \verb|expr| may be an equation, in which case \verb|find_root|
  seeks a root of \verb|lhs(expr) - rhs(expr)|.\\
Given that Maxima can evaluate \verb|expr| or \verb|f| over \verb|[a, b]|
   and that \verb|expr| or \verb|f| is continuous, \verb|find_root|
   is guaranteed to find the root, or one of the roots if there is more than one. \\
%\verb|find_root| initially applies binary search.
%If the function in question appears to be smooth enough, \verb|find_root|
%  applies linear interpolation instead. \\
%The accuracy of \verb|find_root| is governed by \verb|find_root_abs|
%   and \verb|find_root_rel|.
%\verb|find_root| stops when the function in question evaluates
%  to something less than or equal to \verb|find_root_abs|, or if successive
%   approximants \verb|x_0, x_1| differ by no more than \verb|find_root_rel * max(abs(x_0), abs(x_1))|.
%The default values of \verb|find_root_abs| and \verb|find_root_rel| are both zero. \\
%\verb|find_root| expects the function in question to have a different sign at
%   the endpoints of the search interval.
%If this condition is not met, the behavior of \verb|find_root| is governed
%   by \verb|find_root_error|.
%When \verb|find_root_error| is \verb|true|, \verb|find_root| prints an error message.
%Otherwise \verb|find_root| returns the value of \verb|find_root_error|.
%The default value of \verb|find_root_error| is \verb|true|. 
\end{quote}
\normalsize
Let's find a root of the equation $x = \cos(x)$ .
If we make a simple plot of the function $x - \cos(x)$, we see that there is one
  root somewhere between $x = 0$ and $x = 1$.
  
\smallskip
%%\begin{figure}[!ht] 
\begin{figure} [h]
   \centerline{\includegraphics[width=5in]{trans1.eps} }
	\caption{Plot of x - cos(x)}
\end{figure}

%\newpage
We can use either an expression or function as the entry to the first slot
  of \textbf{find\_root}.
I find that the most common mistake I make with \textbf{find\_root} is to
  leave out the underline between "find" and "root", in which case, I
  simply get back the unevaluated "findroot(ex, x, 0, 1)", since Maxima
  has no knowledge of "findroot" (unless I create a homemade function with
  that name).
\newpage
We can make a plot, find the root in a variety of ways using \textbf{find\_root},
   and verify the accuracy of the root as follows:  
\small
\begin{verbatim}
(%i1) fpprintprec:8$
(%i2) plot2d( x - cos(x), [ x, 0, 1 ],
              [style,  [lines, 4, 1] ],
               [xlabel," plot of x - cos(x) "],
            [gnuplot_preamble, "set nokey; set xzeroaxis lw 2 "] )$
(%i3) find_root( x - cos(x),x, 0, 1);
(%o3)                              0.739085
(%i4) ex : x - cos(x)$
(%i5) [find_root( ex, x, 0, 1),find_root( ex, 0, 1)];
(%o5)                        [0.739085, 0.739085]
(%i6) define( f(x), ex )$ 
(%i7) [find_root(f(x), x, 0, 1), find_root(f(x), 0, 1),
          find_root(f, 0, 1), find_root(f, x, 0, 1)];
(%o7)             [0.739085, 0.739085, 0.739085, 0.739085]
(%i8) ev(ex, x = first(%) );
(%o8)                                0.0
\end{verbatim}
\normalsize
% eps file plot
% plot2d( [0.0, x - cos(x)],[x,-5,5],
%              [style, [lines,4,4], [lines,4,1] ],
%               [xlabel," plot of x - cos(x) "],
%            [gnuplot_preamble, "set nokey; set yzeroaxis lw 4  "],
%             [gnuplot_term,ps], [gnuplot_out_file,"c:/work2/trans1.eps"])$
%

\smallskip
As a second example, we will find the roots of the function
$$ f(x) = \cos(x/\pi) \, e^{-(x/4)^{2}} - \sin(x^{3/2}) - 5/4 $$

Here is the plot of $f(x)$.

  \smallskip
%%\begin{figure}[!ht] 
\begin{figure} [h]
   \centerline{\includegraphics[width=5in]{trans2.eps} }
	\caption{Plot of f(x)}
\end{figure}

  % eps file plot for second example f(x)			
%(%i5) plot2d( [0.0, f(x)],[x,0,5],
%              [style, [lines,4,4], [lines,4,1] ],
%               [xlabel," plot of f(x) "],
%            [gnuplot_preamble, "set nokey "],
%             [gnuplot_term,ps], [gnuplot_out_file,"c:/work2/trans2.eps"])$
This plot shows roots near the points \verb|[x = 2.53, x = 2.97]|.
\newpage
Here is our session making a plot and using \textbf{find\_root}:
\small
\begin{verbatim}
(%i1) fpprintprec:8$
(%i2) f(x):= cos(x/%pi)*exp(-(x/4)^2) - sin(x^(3/2)) - 5/4$

(%i3) plot2d(  f(x),[x,0,5],
              [style,  [lines,4,1] ],
               [xlabel," plot of f(x) "],[ylabel," "],
            [gnuplot_preamble, "set nokey; set xzeroaxis lw 2 "] )$
(%i4) [find_root(f,2.5,2.6), find_root(f, x, 2.5, 2.6),
          find_root(f(x),x,2.5,2.6), find_root(f(y),y,2.5,2.6)];
(%o4)           [2.5410501, 2.5410501, 2.5410501, 2.5410501]
(%i5) [x1 : find_root(f,2.5,2.6),x2 : find_root(f, 2.9, 3.0 )];
(%o5)                       [2.5410501, 2.9746034]
(%i6) float( map(f, [x1,x2] ) );
(%o6)                  [3.33066907E-16, 2.77555756E-16]
\end{verbatim}
\normalsize
We see that the numerical accuracy of \textbf{find\_root} using default
  behavior is the normal accuracy of Maxima arithmetic.\\

%  
%\newpage
\subsubsection{find\_root: Quote that Function!}
The Maxima function \textbf{find\_root} is an unusual function.
The source code which governs the behavior of \textbf{find\_root} has been
  purposely designed to allow uses like the following:
\small
\begin{verbatim}
(%i1) fpprintprec:8$
(%i2) find_root(diff(cos(2*x)*sin(3*x)/(1+x^2),x), x, 0, 0.5);
(%o2)                              0.321455
\end{verbatim}
\normalsize
\textbf{find\_root} evaluates the derivative and parks the resulting expression in
  the code function "fr(x)",say, to be used to probe where the function changes sign, as the code
   executes a bisection search for a root in the range called for.
The code first checks that the sign of the function fr(x) has opposite signs at the end points
  of the given range.
Next the code makes small steps in one direction, checking at each step if a sign change has
  occurred.
As soon as a sign change has occurred, the code backs up one step, cuts the size of the step in
  half, say, and starts stepping again, looking for that sign change again.
This is a brute force method which will find that root if there is one in the given
  interval.
%\newpage
Of course the user can always evaluate the derivative and submit the resulting expression
  to \textbf{find\_root} to find the same answer, as in:
\small
\begin{verbatim}
(%i3) ex : trigsimp( diff(cos(2*x)*sin(3*x)/(1+x^2),x) );
              2
(%o3) - (((2 x  + 2) sin(2 x) + 2 x cos(2 x)) sin(3 x)
                                      2                           4      2
                              + (- 3 x  - 3) cos(2 x) cos(3 x))/(x  + 2 x  + 1)
(%i4) plot2d([0.0,ex],[x,-3,3])$
(%i5) find_root(ex,x,0,0.5);
(%o5)                              0.321455
\end{verbatim}
\normalsize
\newpage
If we assign the delayed derivative to an expression ex1, we can then
  use an argument \verb|ev(ex1,diff)| to \textbf{find\_root} as in:
\small
\begin{verbatim}
(%i6) ex1 :  'diff(cos(2*x)*sin(3*x)/(1+x^2),x);
                            d   cos(2 x) sin(3 x)
(%o6)                       -- (-----------------)
                            dx        2
                                     x  + 1
(%i7) find_root(ev(ex1,diff),x,0,0.5);
(%o7)                              0.321455
\end{verbatim}
\normalsize
If we assign the delayed derivative to a function g(x), and
  assign \verb|ev(g(x),diff)| to another function k(x), we can
  use \textbf{find\_root} as in:
\small
\begin{verbatim}
(%i8) g(x) := 'diff(cos(2*x)*sin(3*x)/(1+x^2),x);
                                d   cos(2 x) sin(3 x)
(%o8)                   g(x) := -- (-----------------)
                                dx            2
                                         1 + x
(%i9) k(x) := ev(g(x),diff);
(%o9)                       k(x) := ev(g(x), diff)
(%i10) find_root(k(x),x,0,0.5);
(%o10)                             0.321455
\end{verbatim}
\normalsize
or just use the syntax:
\small
\begin{verbatim}
(%i11) find_root( ev(g(x),diff),x,0,0.5 );
(%o11)                             0.321455
\end{verbatim}
\normalsize
However, the following syntax which leaves out the variable name
   produces an error message:
\small
\begin{verbatim}
(%i12) find_root( k, 0, 0.5 );
Non-variable 2nd argument to diff:
0.0
#0: g(x=0.0)
#1: k(x=0.0)
 -- an error.  To debug this try debugmode(true);
\end{verbatim}
\normalsize

In the above cases we want \textbf{find\_root} to use the default initial
  evaluation of the first slot argument before going to work looking for the root.
%\newpage

The important thing to stress is that the \textbf{find\_root} source code, by default, is designed
  to evaluate the first slot expression before beginning the checking of the sign of the resulting
  function (after evaluation) at the end points and proceeding with the bisection search.
If the user wants to use (for some reason) a function f(x) in the first slot of \textbf{find\_root}
  and wishes to prevent the initial evaluation of the first slot expression, the user
  should use the syntax \verb|'(f(x))| for the first slot entry, rather than \verb|f(x)|.
However, the code is loosely written so that most of the time you can get correct results
  without using the single quote syntax \verb|'(f(x))|.
All of the following examples give the correct answer for this simple function.
\small
\begin{verbatim}
(%i13) f(x) := x - cos(x)$
(%i14) [find_root( f, 0, 1), find_root( 'f, 0, 1),
          find_root( '(f), 0, 1), find_root( f(x), x, 0, 1),
             find_root( '( f(x)), 'x, 0, 1),find_root( '( f(x)), x, 0, 1),
               find_root( 'f(x), x, 0, 1)];
(%o14) [0.739085, 0.739085, 0.739085, 0.739085, 0.739085, 0.739085, 0.739085]
\end{verbatim}
\normalsize

However, one can arrive at specialized homemade functions which require the strict
  syntax (a quoted function entry in the first slot) to behave correctly.
Suppose we need to find the numerical roots of a function defined by an integral.
The following is a toy model which uses such a function in a case where we know
  the answer ahead of time.
Instead of directly looking for the roots of the function
  $f(x)=(x^{2} - 5)$, we look for the roots of the function
  $\int_{\sqrt{5}}^{x} 2\,y\,d\,y$.
\small
\begin{verbatim}
(%i1) fpprintprec : 8$
(%i2) ex : integrate(2*'y,'y,sqrt(5),x);
                                      2
                                     x    5
(%o2)                             2 (-- - -)
                                     2    2
(%i3) ex : expand(ex);
                                     2
(%o3)                               x  - 5
(%i4) define( f(x), ex );
                                         2
(%o4)                           f(x) := x  - 5
(%i5) solve( ex );
(%o5)                    [x = - sqrt(5), x = sqrt(5)]
(%i6) rr : float( map(rhs,%) );
(%o6)                       [- 2.236068, 2.236068]
(%i7) map(f,rr);
(%o7)                   [8.8817842E-16, 8.8817842E-16]
(%i8) find_root(f,0,4);
(%o8)                              2.236068
\end{verbatim}
\normalsize
%\newpage
Let's concentrate on finding the root \verb|2.23606797749979| using the indirect route.
Define a function $g(x)$ in terms of \textbf{integrate}:
\small
\begin{verbatim}
(%i9) g(x) := block([numer,keepfloat,y],
          numer:true,keepfloat:true,
          integrate(2*y,y,sqrt(5),x) )$
(%i10) map(g, [1,2,3]);
(%o10)                        [- 4.0, - 1.0, 4.0]
(%i11) map(f, [1,2,3]);
(%o11)                           [- 4, - 1, 4]
(%i12) [find_root( g, 1, 4), find_root( g(x),x,1,4),
         find_root( '(g(x)),'x,1,4 ), find_root( 'g(x),x,1,4 )];
(%o12)              [2.236068, 2.236068, 2.236068, 2.236068]
\end{verbatim}
\normalsize
We see that we have no problems with getting the function $g(x)$ to "work"
  with \textbf{find\_root}.
However, if we replace \textbf{integrate} with \textbf{quad\_qags}, we find
  problems.
First let's show the numerical integration routine \textbf{quad\_qags} at work
  by itself:
\small
\begin{verbatim}
(%i13) quad_qags(2*'y,'y,sqrt(5),2);
(%o13)                  [- 1.0, 1.11076513E-14, 21, 0]
(%i14) g(2.0);
(%o14)                               - 1.0
\end{verbatim}
\normalsize
The \textbf{quad\_qags} function returns a list of four items, the
  first being the numerical value of the integral, the second being
  an estimate of the error of the answer calculated, the third being the
  number of function evaluations required, and the last  an error code.
The returned error code \verb|0| means no problems were encountered, and we
  will write a function which ignores the error code returned, although
  in "real life" we would always want to report an error code value which
  was not \verb|0|.
\newpage
Here we define \verb|h(x)| which employs the function \textbf{quad\_qags}
  to carry out the numerical itegration.
  
\small
\begin{verbatim}
(%i15) h(x) := block([numer,keepfloat,y,qlist],
          numer:true,keepfloat:true,
          qlist : quad_qags(2*y,y,sqrt(5),x),
           qlist[1] )$
(%i16) map(h,[1,2,3]);
(%o16)                        [- 4.0, - 1.0, 4.0]
(%i17) map(g,[1,2,3]);
(%o17)                        [- 4.0, - 1.0, 4.0]
\end{verbatim}
\normalsize
The function $h(x)$ does the same job as $g(x)$, but uses \textbf{quad\_qags}
  instead of \textbf{integrate}.
Now let's use $h(x)$ in the Maxima function \textbf{find\_root}.
\small
\begin{verbatim}
(%i18) find_root( h(x),x,1,4);
function has same sign at endpoints
[f(1.0) = - 5.0, f(4.0) = - 5.0]
 -- an error.  To debug this try debugmode(true);
\end{verbatim}
\normalsize
We see that the syntax \verb|find_root( h(x),x,1,4)| produced an error due
  to the way \textbf{find\_root} evaluates the first slot. 
Somehow \textbf{find\_root} assigned \verb|-5.0| to the internal
  function fr(x) used to look for the root, and in the first steps of
  that root location, checking for a difference in sign of fr(x) and the
  range end points, found the value \verb|-5.0| at both ends.
In effect, the code was working on the problem
  \verb|find_root( -5.0, x, 1, 4)|.\\
  
The following methods succeed.
\small
\begin{verbatim}
(%i19) [find_root( h, 1, 4),find_root( '(h(x)),'x,1,4 ),
        find_root( '(h(x)),x,1,4 ),find_root( 'h(x),x,1,4 )];
(%o19)              [2.236068, 2.236068, 2.236068, 2.236068]
\end{verbatim}
\normalsize

\subsubsection{newton}
The Maxima function \textbf{newton} can also be used for numerical solutions
  of a single equation.
The Maxima manual describes the syntax as:
\small
\begin{quote}
Function: \textbf{newton}\verb|(expr, x, x_0, eps) |\\
Returns an approximate solution of \verb|expr = 0| by Newton's method,
  considering \verb|expr| to be a function of one variable, \verb|x|.
The search begins with \verb|x = x_0| and proceeds until \verb|abs(expr) < eps|
  (with \verb|expr| evaluated at the current value of \verb|x|). \\
\verb|newton| allows undefined variables to appear in \verb|expr|, so
   long as the termination test \verb|abs(expr) < eps| evaluates to \verb|true|
   or \verb|false|.
Thus it is not necessary that \verb|expr| evaluate to a number. \\
\verb|load(newton1)| loads this function. 
\end{quote}
\normalsize
The two examples in the manual are instructive:
\small
\begin{verbatim}
(%i1) fpprintprec:8$
(%i2) load (newton1);
(%o2) C:/PROGRA~1/MAXIMA~3.0/share/maxima/5.14.0/share/numeric/newton1.mac
(%i3) newton (cos (u), u, 1, 1/100);
(%o3)                              1.5706753
(%i4) ev (cos (u), u = %);
(%o4)                            1.21049633E-4
(%i5) assume (a > 0);
(%o5)                               [a > 0]
(%i6) newton (x^2 - a^2, x, a/2, a^2/100);
(%o6)                             1.0003049 a
(%i7) ev (x^2 - a^2, x = %);
                                              2
(%o7)                          6.09849048E-4 a
\end{verbatim}
\normalsize
Of course both of these examples are found exactly by \textbf{solve}:
\small
\begin{verbatim}
(%i8) solve( cos(x) );
`solve' is using arc-trig functions to get a solution.
Some solutions will be lost.
                                        %pi
(%o8)                              [x = ---]
                                         2
(%i9) float(%);
(%o9)                           [x = 1.5707963]
(%i10) solve( x^2 - a^2,x );
(%o10)                         [x = - a, x = a]
\end{verbatim}
\normalsize
%\newpage
I find the source code (Windows XP) in the folder\\
\verb|c:\Program Files\Maxima-5.14.0\share\maxima\5.14.0\share\numeric|.
Here is the code in the file newton1.mac:
\small
\begin{verbatim}
newton(exp,var,x0,eps):=
    block([xn,s,numer],
      numer:true,
      s:diff(exp,var),
      xn:x0,
      loop,
        if abs(subst(xn,var,exp))<eps then return(xn),
        xn:xn-subst(xn,var,exp)/subst(xn,var,s),
      go(loop)
    )$
\end{verbatim}
\normalsize
We see that the code implements the Newton-Raphson algorithm.
Given a function $f(x)$ and an initial guess $x_{g}$ which can be assigned
  to, say, $x^{i}$, a closer approximation to the value of $x$ for which
  $f(x)=0$ is generated by
  $$ x^{i+1} = x^{i} - \frac{f(x^{i})}{f'(x^{i}) }. $$
The method depends on being able to evaluate the derivative of $f(x)$ which
  appears in the denominator.\\
Steven Koonin's (edited) comments (Computational Physics: Fortran Version, Steven E. Koonin and Dawn Meredith,
   WestView Press, 1998, Ch. 1, Sec.3) are cautionary:
\small
\begin{quote}
When the function $f(x)$ is badly behaved near its root (e.g., there is an inflection
  point near the root) or when there are several roots, the "automatic" Newton-Raphson
   method can fail to converge at all or converge to the wrong answer if the initial
   guess for the root is poor.
\end{quote}
\normalsize

  
  
\newpage
  
\subsection{Two or More Equations: Symbolic and Numerical Solutions}
For sets of equations, we can use \textbf{solve} with the syntax:

\small
\begin{quote}
Function: \textbf{solve}\verb|([eqn_1, ..., eqn_n], [x_1, ..., x_n]) |\\
\verb|solve ([eqn_1, ..., eqn_n], [x_1, ..., x_n])| solves a system of simultaneous
  (linear or non-linear) polynomial equations by calling \verb|linsolve| or
  \verb|algsys| and returns a list of the solution lists in the variables.
In the case of \verb|linsolve| this list would contain a single list of solutions.
This form of \verb|solve| takes two lists as arguments.
The first list represents the equations to be solved; the second list
   is a list of the unknowns to be determined.
If the total number of variables in the equations is equal to the number
  of equations, the second argument-list may be omitted.
\end{quote}
\normalsize

\subsubsection{Numerical or Symbolic Linear Equations with solve or linsolve  }
The Maxima functions \textbf{solve}, \textbf{linsolve}, and \textbf{linsolve\_by\_lu}
   can be used for linear equations.\\
   
Linear equations containing symbolic coefficients  can be
    "solved" by \textbf{solve} and \textbf{linsolve}.
For example the pair of equations
$$ a x + b y = c, d x + e y = f $$	.
Here we solve for the values of \verb|(x,y)| which simultaneously
  satisfy these two equations and check the solutions.
\small
\begin{verbatim}
(%i1) eqns : [a*x + b*y = c, d*x + e*y = f];
(%o1)                   [b y + a x = c, e y + d x = f]
(%i2) solve(eqns,[x,y]);
                              c e - b f      c d - a f
(%o2)                 [[x = - ---------, y = ---------]]
                              b d - a e      b d - a e
(%i3) soln : linsolve(eqns,[x,y]);
                              c e - b f      c d - a f
(%o3)                  [x = - ---------, y = ---------]
                              b d - a e      b d - a e
(%i4) (ev(eqns, soln), ratexpand(%%) );
(%o4)                           [c = c, f = f]
\end{verbatim}
\normalsize
Note the presence of the determinant of the "coefficient matrix" in the
  denominator of the solutions.\\
   
A simple numerical (rather than symbolic) two equation example:
\small
\begin{verbatim}
(%i1) eqns : [3*x-y=4,x+y=2];
(%o1)                      [3 x - y = 4, y + x = 2]
(%i2) solns : solve(eqns,[x,y]);
                                     3      1
(%o2)                          [[x = -, y = -]]
                                     2      2
(%i3) soln : solns[1];
                                     3      1
(%o3)                           [x = -, y = -]
                                     2      2
(%i4) for i thru 2 do disp( ev( eqns[i],soln ) )$
                                     4 = 4
                                     2 = 2
\end{verbatim}
\normalsize
Using \textbf{linsolve} instead returns a list, rather than a list of a list.
\small
\begin{verbatim}
(%i5) linsolve(eqns,[x,y]);
                                     3      1
(%o5)                           [x = -, y = -]
                                     2      2
\end{verbatim}
\normalsize

\subsubsection{Matrix Methods for Linear Equation Sets: linsolve\_by\_lu}
The present version (5.14) of the Maxima manual does not have an index entry
  for the function \textbf{linsolve\_by\_lu}.
These notes include only the simplest of many interesting examples described
  in two mailing list responses by the creator of the linear algebra package,
  Barton Willis (Dept. of Mathematics, Univ. Nebraska at Kearney),
  dated Oct. 27, 2007 and Nov. 21, 2007.
 
 \smallskip
 
If we re-cast the two equation problem we have just been solving 
  in the form of a matrix equation\\
  \verb|A . xcol = bcol|, we need
  to construct the square matrix \verb|A| so that matrix multiplication by the column
  vector \verb|xcol| results in a column vector whose rows contain the left hand sides
  of the equations.
The column vector \verb|bcol| rows hold the right hand sides of the equations.
Our notation below ( as \verb|xycol| and \verb|xylist| ) is only natural for a two equation
  problem (ie., a 2 x 2 matrix): you can invent your own notation to suit
  your problem.

\smallskip
If the argument of the function \textbf{matrix} is a simple list, the result is a
    row vector (a special case of a matrix object).
We can then take the \textbf{transpose} of the row matrix to obtain a column matrix,
   such as \verb|xcol| below.
A direct definition of a two element column matrix is \verb|matrix([x],[y])|, which
  is probably faster than \verb|transpose(matrix([x,y]))|.   
To reduce the amount of space taken up by the default matrix output, we
  can set \verb|display2d:false|.
\small
\begin{verbatim}
(%i6) m : matrix( [3,-1],[1,1] );
                                  [ 3  - 1 ]
(%o6)                             [        ]
                                  [ 1   1  ]
(%i7) display2d:false$
(%i8) m;
(%o8) matrix([3,-1],[1,1])
(%i9) xcol : matrix([x],[y]);
(%o9) matrix([x],[y])
(%i10) m . xcol;
(%o10) matrix([3*x-y],[y+x])
\end{verbatim}
\normalsize
The period allows non-commutative multiplication of matrices.
The linear algebra package function \textbf{linsolve\_by\_lu} allows us to specify
  the column vector \verb|bcol| as a simple list:
\small
\begin{verbatim}
(%i11) b : [4,2];
(%o11) [4,2]
(%i12) linsolve_by_lu(m,b);
(%o12) [matrix([3/2],[1/2]),false]
(%i13) xycol : first(%);
(%o13) matrix([3/2],[1/2])
(%i14) m . xycol - b;
(%o14) matrix([0],[0])
(%i15) xylist : makelist( xycol[i,1],i,1,length(xycol) );
(%o15) [3/2,1/2]
(%i16) xyrules : map("=",[x,y],xylist);
(%o16) [x = 3/2,y = 1/2]
\end{verbatim}
\normalsize
The \textbf{matrix} argument needs to be a square matrix.
The output of \textbf{linsolve\_by\_lu} is a list: the first element is the solution
  column vector  which for this two dimensional problem we have called \verb|xycol|.

We check the solution in input \verb|%i14|.
The output \verb|%o14| is a column vector each of whose elements is zero;
  such a column vector is ordinarily replaced by the number \verb|0|.\\
  
One should always check solutions when using computer algebra software,
  since the are occasional bugs in the algorithms used.
The second list element returned by \textbf{linsolve\_by\_lu} is \textbf{false}, which 
  should always be the value returned when the calculation uses non-floating point numbers
  as we have here.
If floating point numbers are used, the second element should be either \textbf{false} or
  a lower bound to the "matrix condition number". 
(We show an example later.)
We have shown how to convert the returned \verb|xycol| matrix object into an
  ordinary list, and how to then construct a list of replacement rules (as would be
  returned by \textbf{linsolve}) which could then be used for other purposes.
The use of \textbf{makelist} is the recommended way to use parts of
  \textbf{matrix} objects in lists.
However, here is a simple method which avoids \textbf{makelist}:
\small
\begin{verbatim}
(%i17) flatten( args( xycol));
(%o17) [3/2,1/2]
\end{verbatim}
\normalsize
but \textbf{makelist} should normally be the weapon of choice, since the method is foolproof and
  can be extended to many exotic ways of using the various elements of a matrix.  
 
 \smallskip
The Maxima function \textbf{linsolve\_by\_lu} allows the second argument
   to be either a list (as in the example above) or a column matrix, as we
   show here.

\small
\begin{verbatim}
(%i18) bcol : matrix([4],[2])$
(%i19) linsolve_by_lu(m,bcol);
(%o19) [matrix([3/2],[1/2]),false]
(%i20) m . first(%) - bcol;
(%o20) matrix([0],[0])
\end{verbatim}
\normalsize

\subsubsection{Symbolic Linear Equation Solutions: Matrix Methods}
Here we use \textbf{linsolve\_by\_lu} for the pair of equations
$$ a x + b y = c, \qquad  d x + e y = f, $$	
  seeking the values of \verb|(x,y)| which simultaneously
  satisfy these two equations and checking the solutions.
\small
\begin{verbatim}
(%i1) display2d:false$
(%i2) m : matrix( [a,b], [d,e] )$
(%i3) bcol : matrix( [c], [f] )$
(%i4) ls : linsolve_by_lu(m,bcol);
(%o4) [matrix([(c-b*(f-c*d/a)/(e-b*d/a))/a],[(f-c*d/a)/(e-b*d/a)]),false]
(%i5) xycol : ratsimp( first(ls) );
(%o5) matrix([-(b*f-c*e)/(a*e-b*d)],[(a*f-c*d)/(a*e-b*d)])
(%i6) ( m . xycol - bcol, ratsimp(%%) );
(%o6) matrix([0],[0])
(%i7) (display2d:true,xycol);
                                [   b f - c e ]
                                [ - --------- ]
                                [   a e - b d ]
(%o7)                           [             ]
                                [  a f - c d  ]
                                [  ---------  ]
                                [  a e - b d  ]
(%i8) determinant(m);
(%o8)                              a e - b d								
\end{verbatim}
\normalsize
We see the value of the determinant of the "coefficient matrix" \verb|m|
  in the denominator of the solution.

\subsubsection{Multiple Solutions from Multiple Right Hand Sides}
Next we show how one can solve for multiple solutions (with one call to
  \textbf{linsolve\_by\_lu}) corresponding to a number of different "right hand sides".
We will turn back on the default matrix display, and re-define the first 
  (right hand side) column vector as \verb|b1col|, and the corresponding solution
   \verb|x1col|.
\small
\begin{verbatim}
(%i21) display2d:true$
(%i22) b1col : matrix( [4], [2] );
                                     [ 4 ]
(%o22)                               [   ]
                                     [ 2 ]
(%i23) x1col : first( linsolve_by_lu(m,b1col) );
                                     [ 3 ]
                                     [ - ]
                                     [ 2 ]
(%o23)                               [   ]
                                     [ 1 ]
                                     [ - ]
                                     [ 2 ]
(%i24) b2col : matrix( [3], [1] );
                                     [ 3 ]
(%o24)                               [   ]
                                     [ 1 ]
(%i25) x2col : first( linsolve_by_lu(m, b2col) );
                                     [ 1 ]
(%o25)                               [   ]
                                     [ 0 ]
(%i26) bmat : matrix( [4,3], [2,1] );
                                   [ 4  3 ]
(%o26)                             [      ]
                                   [ 2  1 ]
(%i27) linsolve_by_lu( m, bmat );
                                [ 3    ]
                                [ -  1 ]
                                [ 2    ]
(%o27)                         [[      ], false]
                                [ 1    ]
                                [ -  0 ]
                                [ 2    ]
(%i28) xsolns : first(%);
                                   [ 3    ]
                                   [ -  1 ]
                                   [ 2    ]
(%o28)                             [      ]
                                   [ 1    ]
                                   [ -  0 ]
                                   [ 2    ]
\end{verbatim}
\newpage
\begin{verbatim}
(%i29) m . xsolns - bmat;
                                   [ 0  0 ]
(%o29)                             [      ]
                                   [ 0  0 ]
(%i30) x1col : col(xsolns,1);
                                     [ 3 ]
                                     [ - ]
                                     [ 2 ]
(%o30)                               [   ]
                                     [ 1 ]
                                     [ - ]
                                     [ 2 ]
(%i31) x2col : col(xsolns,2);
                                     [ 1 ]
(%o31)                               [   ]
                                     [ 0 ]
\end{verbatim}
\normalsize
In input \verb|%i26| we define the \verb|2 x 2| matrix \verb|bmat| whose first column
  is \verb|b1col| and whose second column is \verb|b2col|.
Using \verb|bmat| as the second argument to \textbf{linsolve\_by\_lu} results in
  a return list whose first element (what we call \verb|xsolns|) is a \verb|2 x 2|
  matrix whose first column is \verb|x1col| (the solution vector corresponding to
  \verb|b1col|) and whose second column is \verb|x2col| (the solution vector
  corresonding to \verb|b2col|).
In input \verb|%i29| we check both solutions simultaneously.
The result is a matrix with every element equal to zero, which would ordinarily
  be replaced by the number \verb|0|.
In input \verb|%i30| we extract \verb|x1col| using the \textbf{col} function.\\

\subsubsection{Three Linear Equation Example }  
We next consider a simple three linear equation example.
Although \textbf{solve} does not require the list \verb|[x,y,z]| in this
  problem, if you leave it out, the solution list returned will be in an
  order determined by the peculiarities of the code, rather than by you.
By including the variable list as \verb|[x,y,z]|, you are forcing the 
  output list to be in the same order.
\small
\begin{verbatim}
(%i1) eqns : [2*x - 3*y + 4*z = 2, 3*x - 2*y + z = 0,
                     x + y - z = 1]$
(%i2) display2d:false$
(%i3) solns : solve( eqns,[x,y,z] );
(%o3) [[x = 7/10,y = 9/5,z = 3/2]]
(%i4) soln : solns[1];
(%o4) [x = 7/10,y = 9/5,z = 3/2]
(%i5) for i thru 3 do disp( ev(eqns[i],soln) )$
2 = 2
0 = 0
1 = 1
\end{verbatim}
\normalsize
The Maxima function \textbf{linsolve} has the same syntax as \textbf{solve} (for
  a set of equations) but you cannot leave out the list of unknowns.
\small
\begin{verbatim}
(%i6) linsolve(eqns,[x,y,z]);
(%o6) [x = 7/10,y = 9/5,z = 3/2]
\end{verbatim}
\normalsize
Notice that \textbf{linsolve} returns a list, while \textbf{solve} returns a
  list of lists.

  \smallskip

\newpage

Next we use \textbf{linsolve\_by\_lu} on this three equation problem.
Using the laws of matrix multiplication, we "reverse engineer" the $3 \times 3$ 
  matrix \verb|m| and the three element column vector \verb|bcol| which provide
  an equivalent problem in matrix form.
\small
\begin{verbatim}
(%i7) m : matrix( [2,-3,4],[3,-2,1],[1,1,-1] )$
(%i8) xcol : matrix( [x],[y],[z] )$
(%i9) m . xcol;
(%o9) matrix([4*z-3*y+2*x],[z-2*y+3*x],[-z+y+x])
(%i10) bcol : matrix( [2],[0],[1] )$
(%i11) linsolve_by_lu(m,bcol);
(%o11) [matrix([7/10],[9/5],[3/2]),false]
(%i12) xyzcol : first(%);
(%o12) matrix([7/10],[9/5],[3/2])
(%i13) m . xyzcol - bcol;
(%o13) matrix([0],[0],[0])
(%i14) xyzlist : makelist( xyzcol[i,1],i,1,length(xyzcol) );
(%o14) [7/10,9/5,3/2]
(%i15) xyzrules : map("=",[x,y,z],xyzlist);
(%o15) [x = 7/10,y = 9/5,z = 3/2]
\end{verbatim}
\normalsize

\smallskip

Both \textbf{linsolve} and \textbf{solve} can handle an "equation list" which
  is actually an "expression list" in which it is understood that the required
  equations are generated by setting each expression to zero.
\small
\begin{verbatim}
(%i16) exs : [2*x - 3*y + 4*z - 2, 3*x - 2*y + z ,
                     x + y - z - 1];
(%o16)       [4 z - 3 y + 2 x - 2, z - 2 y + 3 x, - z + y + x - 1]
(%i17) linsolve(exs,[x,y,z]);
                                 7       9      3
(%o17)                      [x = --, y = -, z = -]
                                 10      5      2
(%i18) solve(exs);
                                 3      9      7
(%o18)                     [[z = -, y = -, x = --]]
                                 2      5      10
\end{verbatim}
\normalsize
The Maxima manual presents a linear equation example in which there is
  an undefined parameter "a".
\small
\begin{verbatim}
(%i1) eqns : [x + z = y,2*a*x - y = 2*a^2,y - 2*z = 2]$
(%i2) solns : linsolve(eqns,[x,y,z] );
(%o2)                   [x = a + 1, y = 2 a, z = a - 1]
(%i3) solve(eqns,[x,y,z]);
(%o3)                  [[x = a + 1, y = 2 a, z = a - 1]]
(%i4) for i thru 3 do (
             e: expand( ev(eqns[i],solns) ),disp(lhs(e) - rhs(e)) )$
                                       0
                                       0
                                       0
(%i5) e;
(%o5)                                2 = 2
(%i6) [kill(e),e];
(%o6)                              [done, e]
\end{verbatim}
\normalsize
The code in input \verb|%i4| binds an equation to the symbol \verb|e|
   which can be removed with \textbf{kill}.
\newpage
We can avoid introducing a global binding to a symbol by using \verb|%%|,
  which allows use of the previous result inside a piece of code.
\small
\begin{verbatim}
(%i7) for i thru 3 do (
             expand( ev(eqns[i],solns) ),disp(lhs(%%) - rhs(%%)) )$
                                       0
                                       0
                                       0
\end{verbatim}
\normalsize
Note the syntax used here: \verb|do ( job1, job2 ) |.

\smallskip
Let's try out \textbf{linsolve\_by\_lu} on this three linear (in the
  unknowns \verb|(x,y,z)|) equation problem which involves the unbound
  parameter \verb|a|.
\small
\begin{verbatim}
(%i8) display2d:false$
(%i9) m : matrix([1,-1,1],[2*a,-1,0],[0,1,-2] )$
(%i10) xcol : matrix( [x],[y],[z] )$
(%i11) m . xcol;
(%o11) matrix([z-y+x],[2*a*x-y],[y-2*z])
(%i12) bcol : matrix( [0],[2*a^2], [2] )$
(%i13) soln : linsolve_by_lu(m,bcol)$
(%i14) xyzcol : ( first(soln), ratsimp(%%) );
(%o14) matrix([a+1],[2*a],[a-1])
(%i15) ratsimp( m . xyzcol - bcol);
(%o15) matrix([0],[0],[0])
\end{verbatim}
\normalsize
In input \verb|%i15| we check the solution, and the result is a three element
  column vector, all of whose elements are zero.
Such a column matrix is ordinarily replaced by the number \verb|0|.\\

  
\subsubsection{Surpressing rat Messages: ratprint}
If we start with two linear equations with decimal coefficients, \textbf{solve} 
  (and the other methods) converts the decimals to ratios of integers, and tries
  to find an exact solution.
You can avoid seeing all the rational replacements done by setting the
  option variable \textbf{ratprint} to \textbf{false}.
Despite the assertion, in the manual section on \textbf{rat}, that
\small
\begin{quote}
\verb|keepfloat| if \verb|true| prevents floating point numbers from being
 converted to rational numbers. 
\end{quote}
\normalsize
setting \textbf{keepfloat} to \textbf{true} here does not stop \textbf{solve} from
  converting decimal numbers to ratios of integers.  
\small
\begin{verbatim}
(%i1) [keepfloat,ratprint];
(%o1)                            [false, true]
(%i2) display2d:false$
(%i3) fpprintprec:8$
(%i4) eqns : [0.2*x + 0.3*y = 3.3,0.1*x - 0.8*y = 6.6]$
(%i5) solns : solve(eqns, [x,y]);
`rat' replaced -3.3 by -33/10 = -3.3
`rat' replaced 0.2 by 1/5 = 0.2
`rat' replaced 0.3 by 3/10 = 0.3
`rat' replaced -6.6 by -33/5 = -6.6
`rat' replaced 0.1 by 1/10 = 0.1
`rat' replaced -0.8 by -4/5 = -0.8
(%o5) [[x = 462/19,y = -99/19]]
(%i6) linsolve(eqns,[x,y]);
`rat' replaced -3.3 by -33/10 = -3.3
`rat' replaced 0.2 by 1/5 = 0.2
`rat' replaced 0.3 by 3/10 = 0.3
`rat' replaced -6.6 by -33/5 = -6.6
`rat' replaced 0.1 by 1/10 = 0.1
`rat' replaced -0.8 by -4/5 = -0.8
(%o6) [x = 462/19,y = -99/19]
(%i7) m : matrix([0.2,0.3],[0.1,-0.8] )$
(%i8) bcol : matrix( [3.3], [6.6] )$
(%i9) linsolve_by_lu(m,bcol);
`rat' replaced 0.2 by 1/5 = 0.2
`rat' replaced 0.2 by 1/5 = 0.2
`rat' replaced 0.2 by 1/5 = 0.2
`rat' replaced 0.95 by 19/20 = 0.95
(%o9) [matrix([24.315789],[-5.2105263]),false]
(%i10) ratprint:false$
(%i11) solns : solve(eqns, [x,y]);
(%o11) [[x = 462/19,y = -99/19]]
(%i12) linsolve(eqns,[x,y]);
(%o12) [x = 462/19,y = -99/19]
(%i13) linsolve_by_lu(m,bcol);
(%o13) [matrix([24.315789],[-5.2105263]),false]
(%i14) float(solns);
(%o14) [[x = 24.315789,y = -5.2105263]]
\end{verbatim}
\normalsize
Matrix methods for sets of linear equations can be solved using IEEE double floats
  (as well as "big floats") by including an optional "method" specification \verb|'floatfield| 
  after the input column vector (or matrix of input column vectors).
\small
\begin{verbatim}
(%i15) linsolve_by_lu(m,bcol, 'floatfield);
(%o15) [matrix([24.315789],[-5.2105263]),8.8815789]
\end{verbatim}
\normalsize
In this example the lower bound of the matrix condition number appears as
  the second element of the returned list.
\subsubsection{Non-Linear Polynomial Equations}
Here is an example of using \textbf{solve} to find a pair of exact solutions
  of a pair of equations, one equation being linear, the other quadratic.
The pair of solutions represent the two intersections of the unit circle
  with the line $y = -x/3.$
\small
\begin{verbatim}
(%i1) fpprintprec:8$
(%i2) eqns : [x^2 + y^2 = 1, x + 3*y = 0]$
(%i3) solns : solve(eqns,[x,y]);
                 3                 1
(%o3) [[x = - --------, y = ---------------],
              sqrt(10)      sqrt(2) sqrt(5)
                                                 3                   1
                                         [x = --------, y = - ---------------]]
                                              sqrt(10)        sqrt(2) sqrt(5)
\end{verbatim}
\newpage
\begin{verbatim}
(%i4) solns : rootscontract(solns);
                  3             1               3               1
(%o4)  [[x = - --------, y = --------], [x = --------, y = - --------]]
               sqrt(10)      sqrt(10)        sqrt(10)        sqrt(10)
(%i5) for i thru 2 do for j thru 2 do (
                ev(eqns[i],solns[j]), disp(lhs(%%)-rhs(%%)) )$
                                       0
                                       0
                                       0
                                       0
(%i6) float(solns);
(%o6)  [[x = - 0.948683, y = 0.316228], [x = 0.948683, y = - 0.316228]]
\end{verbatim}
\normalsize
The pair of solutions reflect the symmetry of the given set of equations, which
  remain invariant under the transformation $x \rightarrow - y, y \rightarrow - x$.

  \smallskip
A set of two nonlinear polynomial equations with four solutions generated 
  by \textbf{solve} is one of the examples in the Maxima manual.
One of the solutions is exact, one is a real inexact solution, and the other two
  solutions are inexact complex solutions.
\small
\begin{verbatim}
(%i1)  fpprintprec:8$
(%i2)  eqns : [4*x^2 - y^2 = 12, x*y - x = 2]$
(%i3)  solns : solve( eqns, [x,y] );
(%o3) [[x = 2, y = 2], [x = 0.520259 %i - 0.133124,
y = 0.0767838 - 3.6080032 %i], [x = - 0.520259 %i - 0.133124,
y = 3.6080032 %i + 0.0767838], [x = - 1.7337518, y = - 0.153568]]
(%i4) for i thru 2 do for j thru length(solns) do (
      expand( ev(eqns[i],solns[j]) ),
      abs( lhs(%%) - rhs(%%) ), disp(%%) )$
                                       0
                                2.36036653E-15
                                2.36036653E-15
                                 1.13954405E-6
                                       0
                                      0.0
                                      0.0
                                 9.38499825E-8
\end{verbatim}
\normalsize
To get real numbers from the complex solutions, we used the \textbf{abs}
   function, which calculates the absolute value of a complex number.
Note the syntax used to check the solutions: \verb| do ( job1, job2, job3)|.

\newpage

\subsubsection{General Sets of Nonlinear Equations: eliminate, mnewton}
Solving systems of nonlinear equations is much more difficult than solving
  one nonlinear equation.
A wider variety of behavior is possible: determining the existence and
  number of solutions or even a good starting guess is more complicated.
There is no method which can guarantee convergence to the desired solution.
The computing labor increases rapidly with the number of dimensions of the problem.


\subsubsection{Intersections of Two Circles:  implicit\_plot}
Given two circles, we seek the intersections points.
We first write down the defining equations of the two circles, and
  look visually for points \verb|(x,y)| which simultaneously lie on each circle.
We use \textbf{implicit\_plot} for this visual search.
\small
\begin{verbatim}
(%i1) [eq1 : x^2 + y^2 = 1,eq2 : (x-2)^2 + (y-2)^2 = 4]$
(%i2) eqns : [eq1,eq2]$
(%i3) load(implicit_plot);
(%o3) 
    C:/PROGRA~1/MAXIMA~3.0/share/maxima/5.14.0/share/contrib/implicit_plot.lisp
(%i4) implicit_plot(eqns,[x,-6,6],[y,-6,6],[nticks,200],
          [gnuplot_preamble, "set zeroaxis" ])$
\end{verbatim}
\normalsize
We are not taking enough care with the \verb|x| and \verb|y| ranges to make
  the "circles" circular, but we can use the cursor to read off approximate
  intersections points: \verb|(x,y) = (0.26,0.98), (0.96,0.30)|.
However, the defining equations are invariant under the symmetry
  transformation $x \leftrightarrow y$, so the solution pairs must
  also respect this symmetry.
We next eliminate \verb|y| between the two equations and use \textbf{solve}
  to find accurate values for the \verb|x|'s.
Since we know that both solutions have positive values for \verb|y|, we
  enforce this condition on equation 1.  
\small
\begin{verbatim}
(%i5) solve(eq1,y);
                                     2                 2
(%o5)               [y = - sqrt(1 - x ), y = sqrt(1 - x )]
(%i6) ysoln : second(%);
                                             2
(%o6)                          y = sqrt(1 - x )
(%i7) eliminate(eqns,[y]);
                                   2
(%o7)                         [32 x  - 40 x + 9]
(%i8) xex : solve(first(%));
                            sqrt(7) - 5      sqrt(7) + 5
(%o8)                [x = - -----------, x = -----------]
                                 8                8
(%i9) (fpprintprec:8, xex : float(xex) );
(%o9)                   [x = 0.294281, x = 0.955719]
(%i10) [x1soln : first(xex), x2soln : second(xex) ]$
(%i11) [ev(%o7,x1soln), ev(%o7,x2soln)];
(%o11)                    [[- 4.4408921E-16], [0.0]]
(%i12) y1soln : ev(ysoln,x1soln);
(%o12)                           y = 0.955719
(%i13) y2soln : ev(ysoln,x2soln);
(%o13)                           y = 0.294281
(%i14) [soln1:[x1soln,y1soln],soln2:[x2soln,y2soln] ]$
(%i15) [ev(eqns,soln1), ev(eqns,soln2) ];
(%o15)             [[1.0 = 1, 4.0 = 4], [1.0 = 1, 4.0 = 4]]
(%i16) [soln1,soln2];
(%o16)   [[x = 0.294281, y = 0.955719], [x = 0.955719, y = 0.294281]]
\end{verbatim}
\normalsize
We have solutions (\verb|%o16|) which respect the symmetry of the equations.
The solutions have been numerically checked in input \verb|%i15|.  \\
  
\smallskip
%%\begin{figure}[!ht] 
\begin{figure} [h]
   \centerline{\includegraphics[width=5in]{mycircle2.eps} }
	\caption{two circles}
\end{figure}
  
\subsubsection{Using Draw for Implicit Plots}
The figure above was created using the draw package with the following code
  in a separate work file "implicitplot1.mac".
The code the the file is
\small
\begin{verbatim}
/* file implicitplot1.mac  */
/*  need load(implicit_plot);  to use this code  */
disp(" doplot() ")$

doplot() := block([ x,y, eq1, eq2, y1:-2, y2:4,r, x1 ,x2 ],
                  r : 1.56,
                  x1 : r*y1,
                  x2 : r*y2,
                  eq1 : x^2 + y^2 = 1,
              eq2 : (x-2)^2 + (y-2)^2 = 4,
          draw2d( 
             grid      = true,
             line_type = solid,
             line_width  = 3,
             color = blue,
             implicit(eq1, x, x1,x2, y, y1,y2),
             color = red,
             implicit(eq2, x, x1,x2, y, y1,y2),
             color = black,
             point_type = filled_circle,
             point_size = 2,
             points( [ [0.294281,0.955719], [ 0.955719, 0.294281] ] ),
             title     = "Intersections of Two Circles" ,
             terminal = 'eps ,
			  file_name = "c:/work2/mycircle2" )
         )$
\end{verbatim}
\normalsize
Here is record of use:
\small
\begin{verbatim}
(%i1) load(draw);
(%o1)   C:/PROGRA~1/MAXIMA~3.0/share/maxima/5.14.0/share/draw/draw.lisp
(%i2) load(implicitplot1);
                                   doplot() 

(%o2)                     c:/work2/implicitplot1.mac
(%i3) doplot();
(%o3)                 [gr2d(implicit, implicit, points)]
\end{verbatim}
\normalsize
Note that no actual plot was drawn on the screen, since an "eps" file
  was created "mycircle2.eps" in my work folder \verb|c:\work2| for use
  in my latex file.
To use this code to get a figure on your screen, you would remove the last two
  lines: \verb| terminal = 'eps , file_name = "c:/work2/mycircle2"| and,
  very important!, also remove the comma at the end of "two circles".

\subsubsection{Another Example}  
We next work through Example 6.5.1 (page 149), in "Numerical Methods for Unconstrained Optimization
  and Nonlinear Equations" by J. E. Dennis, Jr. and  Robert B. Schnabel (Prentice-Hall, 1983).
\small
\begin{verbatim}
(%i1) eq1 : x^2 + y^2 = 2;
                                   2    2
(%o1)                             y  + x  = 2
(%i2) eq2 : exp(x - 1) + y^3 = 2;
                                3     x - 1
(%o2)                          y  + %e      = 2
\end{verbatim}
\normalsize
We will concentrate on the solution \verb|(x,y) = (1,1)|.
We can eliminate \verb|y| and plot the resulting function of \verb|x|
  to visually locate the \verb|x| solutions.
\small
\begin{verbatim}
(%i3) eliminate([eq1,eq2],[y]);
             2 x       x + 1     2  6       2  4        2  2       2
(%o3)     [%e    - 4 %e      + %e  x  - 6 %e  x  + 12 %e  x  - 4 %e ]
(%i4) ex : first(%);
             2 x       x + 1     2  6       2  4        2  2       2
(%o4)      %e    - 4 %e      + %e  x  - 6 %e  x  + 12 %e  x  - 4 %e
(%i5) plot2d([0.0,ex],[x,-5,5] )$
(%i6) plot2d([0.0,ex],[x,0,2] )$
(%i7) x1 : find_root(ex,x,0.5,1.5);
(%o7)                                 1.0
(%i8) yeqn : ev(eq1,x = x1);
                                  2
(%o8)                            y  + 1.0 = 2
(%i9) solve(yeqn);
`rat' replaced -1.0 by -1/1 = -1.0
(%o9)                          [y = - 1, y = 1]
\end{verbatim}
\newpage
\begin{verbatim}
(%i10) ysol : second(%);
(%o10)                               y = 1
(%i11) soln1 : [x = x1, ysol];
(%o11)                         [x = 1.0, y = 1]
(%i12) ev(eq1,soln1);
(%o12)                              2.0 = 2
(%i13) ev(eq2,soln1);
(%o13)                              2.0 = 2
\end{verbatim}
\normalsize
One solution is then \verb|%o11| and the solution has been checked.\\

\smallskip
%%\begin{figure}[!ht] 
\begin{figure} [h]
   \centerline{\includegraphics[width=5in]{dennis1.eps} }
	\caption{Dennis and Schnabel Example}
\end{figure}


Let's try \textbf{mnewton} on this problem.
Wikipedia, under "Newton's Method",  has the introduction:

\small
\begin{quote}
In numerical analysis, Newton's method (also known as the Newton Raphson method
 or the Newton Fourier method) is perhaps the best known method for finding
 successively better approximations to the zeros (or roots) of a
 real valued function.
Newton's method can often converge remarkably quickly, especially if
  the iteration begins "sufficiently near" the desired root.
Just how near "sufficiently near" needs to be and just how quickly
  "remarkably quickly" can be depends on the problem, as is
  discussed in detail below.
Unfortunately, far from the desired root, Newton's method can easily lead
  an unwary user astray, and astray with little warning.
Such users are advised to heed the advice of Press, et. al. (1992), who
  suggest embedding Newton's method in a routine that also detects
  possible convergence failures.\\
Newton's method can also be used to find a minimum or maximum of
  such a function, by finding a zero in the function's first derivative,
  see Newton's method as an optimization algorithm.\\
The algorithm is first in the class of Householder's methods, succeeded by Halley's method.
\end{quote}
\normalsize
The Maxima manual entry for \textbf{mnewton} begins:
\small
\begin{quote}
Function: \textbf{mnewton}\verb|(FuncList,VarList,GuessList)|\\ 
Multiple nonlinear functions solution using the Newton method.
FuncList is the list of functions to solve, VarList is the list
 of variable names, and GuessList is the list of initial approximations. \\
The solution is returned in the same format that solve() returns.
If the solution isn't found, \verb|[]| is returned. \\
This function is controlled by global variables \verb|newtonepsilon|
 and \verb|newtonmaxiter|. 
\end{quote}
\normalsize

Here are the Maxima manual examples of \textbf{mnewton}:
\small
\begin{verbatim}
(%i1) load(mnewton);
(%o1) C:/PROGRA~1/MAXIMA~3.0/share/maxima/5.14.0/share/contrib/mnewton.mac
(%i2) mnewton([2*a^a-5],[a],[1]);

(%o2)                      [[a = 1.70927556786144]]
(%i3) mnewton([2*3^u-v/u-5, u+2^v-4], [u, v], [2, 2]);
(%o3)          [[u = 1.066618389595407, v = 1.552564766841786]]
(%i4) mnewton([x1+3*log(x1)-x2^2, 2*x1^2-x1*x2-5*x1+1],
              [x1, x2], [5, 5]);
(%o4)         [[x1 = 3.756834008012769, x2 = 2.779849592817898]]
\end{verbatim}
\normalsize

In the above examples, \textbf{mnewton} is presented with a list of expressions.
Here we use \textbf{mnewton} on the Dennis and Schnabel problem we solved earlier
 using eliminate and find\_root. 
We rewrite the equations as expressions here.

\small
\begin{verbatim}
(%i1) fpprintprec:8$
(%i2) load(mnewton);
(%o2) C:/PROGRA~1/MAXIMA~3.0/share/maxima/5.14.0/share/contrib/mnewton.mac
(%i3) exs : [x^2 + y^2 -2, exp(x-1)+y^3-2]$
(%i4) mn(x0,y0) := mnewton(exs,[x,y],[x0,y0] )$
(%i5) mn(1.1,0.9);
(%o5)                        [[x = 1.0, y = 1.0]]
(%i6) mn(1.2,0.8);
(%o6)                        [[x = 1.0, y = 1.0]]
(%i7) mn(1.3,0.7);
(%o7)                        [[x = 1.0, y = 1.0]]
(%i8) mn(1.4,0.6);
(%o8)                        [[x = 1.0, y = 1.0]]
(%i9) mn(1.5,0.5);
(%o9)                  [[x = - 0.713747, y = 1.2208868]]
(%i10) mn(1.5,0.6);
(%o10)                       [[x = 1.0, y = 1.0]]
(%i11) mn(1.7,0.6);
(%o11)                       [[x = 1.0, y = 1.0]]
(%i12) mn(1.9,0.6);
(%o12)                 [[x = - 0.713747, y = 1.2208868]]
(%i13) mn(1.9,0.7);
(%o13)                       [[x = 1.0, y = 1.0]]
(%i14) mn(2,0.7);
(%o14)                       [[x = 1.0, y = 1.0]]
(%i15) mn(0.8,1.1);
(%o15)                       [[x = 1.0, y = 1.0]]
(%i16) mn(0.5,1.2);
(%o16)                       [[x = 1.0, y = 1.0]]
(%i17) mn(0.1,1.2);
(%o17)                 [[x = - 0.713747, y = 1.2208868]]
\end{verbatim}
\normalsize
We have to be "close enough" to find the (1,1) root with \textbf{mnewton}.
Note that Maxima's function \textbf{mnewton} can't find the desired root starting
 with \verb|(x0,y0) = (2,0.5)| as the textbook example does with eventual success.\\

\subsubsection{Error Messages and Do It Yourself Mnewton}
Let's explore the Newton Raphson method using an easy example which
   \textbf{solve} has no trouble with.
As a by-product, we show that \textbf{mnewton} can deal with a list of equations, rather
than a list of expressions. 
After getting an error message from Maxima's \textbf{mnewton}, we work this same
  problem "by hand", using matrix methods.
A good reference is Chapter 5, Nonlinear Equations, of the text
  "Scientific Computing: An Introductory Survey" (2nd ed.), by Michael T. Heath.
( see webpage:\\
 \verb|http://www.cse.uiuc.edu/heath/scicomp/pubdata/index.html|)
\small
\begin{verbatim}
(%i18) eqns : [x+y=3,x^2+y^2=9]$
(%i19) mn(x0,y0) := mnewton(eqns,[x,y],[x0,y0] )$
(%i20) solve(eqns,[x,y]);
(%o20)                 [[x = 3, y = 0], [x = 0, y = 3]]
(%i21) mn(1,2);
(%o21)                  [[x = 6.37714165E-17, y = 3.0]]
(%i22) mn(-1,4);
(%o22)                  [[x = 6.37714165E-17, y = 3.0]]
(%i23) mn(-2,5);
(%o23)                  [[x = 1.13978659E-16, y = 3.0]]
(%i24) mn(0,0);
Maxima encountered a Lisp error:

 Error in FUNCALL [or a callee]: Zero divisor.

Automatically continuing.
To reenable the Lisp debugger set *debugger-hook* to nil.
(%i25) mn(0,1);
(%o25)                  [[x = 1.13978659E-16, y = 3.0]]
(%i26) mn(2,0);
(%o26)                  [[x = 3.0, y = 1.41267055E-16]]
\end{verbatim}
\normalsize
The "zero divisor" message from the Maxima code for \textbf{mnewton} probably 
  means that the starting point \verb|(x,y) = (0,0)| resulted in an attempted
  division by $0$.
To explore what kind of problems can arise, we implement a naive (ie., a 
  strategy-less) iteration algorithm as presented in Heath's text.
We present the algorithm first in terms of a $2 \times 2$ matrix and
  a two element column vector which are functions of the scalars \verb|(x,y)|.
We then convert the algorithm to a form which works with two element
  column vectors \verb|v| and \verb|b|.
We let the column vector \verb|b| hold the elements \verb|x| and \verb|y| and we 
  seek \verb|b| such that the equation $ f(b) = 0$, where \verb|f| is a column
  vector which defines the problem.
If \verb|b| is an approximate solution of this equation, then $bbetter = b + s$,
  where \verb|s| is the solution of the matrix equation \verb|j(b) . s = -f(b)|, and
  the $2 \times 2$ matrix \verb|j(b)| is the jacobian matrix:
  \verb|j[1,1] : diff( f[1,1],x ), j[1,2] : diff( f[1,1],y )|,
   and \verb|j[2,1] : diff( f[2,1],x), j[2,2] : diff( f[2,1],y )|.  
\small
\begin{verbatim}
(%i1) (fpprintprec:8,ratprint:false)$
(%i2) g : matrix( [x + y -3],[x^2 + y^2 -9] );
                                [  y + x - 3  ]
(%o2)                           [             ]
                                [  2    2     ]
                                [ y  + x  - 9 ]
\end{verbatim}
\newpage
\begin{verbatim}
(%i3) gv : ev(g, x=v[1,1], y=v[2,1] );
                             [ v     + v     - 3 ]
                             [  2, 1    1, 1     ]
(%o3)                        [                   ]
                             [  2       2        ]
                             [ v     + v     - 9 ]
                             [  2, 1    1, 1     ]
(%i4) define(f(v), gv);
                                 [ v     + v     - 3 ]
                                 [  2, 1    1, 1     ]
(%o4)                    f(v) := [                   ]
                                 [  2       2        ]
                                 [ v     + v     - 9 ]
                                 [  2, 1    1, 1     ]
(%i5) b : matrix([1],[2]);
                                     [ 1 ]
(%o5)                                [   ]
                                     [ 2 ]
(%i6) f(b);
                                    [  0  ]
(%o6)                               [     ]
                                    [ - 4 ]
(%i7) (r1 : g[1,1], r2 : g[2,1] )$
(%i8) h : matrix( [diff(r1,x), diff(r1,y)],
               [diff(r2,x), diff(r2,y) ] );
                                 [  1    1  ]
(%o8)                            [          ]
                                 [ 2 x  2 y ]
(%i9) hv : ev(h, x=v[1,1], y=v[2,1] );
                             [    1        1    ]
(%o9)                        [                  ]
                             [ 2 v      2 v     ]
                             [    1, 1     2, 1 ]
(%i10) define( j(v), hv );
                                 [    1        1    ]
(%o10)                   j(v) := [                  ]
                                 [ 2 v      2 v     ]
                                 [    1, 1     2, 1 ]
(%i11) j(b);
                                   [ 1  1 ]
(%o11)                             [      ]
                                   [ 2  4 ]
(%i12) ls : linsolve_by_lu(j(b),-f(b) );
                                [ - 2 ]
(%o12)                         [[     ], false]
                                [  2  ]
(%i13) s :  first(ls) ;
                                    [ - 2 ]
(%o13)                              [     ]
                                    [  2  ]
\end{verbatim}
\newpage
\begin{verbatim}
(%i14) b : b + s;
                                    [ - 1 ]
(%o14)                              [     ]
                                    [  4  ]
(%i15) b : b + first( linsolve_by_lu(j(b),-f(b) )  );
                                    [   1 ]
                                    [ - - ]
                                    [   5 ]
(%o15)                              [     ]
                                    [ 16  ]
                                    [ --  ]
                                    [ 5   ]
(%i16) b : b + first( linsolve_by_lu(j(b),-f(b) )  );
                                   [   1  ]
                                   [ - -- ]
                                   [   85 ]
(%o16)                             [      ]
                                   [ 256  ]
                                   [ ---  ]
                                   [ 85   ]
(%i17) b : b + first( linsolve_by_lu(j(b),-f(b) )  );
                                  [     1   ]
                                  [ - ----- ]
                                  [   21845 ]
(%o17)                            [         ]
                                  [  65536  ]
                                  [  -----  ]
                                  [  21845  ]
(%i18) b : b + float( first( linsolve_by_lu(j(b),-f(b) )  ) );
                             [ - 6.98491931E-10 ]
(%o18)                       [                  ]
                             [       3.0        ]				
(%i19) f(b);
                               [      0.0      ]
(%o19)                         [               ]
                               [ 4.19095159E-9 ]							 
\end{verbatim}
\normalsize
Starting with the guess \verb|(x0,y0) = (1,2)|, this iteration process
  has converged to the approximate solution given by \verb|%o18|, which we
  check as an approximate solution in input \verb|%i19|.
Now let's start with the "dangerous" guess: \verb|(x0,y0) = (0,0)|.
\small
\begin{verbatim}
(%i20) b : matrix( [0],  [0] );
                                     [ 0 ]
(%o20)                               [   ]
                                     [ 0 ]
(%i21) f(b);
                                    [ - 3 ]
(%o21)                              [     ]
                                    [ - 9 ]
(%i22) j(b);
                                   [ 1  1 ]
(%o22)                             [      ]
                                   [ 0  0 ]
(%i23) ls : linsolve_by_lu(j(b),-f(b) );
Division by 0
 -- an error.  To debug this try debugmode(true);
(%i24) determinant(j(b));
(%o24)                                 0
\end{verbatim}
\normalsize
Thus, a check of the non-vanishing of the determinant of the jacobian
  matrix would have kept us out of trouble.

\subsubsection{Automated Code for mymnewton}
Writing code for an arbitrary number of dimensions is a suggested homework problem.
Here we just assume the problem is two dimensional and assume the variables
  are called \verb|x| and \verb|y|.
To check for the "nonvanishing" of the determinant of the jacobian,
   we ask if the absolute value is less than some very small number.
Here is the code, written with notepad2 in a file "mymnewton.mac", 
  placed in my work directory \verb|c:\work2\|.
You can download this file from the author's webpage, and experiment with it.
You can reduce the size of the output on the screen by adding the line
  "\verb|display2d:false,|" in the program, or outside the program in your work
  session.
If you make changes to this code, add some extra "debug" printouts at first
  like "\verb|display(newval1,newval2)|", or "\verb|print(" v1 = ",v1),|"
  to make sure you are on the right track.
Once the program has been "debugged", you can comment out the debug version in your
 work file, copy the whole code to a new section, remove the debug printouts,
  and use as your "working version".
\small
\begin{verbatim}
/*  working version  */   
/* file: mymnewton.mac
    e. woollett, april, 08  */

   
disp("working version mymnewton,
assumes two dimensional problem only,
syntax: 

mymnewton( exprlist, guesslist, numiter )$

exprlist should have the form: [expr1, expr2],
to find (x,y) such that simultaneously expr1=0,expr2=0,
expr1, expr2 should be explicit functions of x and y,
guesslist should be in the form: [xguess,yguess],
numiter = number of iterations ")$

mymnewton( exprlist, guesslist, numiter) :=

   block([numer,ratprint,fpprintprec, small : 1.0e-30 ,
                g,x,y,gv,h,hv,b,v,ls,d  ] , local(f,j),
              
     numer:true, ratprint:false, fpprintprec:8,
      /* g = col vec: row 1=expr 1, row 2=expr 2 depends on (x,y) */
      
     g : matrix( [ exprlist[1] ],[ exprlist[2] ] ),
     display(g),
   
     gv : ev(g, x=v[1,1], y=v[2,1] ), 
   
      /* v is generic col vector */
         
     define( f(v), gv ), 
   
     /* h is jacobian matrix associated with col vec g(x,y)  */
     
     h : matrix( [diff( g[1,1], x), diff( g[1,1], y)],
           [diff( g[2,1], x), diff(g[2,1], y) ] ), 
           
     hv : ev(h, x=v[1,1], y=v[2,1] ), 
   
     define( j(v), hv ), 
     
     /* b is col vec containing (x,y) values  */
     
     b : matrix([ guesslist[1] ],[ guesslist[2] ]),
    
         /*  start iterations  */
         
     for i:0 thru numiter do (
     
       print("  "),
       print(" i = ",i,"    b = ",b,"   f(b) = ",f(b)  ),
       if ( i = 0) then print(" starting values ") else 
                  print("    condition number = ", second(ls) ),
     
              /*  check jacobian determinant  */
              
       d : ( determinant( j(b) ), float(%%) , abs(%%) ),
                
       if ( d < small ) then (     
               print("  abs(det(jacobian)) is ", d,"  program halt " ),
                                  return()  ),  
      
        /*  using 'floatfield arg gets condition number returned */
     
       ls : linsolve_by_lu(j(b),-f(b), 'floatfield ),
      
        /*  improved (hopefully) estimates of values of (x,y)
            which simultaneously satisfy expr1=0 and expr2=0 */
            
       b : b +  first(ls) 
                                                         
      )  /* end do  */
      
   )$   /*  end block  */
\end{verbatim}
\normalsize
\newpage
and here is a little output from this code:
\small
\begin{verbatim}
(%i1) load("mymnewton.mac");
working version mymnewton,
assumes two dimensional problem only,
syntax: 
mymnewton( exprlist, guesslist, numiter )$
exprlist should have the form: [expr1, expr2],
to find (x,y) such that simultaneously expr1=0,expr2=0,
expr1, expr2 should be explicit functions of x and y,
guesslist should be in the form: [xguess,yguess],
numiter = number of iterations 
(%o1)                       c:/work2/mymnewton.mac
(%i2) mymnewton([x+y-3, x^2+y^2-9],[1,2], 5)$
                                  [  y + x - 3  ]
                              g = [             ]
                                  [  2    2     ]
                                  [ y  + x  - 9 ]

   
                 [ 1 ]            [  0  ]
 i =  0     b =  [   ]    f(b) =  [     ] 
                 [ 2 ]            [ - 4 ]
 starting values  
   
                 [ - 1.0 ]            [ 0.0 ]
 i =  1     b =  [       ]    f(b) =  [     ] 
                 [  4.0  ]            [ 8.0 ]
    condition number =  22.5 
   
                 [ - 0.2 ]            [ 0.0  ]
 i =  2     b =  [       ]    f(b) =  [      ] 
                 [  3.2  ]            [ 1.28 ]
    condition number =  19.5 
   
                 [ - 0.0117647 ]            [    0.0    ]
 i =  3     b =  [             ]    f(b) =  [           ] 
                 [  3.0117647  ]            [ 0.0708651 ]
    condition number =  10.92 
   
                 [ - 4.57770657E-5 ]            [ 4.4408921E-16 ]
 i =  4     b =  [                 ]    f(b) =  [               ] 
                 [    3.0000458    ]            [ 2.74666585E-4 ]
    condition number =  7.212872 
   
                 [ - 6.98491837E-10 ]            [      0.0      ]
 i =  5     b =  [                  ]    f(b) =  [               ] 
                 [       3.0        ]            [ 4.19095159E-9 ]
    condition number =  7.000824 
\end{verbatim}
\normalsize
\newpage
Here is a test of the "dangerous" initial condition case:
\small
\begin{verbatim}
(%i3) mymnewton([x+y-3, x^2+y^2-9],[0,0], 5)$
                                  [  y + x - 3  ]
                              g = [             ]
                                  [  2    2     ]
                                  [ y  + x  - 9 ]

   
                 [ 0 ]            [ - 3 ]
 i =  0     b =  [   ]    f(b) =  [     ] 
                 [ 0 ]            [ - 9 ]
 starting values  
  abs(det(jacobian)) is  0.0   program halt  
\end{verbatim}
\normalsize
It is left as a homework problem to incorporate a criterion for finding an 
  "accurate enough" solution without providing the program a number of 
  iterations, and allowing the program to "hide" the details of the iteration process,
  providing the user with an "answer" and an estimate of the likely "error".

\end{document}