% e. woollett
%   july 08 - april 09
% file fltrans.tex this is new ch. 10
% copied from calculus2.tex
% ch. 8 nint.tex is just the quadpack romberg stuff nint.tex
% and ch. 9 bigfloats and arb prec integration
% and ch. 11 fast fourier transforms
% check with new version of maxima
% after cleaning up line numbers, check
% 
% 
% 
% things to add: 
% mailing list example see 
% work3, maxima tricks3a,
%  fourier-series-abs-lambda-pw-mac.txt
% 
% 
% 
% 
% edit with Notepad++, then load into LED for latexing
\documentclass[12pt]{article}
\usepackage[dvips,top=1.5cm,left=1.5cm,right=1.5cm,foot=1cm,bottom=1.5cm]{geometry}
\usepackage{times,amsmath,amsbsy,graphicx,fancyvrb,url}
\usepackage[usenames]{color}
%\definecolor{MyDarkBlue}{rgb}{0,0.08,0.45}
\definecolor{mdb}{rgb}{0.1,0,0.55}
\newcommand{\tcdb}{\textcolor{mdb}}
\newcommand{\tcbr}{\textcolor{BrickRed}}
\newcommand{\tcb}{\textcolor{blue}}
\newcommand{\tcr}{\textcolor{red}}
\urldef\tedhome\url{ http://www.csulb.edu/~woollett/  }
\urldef\tedmail\url{ woollett@charter.net}
%1.  this is for maxima code: red framed bold, footnotesize 
\DefineVerbatimEnvironment%
   {myVerbatim}%
   {Verbatim}%
   {fontfamily=courier,fontseries=b,fontsize=\footnotesize ,frame=single,rulecolor=\color{BrickRed}}
\DefineVerbatimEnvironment%
   {myVerbatim1}%
   {Verbatim}%
   {fontfamily=courier,fontseries=b,fontsize=\scriptsize ,frame=single,rulecolor=\color{BrickRed}}
%2.  this is for blue framed bold 
\DefineVerbatimEnvironment%
   {myVerbatim2}%
   {Verbatim}%
   {fontfamily=courier,fontseries=b,frame=single,rulecolor=\color{blue}}
\DefineVerbatimEnvironment%
   {myVerbatim2s}%
   {Verbatim}%
   {fontfamily=courier,fontseries=b,fontsize=\small,frame=single,rulecolor=\color{blue}}
\DefineVerbatimEnvironment%
   {myVerbatim2f}%
   {Verbatim}%
   {fontfamily=courier,fontseries=b,fontsize=\footnotesize,frame=single,rulecolor=\color{blue}}
% 3.  this is for black framed  bold
\DefineVerbatimEnvironment%
   {myVerbatim3}%
   {Verbatim}%
   {fontfamily= courier, fontseries=b, frame=single}
% 4.  this is for no frame bold
\DefineVerbatimEnvironment%
   {myVerbatim4}%
   {Verbatim}%
   {fontfamily=courier, fontseries=b}
% 6.  for defaults use usual verbatim
\newcommand{\mv}{\Verb[fontfamily=courier,fontseries=b]}
\newcommand{\mvs}{\Verb[fontfamily=courier,fontseries=b,fontsize=\small]}
\newcommand{\mvf}{\Verb[fontfamily=courier,fontseries=b,fontsize=\footnotesize]}

\renewcommand{\thefootnote}{\ensuremath{\fnsymbol{footnote}}}
%%%%%%%%%%%%%%%%%%%%%%%%%%%%%%%%%%%%%%%%%%%%%%%%%%%%%%%%%%%%%%%%%%%%%%%%
%   title page
%%%%%%%%%%%%%%%%%%%%%%%%%%%%%%%%%%%%%%%%%%%%%%%%%%%%%%%%%%%%%%%%%%%%%%

\title{  Maxima by Example:\\ Ch.10: Fourier Series, Fourier and Laplace Transforms
            \thanks{This version uses \textbf{Maxima 5.18.1}. This is a live
            document. Check \; \textbf{ \tedhome } \; for the latest version of these notes. Send comments and
			 suggestions to \textbf{\tedmail} } }


\author{ Edwin L. Woollett}
\date{\today}
%%%%%%%%%%%%%%%%%%%%%%%%%%%%%%%%%%%%%%%%%
%          document
%%%%%%%%%%%%%%%%%%%%%%%%%%%%%%%%%%%%%%%%%%
\begin{document}
%
%\small
\maketitle
\tableofcontents
\numberwithin{equation}{section}
\newpage
\begin{myVerbatim2} 
COPYING AND DISTRIBUTION POLICY    
This document is part of a series of notes titled
"Maxima by Example" and is made available
via the author's webpage http://www.csulb.edu/~woollett/
to aid new users of the Maxima computer algebra system.	
	
NON-PROFIT PRINTING AND DISTRIBUTION IS PERMITTED.
	
You may make copies of this document and distribute them
to others as long as you charge no more than the costs of printing.	

These notes (with some modifications) will be published in book form
eventually via Lulu.com in an arrangement which will continue
to allow unlimited free download of the pdf files as well as the option
of ordering a low cost paperbound version of these notes.
\end{myVerbatim2}	
\smallskip
\noindent \tcbr{Feedback from readers is the best way for this series of notes
  to become more helpful to new users of Maxima}.
\tcdb{\emph{All} comments and suggestions for improvements will be appreciated and
  carefully considered}.
\smallskip
\begin{myVerbatim2s}
The Maxima code examples in this chapter were generated 
using the XMaxima graphics interface on a Windows XP computer, and copied
into a fancy verbatim environment in a latex file which uses the fancyvrb
and color packages.
We use qdraw.mac for plots (see Ch.5), which uses draw2d defined
in share/draw/draw.lisp.
\end{myVerbatim2s}  
\smallskip
\begin{myVerbatim}
Maxima.sourceforge.net. Maxima, a Computer Algebra System. Version 5.18.1
 (2009). http://maxima.sourceforge.net/
\end{myVerbatim}
\newpage
\setcounter{section}{10}
\subsection{Introduction}
In chapter 10 we discuss the Fourier series expansion of a given function, the
  computation of Fourier transform integrals, and the calculation
  of Laplace transforms (and inverse Laplace transforms).
\subsection{Fourier Series Expansion of a Function}
\subsubsection{Fourier Series Expansion of a Function over $(-\boldsymbol{\pi,\,\pi})$}
A Fourier series expansion designed to represent a given function
  \textbf{f(x)} defined over a finite interval $(-\boldsymbol{\pi, \pi})$, is a sum of terms
\begin{equation}  \label{fxpi}
\mathbf{f(x) = \frac{1}{2} \, a_{0} + \sum_{n=1}^{\infty} \left[ 
   a_{n}\,\boldsymbol{\cos} (n\,x) + b_{n} \,\boldsymbol{\sin} (n\,x) \right] }
\end{equation}
  
\noindent and the constant coefficients ($\mathbf{a_{n},\, b_{n}}$) are 
\begin{equation} \label{fcoeff}
 \mathbf{ a_{n} = \frac{1}{\boldsymbol{\pi}} \, \int_{-\boldsymbol{\pi}}^{\boldsymbol{\pi}} f(y)\,
   \boldsymbol{\cos} (n\,y) \, dy}, \quad
   \mathbf{ b_{n} = \frac{1}{\boldsymbol{\pi}} \, \int_{-\boldsymbol{\pi}}^{\boldsymbol{\pi}} f(y)\,
   \boldsymbol{\sin} (n\,y) \, dy}. 
\end{equation}
(For a derivation of these equations see Sec.\ref{expfs} and Eqs.(\ref{Eq:piex1}) and
  (\ref{Eq:piex2}).)\\

\noindent Whether or not you are working with a function which is periodic, the Fourier
  expansion will represent a periodic function for all \textbf{x}, in this case
  having period $\boldsymbol{2\,\pi}$.\\

\noindent The first term of the expansion
\begin{equation}
\mathbf{ \frac{1}{2} \, a_0  =
 \frac{1}{2\, \boldsymbol{\pi}}  \int_{-\boldsymbol{\pi}}^{\boldsymbol{\pi}} f(y)\,dy =
 \langle f(x) \rangle }
\end{equation}
is the (un-weighted) average of \textbf{f(x)} over the domain $(-\boldsymbol{\pi},\boldsymbol{\pi})$.
Hence $\mathbf{a_{0}}$ will always be twice the average value of the function over
  the domain.\\

\noindent If \textbf{f(x)} is an even function ($\mathbf{f(-x) =  f(x)}$),
  then only the $\mathbf{\boldsymbol{\cos}(n\,x)}$ terms
  contribute. 
If \textbf{f(x)} is an odd function ($\mathbf{f(-x) =  -f(x)}$),
  then only the $\mathbf{\boldsymbol{\sin}(n\,x)}$ terms contribute. \\
  
\noindent If you are trying to find a fourier expansion for a function or expression \textbf{f(x)} 
   which is a homemade function which involves "if..then..else"
  constructs, it is necessary to do the preliminary work "by hand".\\
  
\noindent On the other hand, if the given function is a smooth function defined in terms of elementary
  functions and polynomials, or includes \textbf{abs} of elements of the
  function, one can use the \textbf{fourie} package to do most of the work
  in obtaining the desired Fourier series expansion.
This package is \mv|calculus/fourie.mac|, and there is also a short \textbf{demo} file
  \mv|fourie.dem|.\\
  
\noindent Although many secondary functions defined in this mac file are usable once you
  load the package, they are not documented in the Maxima manual.
The Maxima manual gives a brief definition of the primary tools available, but
  gives no examples of use, nor is there an \textbf{example} file for such
  primary functions as \textbf{fourier}, \textbf{foursimp}, and \textbf{fourexpand}.\\
  
\noindent If the Fourier series integrals needed for the coefficients are too difficult
  for \textbf{integrate}, you should use the Quadpack function \mvs|quad_qawo|
  described in Chapter 8, Numerical Integration.  
\newpage
\subsubsection{Fourier Series Expansion of $\mathbf{f(x) = x}$ \; over ($-\boldsymbol{\pi,\, \pi}$)}
We first use the coefficient formulas Eq.(\ref{fcoeff}) to find the Fourier expansions
  of this simple linear function $\mathbf{f(x) = x}$ by hand.\\
  
\noindent Because the average value of \textbf{f(x)} over the domain $(-\boldsymbol{\pi,\pi})$ is zero,
   $\mathbf{a_{0} = 0}$.
Because \textbf{f(x)} is an odd function, we have $\mathbf{a_{n} = 0}$
  for all $\mathbf{n > 0}$.
\begin{myVerbatim}
(%i1) (declare(n,integer), assume(n > 0 ), facts() );
(%o1)                      [kind(n, integer), n > 0]
(%i2) define(b(n),integrate(x*sin(n*x),x,-%pi,%pi)/%pi );
                                               n
                                        2 (- 1)
(%o2)                         b(n) := - --------
                                           n
(%i3) map('b,makelist(i,i,1,7));
                                   2    1  2    1  2
(%o3)                     [2, - 1, -, - -, -, - -, -]
                                   3    2  5    3  7
(%i4) fs(nmax) := sum( b(m)*sin(m*x),m,1,nmax)$
(%i5) map('fs,[1,2,3,4] );
                                      2 sin(3 x)
(%o5) [2 sin(x), 2 sin(x) - sin(2 x), ---------- - sin(2 x) + 2 sin(x), 
                                          3
                                   sin(4 x)   2 sin(3 x)
                                 - -------- + ---------- - sin(2 x) + 2 sin(x)]
                                      2           3
\end{myVerbatim}
The list contains, in order, the lowest approximation $\mathbf{2\,\boldsymbol{\sin}(x)}$
  which retains only the $\mathbf{n = 1}$ term in the expansion, the two term approximation
  $\mathbf{2\,\boldsymbol{\sin}(x) - \boldsymbol{\sin}(2\,x)}$,
  which includes the $\mathbf{n = 1,\, 2}$ terms, etc.
We now load the \textbf{draw} package and the \textbf{qdraw} package (the
  latter available with Ch. 5 material on the author's webpage)
  to make two simple plots.
We first make a plot showing the function $\mathbf{f(x) = x}$ in \textcolor{blue}{blue}, the
  one term approximation (\mvf|fs(1) = 2 sin(x)|) in \textcolor{red}{red}, and the
  two term approximation (\mvf|fs(2) = 2 sin(x) - sin(2 x)|) in \textcolor{PineGreen}{green}.
\begin{myVerbatim}
(%i6) (load(draw), load(qdraw) )$
               qdraw(...), qdensity(...), syntax: type qdraw(); 

(%i7) qdraw( xr(-5.6,5.6),yr(-4,4),
               ex([x,fs(1),fs(2)],x,-%pi,%pi),key(bottom) )$
\end{myVerbatim}
In this plot (see next page), we have taken control of the \textbf{x} and \textbf{y} range, using
  the approximate fudge factor \mv|1.4| to relate the horizontal canvas
  extent to the vertical canvas extent (see our discussion in Ch. 5 if this is
  all new to you) to get the geometry approximately correct.
In the next plot we include one, two, three and four term approximations.
\begin{myVerbatim}
(%i8) qdraw( xr(-5.6,5.6),yr(-4,4),
               ex([x,fs(1),fs(2),fs(3),fs(4)],x,-%pi,%pi),key(bottom) )$
\end{myVerbatim}
with the four term approximation in purple being a closer approximation to
  $\mathbf{f(x) = x}$ (see the figure on the next page).
\newpage
\begin{figure} [ht]
   \centerline{\includegraphics[scale=.8]{ch7p3.eps} }
	\caption{ One and Two Term Approximations  }
\end{figure}   
% eps code for two term approx  
%(%i9) qdraw( xr(-5.6,5.6),yr(-4,4), line(-5.6,0,5.6,0,lw(3) ),
%          line(0,-4,0,4,lw(3) ),
%          ex([x,fs(1),fs(2)],x,-%pi,%pi ), key(bottom),
%         pic(eps,"ch7p3",font("Times-Bold",20) ) )$ 
%
%\smallskip
\begin{figure} [h]
   \centerline{\includegraphics[scale=1.2]{ch7p4.eps} }
	\caption{ 1, 2, 3, and 4 Term Approximations  }
\end{figure}     
% eps code for 1,2,3,and 4 term approximations
% (%i11) qdraw( xr(-5.6,5.6),yr(-4,4), line(-5.6,0,5.6,0,lw(3) ),
%          line(0,-4,0,4,lw(3) ),
%          ex([x,fs(1),fs(2),fs(3),fs(4)],x,-%pi,%pi ), key(bottom),
%         pic(eps,"ch7p4",font("Times-Bold",20) ) )$ 
%
%\newpage

\subsubsection{The \textbf{calculus/fourie.mac} Package: \textbf{fourier}, \textbf{foursimp}, \textbf{fourexpand} }
Now let's show how to get the expansions of this simple linear function 
  using the package \textbf{calculus/fourie.mac}. A curious feature of \textbf{fourie.mac} is 
  the role that the symbol \mvs|n| plays in the calculation of the fourier coefficients.
If you look at the code of \textbf{calculus/fourie.mac} (about a four page text file), 
  you will see that the function \textbf{fourier} calls either \textbf{fourcos},
  \textbf{foursin}, or \textbf{fourcoeff}, and each of these latter functions
  declare \mvs|n| to be a local variable, and then have the statement (inside the
  \textbf{block}) \mvs|assume( n > 0 )|, which "leaks out" of the \textbf{block}
  to become a global assumption.
These functions then call a package function \textbf{adefint(..)} to integrate
  the function to be expanded times either $\mathbf{\boldsymbol{\cos}(n\,\boldsymbol{\pi}\,x/p)}$
  or $\mathbf{\boldsymbol{\sin}(n\,\boldsymbol{\pi}\,x/p)}$, (where $\mathbf{p}$ will be
  $\boldsymbol{\pi}$ for our expansion domain here).\\
  
\noindent Finally, after dealing with possible instances of \textbf{abs(..)} in the function
  to be expanded, \textbf{adefint} calls either \textbf{ldefint} or \textbf{integrate}
  with an argument involving these same \mv|n| dependent trig functions.\\

\noindent Here is a small test of that leakage for both an \textbf{assume} statement
  and a \textbf{declare} statement.
\begin{myVerbatim}
(%i1) facts();
(%o1)                                 []
(%i2) f(m) := block([n],declare(n,integer),assume( n > 0 ),
                 if m < 2 then n :2 else n:3,(2*n*m) )$
(%i3) f(1);
(%o3)                                  4
(%i4) facts();
(%o4)                      [kind(n, integer), n > 0]
(%i5) is(n>0);
(%o5)                                true
\end{myVerbatim}
Another curious feature of the \mv|fourie.mac| package is that \mv|n| is not declared to
  be an integer, and unless the user of the package does this first, the integration
  routine may ask questions about signs which may seem irrelevant to the result.
We avoid that trap here and use \mv|declare(n, integer)| before calling
  \textbf{fourier}.
\begin{myVerbatim}
(%i1) facts();
(%o1)                                 []
(%i2) (load(fourie), facts() );
(%o2)                                 []
(%i3) (declare(n,integer), facts() );
(%o3)                         [kind(n, integer)]
(%i4) clist : fourier(x,x,%pi);
(%t4)                               a  = 0
                                     0

(%t5)                               a  = 0
                                     n

                                              n
                                       2 (- 1)
(%t6)                           b  = - --------
                                 n        n

(%o6)                           [%t4, %t5, %t6]
(%i7) facts();
(%o7)                      [kind(n, integer), n > 0]
(%i8) fs(nmax) := fourexpand(clist,x,%pi, nmax )$
(%i9) map( 'fs, [1,2,3,4] );
                                      2 sin(3 x)
(%o9) [2 sin(x), 2 sin(x) - sin(2 x), ---------- - sin(2 x) + 2 sin(x), 
                                          3
                                   sin(4 x)   2 sin(3 x)
                                 - -------- + ---------- - sin(2 x) + 2 sin(x)]
                                      2           3
(%i10) b(n);
(%o10)                               b(n)
\end{myVerbatim}
Some comments: you can use \textbf{foursimp} to get extra simplification of the
  fourier coefficients (if needed) before defining what I call \textbf{clist} (for
  coefficient list) which is the list \textbf{fourexpand} needs to generate the
  expansions you want.
Note \textbf{fourier} produces a separate output for $\mathbf{a_{0}}$ and $\mathbf{a_{n}}$,
  with the second meant for $\mathbf{n = 1,\,2,\,\ldots }$, and there is never an
  output for $\mathbf{b_{0}}$.  
We have defined the small function \textbf{fs(nmax)} to make it easier
  to call \textbf{fourexpand} with any value of $\mathbf{nmax}$ and to
  allow the function to be mapped onto a list of integers to show
  us the first few approximations.  
Note that once you call \textbf{fourier}, the assumption $\mathbf{n > 0}$ becomes
  a global fact.
Also note that the \textbf{fourie.mac} package does not define a Maxima function $\mathbf{b(n)}$,
  although you could use:
\begin{myVerbatim}
(%i11) define(b(n),rhs(%t6) );
                                               n
                                        2 (- 1)
(%o11)                        b(n) := - --------
                                           n
(%i12) map( 'b, makelist(i,i,1,7) );
                                   2    1  2    1  2
(%o12)                    [2, - 1, -, - -, -, - -, -]
                                   3    2  5    3  7
\end{myVerbatim} 
If you look at the Maxima code in \mv|fourie.mac|, you see that because
  we have an "odd" function $\mathbf{f(x) = x}$, \textbf{fourier} calls
  the package function \textbf{foursin}, which calls package function
  \textbf{adefint}, which calls the core Maxima function \textbf{ldefint}
 for this case.\\

\noindent Those expansion expressions can then be used for plots as above.
\subsubsection{Fourier Series Expansion of a Function  Over ($\mathbf{-p,\, p}$)}
The Fourier series expansion of a function defined over the interval
  $\mathbf{-p \le x \le p}$ (and whose Fourier expansion will represent
  a function which has a period $\mathbf{2\,p}$) can be found from the expansion over
  the interval $(-\boldsymbol{\pi, \pi})$ which we have been using above
  by a simple change of variables in the integrals which appear in
  Eqs.(\ref{fcoeff}) and (\ref{fxpi}).\\
  
\noindent However, we will simply use the results derived in Sec.\ref{expfs}, and written down
  in Eqns (\ref{Eq:pex1}) and (\ref{Eq:pex2}).
\begin{equation}  \label{pfexpan}
\mathbf{f(x) = \frac{1}{2} \, a_{0} + \sum_{n=1}^{\infty} \left[ 
   a_{n}\,\boldsymbol{\cos} \left( \frac{ \boldsymbol{\pi}\,n\,x}{p} \right) +
   b_{n} \,\boldsymbol{\sin} \left( \frac{ \boldsymbol{\pi}\,n\,x}{p} \right) \right] }
\end{equation}
with the corresponding coefficients (for $\mathbf{a_{n}}$, $\mathbf{n = 0, 1, ...}$,
  for $\mathbf{b_{n}}$, $\mathbf{n = 1, ...}$):
\begin{equation} \label{pfcoeff}
\mathbf{ a_{n} = \frac{1}{p} \, \int_{-p}^{p} f(y)\,
   \boldsymbol{\cos} \left( \frac{ \boldsymbol{\pi}\,n\,y}{p} \right) \, dy}, \quad
   \mathbf{ b_{n} = \frac{1}{p} \, \int_{-p}^{p} f(y)\,
   \boldsymbol{\sin} \left( \frac{ \boldsymbol{\pi}\,n\,y}{p} \right) \, dy}. 
\end{equation}
We need to warn the user of the package \textbf{fourie.mac} that they use the
  symbol $\mathbf{a_{0}}$ to mean \textit{our} $\mathbf{a_{0}/2}$.  \\
  
\noindent \textit{Our} $\mathbf{a_{0}}$ is defined by
\begin{equation}
\mathbf{ a_{0} = \frac{1}{p} \, \int_{-p}^{p} f(x)\,dx }
\end{equation}
The \textbf{fourie.mac} package's definition of \textit{their} $\mathbf{a_{0}}$ is
\begin{equation} \label{fa0}
\mathbf{ [a_0]_{fourie.mac} = \frac{1}{2\,p} \, \int_{-p}^{p} f(x)\,dx }
\end{equation}
 which defines the average value of the function over the domain
 and which becomes the first term of the fourier series expansion they provide.
\subsubsection{Fourier Series Expansion of the Function $\mathbf{\vert x \vert}$ }
We define the function to have period $\mathbf{4}$ with 
  $\mathbf{f(x) = \vert x \vert}$ for $\mathbf{-2 \leq x \leq 2}$.   
This function  is an even function of \textbf{x} so the $\boldsymbol{\sin}$
  coefficients $\mathbf{b_{n}}$ are all zero.
The average value of $\mathbf{\vert x \vert}$ over the domain $\mathbf{(-2,2)}$
 is greater than zero  so we willl have a non-zero coefficient $\mathbf{a_{0}}$
 which we will calculate separately. We do this calculation by hand here.\\
 
\noindent Note that the Maxima function \textbf{integrate} cannot cope with \textbf{abs(x)}:
\begin{myVerbatim}
(%i13) integrate(abs(x)*cos(n*%pi*x/2),x,-2,2)/2;
                           2
                          /
                          [               %pi n x
                          I    abs(x) cos(-------) dx
                          ]                  2
                          /
                           - 2
(%o13)                    ---------------------------
                                       2
\end{myVerbatim}
so we will split up the region of integration into two sub-intervals:
  $\mathbf{-2 \leq x \leq 0}$, in which $\mathbf{\vert x \vert = -x}$, and  the interval 
   $\mathbf{0 \leq x \leq  2}$ in which $\mathbf{\vert x \vert = x}$.
We will use the formula Eq. (\ref{pfcoeff})   for the coefficients $\mathbf{a_{n}}$
  (note that $\mathbf{a_{n} = 0}$ if \textbf{n} is even)
  and the expansion formula Eq. (\ref{pfexpan}).
\begin{myVerbatim}
(%i1) (declare(n,integer),assume(n > 0), facts() );
(%o1)                      [kind(n, integer), n > 0]
(%i2) a0 :integrate(-x,x,-2,0)/2 + integrate(x,x,0,2)/2;
(%o2)                                  2
(%i3) an : integrate((-x)*cos(n*%pi*x/2),x,-2,0)/2 +
           integrate(x*cos(n*%pi*x/2),x,0,2)/2;
                                     n
                              4 (- 1)       4
(%o3)                         -------- - -------
                                 2  2       2  2
                              %pi  n     %pi  n
(%i4) an : (ratsimp(an), factor(%%) );
                                        n
                                4 ((- 1)  - 1)
(%o4)                           --------------
                                      2  2
                                   %pi  n
(%i5) define(a(n),an);
                                            n
                                    4 ((- 1)  - 1)
(%o5)                       a(n) := --------------
                                          2  2
                                       %pi  n
(%i6) map( 'a, [1,2,3,4,5] );
                          8           8             8
(%o6)                 [- ----, 0, - ------, 0, - -------]
                            2            2             2
                         %pi        9 %pi        25 %pi
(%i7) fs(nmax) := a0/2 + sum(a(m)*cos(m*%pi*x/2),m,1,nmax)$
\end{myVerbatim}
\newpage
\begin{myVerbatim}
(%i8) map('fs, [1, 3, 5] );
                 %pi x           3 %pi x          %pi x
           8 cos(-----)    8 cos(-------)   8 cos(-----)
                   2                2               2
(%o8) [1 - ------------, - -------------- - ------------ + 1, 
                  2                 2              2
               %pi             9 %pi            %pi
                                  5 %pi x          3 %pi x          %pi x
                            8 cos(-------)   8 cos(-------)   8 cos(-----)
                                     2                2               2
                          - -------------- - -------------- - ------------ + 1]
                                     2                2              2
                               25 %pi            9 %pi            %pi
(%i9) (load(draw),load(qdraw))$
               qdraw(...), qdensity(...), syntax: type qdraw(); 

(%i10) qdraw( ex([abs(x),fs(1)],x,-2,2),key(bottom) )$
(%i11) qdraw( ex([abs(x),fs(1),fs(3)],x,-2,2),key(bottom) )$
(%i12) qdraw( ex([abs(x),fs(5) ],x,-2,2),key(bottom) )$
\end{myVerbatim}
The first function in the plot list is $\mathbf{\vert x \vert}$, represented by
  \textbf{abs(x)}, which appears in the color blue.
We see that the expansion out to $\mathbf{n = 5}$ provides a close fit to $\mathbf{\vert x \vert}$.
Here is that comparison:
\begin{figure} [h]
   \centerline{\includegraphics[scale=1.2]{ch7p5.eps} }
	\caption{$\mathbf{n = 5}$ Approximation to $\mathbf{\vert x \vert}$  }
\end{figure}   

% eps code for n = 5 approximation
% qdraw( line(0,0,0,2,lw(3) ),
%       ex([abs(x),fs(5) ],x,-2,2 ), key(bottom),
%       pic(eps,"ch7p5",font("Times-Bold",20) ) )$
%
We now try out the package \textbf{fourie.mac} on this example:
\begin{myVerbatim}
(%i1) ( load(fourie),facts() );
(%o1)                                 []
(%i2) (declare(n,integer),facts());
(%o2)                         [kind(n, integer)]
\end{myVerbatim}
\newpage
\begin{myVerbatim}
(%i3) fourier(abs(x),x,2);
(%t3)                               a  = 1
                                     0

                                        n
                                 4 (- 1)       4
(%t4)                       a  = -------- - -------
                             n      2  2       2  2
                                 %pi  n     %pi  n

(%t5)                               b  = 0
                                     n

(%o5)                           [%t3, %t4, %t5]
(%i6) clist : foursimp(%);
(%t6)                               a  = 1
                                     0

                                           n
                                   4 ((- 1)  - 1)
(%t7)                         a  = --------------
                               n         2  2
                                      %pi  n

(%t8)                               b  = 0
                                     n

(%o8)                           [%t6, %t7, %t8]
(%i9) facts();
(%o9)                      [kind(n, integer), n > 0]
(%i10) fs(nmax) := fourexpand(clist,x,2,nmax )$
(%i11) map( 'fs, [1,3,5] );
                  %pi x           3 %pi x          %pi x
            8 cos(-----)    8 cos(-------)   8 cos(-----)
                    2                2               2
(%o11) [1 - ------------, - -------------- - ------------ + 1, 
                   2                 2              2
                %pi             9 %pi            %pi
                                  5 %pi x          3 %pi x          %pi x
                            8 cos(-------)   8 cos(-------)   8 cos(-----)
                                     2                2               2
                          - -------------- - -------------- - ------------ + 1]
                                     2                2              2
                               25 %pi            9 %pi            %pi
\end{myVerbatim}
Notice that $\mathbf{(a_{0})_{fourie} = 1 = \frac{1}{2} \, (a_{0})_{our def}}$, 
  (see Eq. (\ref{fa0}) ) so that \mv|fourie.mac|'s expansion starts off with 
  the term $\mathbf{a_{0}}$ rather
  than $\mathbf{\frac{1}{2} \, a_{0}}$.
Of course, the actual end results look the same, with the first
  term in this example being \textbf{1}, which is the average value of
  $\mathbf{\vert x \vert}$ over the domain $\mathbf{(-2, 2)}$.\\  
  
\noindent Here we chose to use the package function \textbf{foursimp} to simplify
  the appearance of the coefficients.
We see that \mv|fourie.mac| is able to cope with the appearance of \textbf{abs(x)} in
  the integrand, and produces the same coefficients and expansions as were found
  "by hand".
\subsubsection{Fourier Series Expansion of a Rectangular Pulse}
We define \textbf{f(x)} to be a function of period \textbf{4}, with $\mathbf{f = 0}$
  for $\mathbf{-2 \leq x < -1}$, $\mathbf{3/2}$ for $\mathbf{-1 \leq x \leq 1}$ and
  $\mathbf{f = 0}$ for $\mathbf{1 < x \leq 2}$. 
\begin{myVerbatim}
(%i1) f(x):= if x >= -1 and x <= 1 then 3/2 else 0$
(%i2) map('f, [-3/2,-1,0,1,3/2] );
                                    3  3  3
(%o2)                          [0, -, -, -, 0]
                                    2  2  2
(%i3) (load(draw),load(qdraw) )$
               qdraw(...), qdensity(...), syntax: type qdraw(); 

(%i4) qdraw( yr(-0.5,2), ex1(f(x),x,-2,2,lw(5),lc(blue) ) )$
\end{myVerbatim}
The plot of \textbf{f(x)} shows a square pulse with height $\mathbf{3/2}$ above the
  \textbf{x} axis and with a width of $\mathbf{2}$ units over the
  interval $\mathbf{-1 \leq x \leq 1}$.
Over the rest of the domain $\mathbf{-2 \leq x \leq 2}$, \textbf{f(x)} is defined to be
  zero.  
\smallskip
\begin{figure} [h]
   \centerline{\includegraphics[scale=0.7]{ch7p6.eps} }
	\caption{Rectangular Pulse of Height $3/2$}
\end{figure}   
% eps code for square pulse plot
% qdraw( yr(-0.5,2), line(-1,0,1,0,lw(4)),
%              line(0,-0.5,0,2,lw(4) ),
%       ex1(f(x),x,-2,2,lw(10),lc(blue) ),
%       pic(eps,"ch7p6",font("Times-Bold",18) ) )$
%
% \smallskip  

\noindent Although we can make a plot of the function \textbf{f(x)} as defined ,
   Maxima's \textbf{integrate} function cannot
  do anything useful with it, and hence neither can \mv|fourie.mac| at
   the time of writing.
\begin{myVerbatim}
(%i5) integrate(f(x),x,-2,2);
               2
              /
              [                                     3
(%o5)        I    (if (x >= - 1) and (x <= 1) then - else 0) dx
              ]                                     2
              /
               - 2
\end{myVerbatim}
Hence we must compute the Fourier series expansion "by hand".
We see that \textbf{f(x)} is an even function\\
 ($\mathbf{f(-x) = f(x)}$),
  so only the $\mathbf{\boldsymbol{\cos}(n\,\boldsymbol{\pi}\,x/2)}$
  terms will contribute, so we only need to calculate the $\mathbf{a_{n}}$ coefficients.
\begin{myVerbatim}
(%i6) (declare(n,integer), assume(n>0),facts() );
(%o6)                      [kind(n, integer), n > 0]
\end{myVerbatim}
\newpage
\begin{myVerbatim}
(%i7) a0 : (1/2)*integrate( (3/2),x,-1,1 );
                                       3
(%o7)                                  -
                                       2
(%i8) define(a(n),(1/2)*integrate((3/2)*cos(n*%pi*x/2),x,-1,1));
                                           %pi n
                                     3 sin(-----)
                                             2
(%o8)                        a(n) := ------------
                                        %pi n
(%i9) map( 'a, makelist(i,i,1,7) );
                       3         1        3           3
(%o9)                [---, 0, - ---, 0, -----, 0, - -----]
                      %pi       %pi     5 %pi       7 %pi
\end{myVerbatim}
We see that for $\mathbf{n>0}$, $\mathbf{a_{n} = 0}$ for \textbf{n} even, and the non-zero  coefficients have
   $\mathbf{n = 1, 3, 5, 7, ...}$.
Hence we only get a better approximation if we increase $\mathbf{n_{max}}$ by $\mathbf{2}$ each
  time.
\begin{myVerbatim}
(%i10) fs(nmax) := a0/2 + sum( a(m)*cos(m*%pi*x/2),m,1,nmax )$
(%i11) map( 'fs, [1,3] );
                    %pi x             3 %pi x          %pi x
              3 cos(-----)        cos(-------)   3 cos(-----)
                      2      3           2               2      3
(%o11)        [------------ + -, - ------------ + ------------ + -]
                  %pi        4        %pi            %pi        4
(%i12) qdraw( yr(-0.5,2),ex([f(x),fs(1),fs(3) ],x,-2,2) )$
(%i13) qdraw( yr(-0.5,2),ex([f(x),fs(11) ],x,-2,2) )$
\end{myVerbatim}
The plot with the approximations \mv|fs(1)| and \mv|fs(3)| was drawn first.
\smallskip
\begin{figure} [h]
   \centerline{\includegraphics[scale=0.9]{ch7p7.eps} }
	\caption{$\mathbf{nmax = 1, 3}$ Approx. to Rectangular Pulse  }
\end{figure}   
% eps code for rectangular pulse plot
% qdraw( yr(-0.5,2), line(-2,0,2,0,lw(3)),
%              line(0,-0.5,0,2,lw(3) ),
%       ex([f(x),fs(1),fs(3)],x,-2,2 ),
%       pic(eps,"ch7p7",font("Times-Bold",20) ) )$
%
%\newpage

\newpage
\noindent Then a plot showing the \mv|fs(11)| approximation:
\begin{figure} [h!]
   \centerline{\includegraphics[scale=0.7]{ch7p8.eps} }
	\caption{$\mathbf{nmax = 11}$ Approx. to Rectangular Pulse  }
\end{figure}   
% eps code for rectangular pulse plot
% qdraw( yr(-0.5,2), line(-2,0,2,0,lw(3)),
%              line(0,-0.5,0,2,lw(3) ),
%       ex([f(x),fs(11) ],x,-2,2 ),
%       pic(eps,"ch7p8",font("Times-Bold",20) ) )$
%

\subsubsection{Fourier Series Expansion of a Two Element Pulse}
We compute the Fourier series expansion of a function 
  which has the definition over the interval $\mathbf{(-10,10)}$
  (and whose Fourier series expansion will be a function with period $\mathbf{20}$
  for all \textbf{x})
  given by:   for $\mathbf{-10 \leq x < -5}$, $\mathbf{f = 0}$,
  and for $\mathbf{-5 \leq x < 0}$, $\mathbf{f = -5}$,
  and for $\mathbf{0 \leq x \leq 5}$, $\mathbf{f = 5}$,
  and for $\mathbf{5 < x \leq 10}$, $\mathbf{f = 0}$.
First we define such a function for our plots.
\begin{myVerbatim}
(%i1) f(x):= if x >= -5 and x < 0 then -5
              elseif x >= 0 and x <= 5 then 5 else 0$
(%i2) map('f,[-6,-5,-1,0,1,5,6]);
(%o2)                      [0, - 5, - 5, 5, 5, 5, 0]
\end{myVerbatim}
and plot the function
\begin{myVerbatim}
(%i3) ( load(draw),load(qdraw) )$
               qdraw(...), qdensity(...), syntax: type qdraw(); 
(%i4) qdraw( yr(-8,8), ex1(f(x),x,-10,10,lw(5),lc(blue) ) )$
\end{myVerbatim}
which looks like
\smallskip
\begin{figure} [h]
   \centerline{\includegraphics[scale=0.5]{ch7p15.eps} }
	\caption{Two Element Pulse with Period $\mathbf{20}$ }
\end{figure}   

% eps code for two element pulse plot
% qdraw( yr(-8,8), line(-5,0,5,0,lw(3)),line(0,-8,0,-5,lw(3) ),
%       line(0,5,0,8,lw(3) ),
%       ex1(f(x),x,-10,10,lw(10),lc(blue) ),
%       pic(eps,"ch7p15",font("Times-Bold",18) ) )$
%
\newpage
\noindent Since we have an odd function, only terms like
   $\mathbf{b_{n} \, \boldsymbol{\sin}(n\,\boldsymbol{\pi} \, x/10)}$ will contribute to the
   expansion, but, just for practice, we find $\mathbf{a_{n}}$ too.
\begin{myVerbatim1}
(%i5) a0 : (1/10)*( integrate( -5, x, -5, 0 ) +
                              integrate( 5, x, 0, 5 )  );
(%o5)                                  0
(%i6) an : (1/10)*(integrate( -5*cos( n*%pi*x/10 ), x, -5, 0 ) +
                       integrate(5*cos(n*%pi*x/10), x, 0, 5 )  );
(%o6)                                  0
(%i7) bn : ( (1/10)*(integrate( -5*sin(n*%pi*x/10), x, -5, 0 ) +
             integrate( 5*sin(n*%pi*x/10), x, 0, 5 )  ),
              ratsimp(%%) );
                                      %pi n
                               10 cos(-----) - 10
                                        2
(%o7)                        - ------------------
                                     %pi n
(%i8) define( b(n), bn );
                                          %pi n
                                   10 cos(-----) - 10
                                            2
(%o8)                    b(n) := - ------------------
                                         %pi n
(%i9) map('b,makelist(i,i,1,7));
                     10   10    10        2    10     10
(%o9)              [---, ---, -----, 0, ---, -----, -----]
                     %pi  %pi  3 %pi     %pi  3 %pi  7 %pi
(%i10) fs(nmax) :=  sum( b(m)*sin(m*%pi*x/10), m, 1, nmax )$
(%i11) map('fs,[1,2,3]);
               %pi x          %pi x           %pi x
        10 sin(-----)  10 sin(-----)   10 sin(-----)
                10              5              10
(%o11) [-------------, ------------- + -------------, 
             %pi            %pi             %pi
                                      3 %pi x           %pi x           %pi x
                               10 sin(-------)   10 sin(-----)   10 sin(-----)
                                        10                5              10
                               --------------- + ------------- + -------------]
                                    3 %pi             %pi             %pi
(%i12) qdraw(  xr(-15, 15), yr(-10, 10),
                            ex( [f(x), fs(1), fs(2) ], x, -10, 10 ) )$
\end{myVerbatim1}
The plot of \textbf{f(x)} with the two lowest approximations looks like
\begin{figure} [h]
   \centerline{\includegraphics[scale=0.9]{ch7p16.eps} }
	\caption{Two Lowest Order Approximations }
\end{figure} 
% eps code for two element pulse plot
% qdraw(xr(-15, 15), yr(-10,10), line(-5,0,5,0,lw(3)),line(0,-10,0,-5,lw(3) ),
%       line(0,5,0,10,lw(3) ), lw(10),
%       ex( [f(x), fs(1), fs(2) ], x, -10, 10 ),
%       pic(eps,"ch7p16",font("Times-Bold",18) ) )$

\newpage
\noindent The expansion \textbf{fs(11)} does pretty well as a rough approximation
\begin{myVerbatim}
(%i13) qdraw( xr(-15, 15), yr(-10, 10),
                    ex(  [ f(x), fs(11) ], x, -10, 10 ) )$
\end{myVerbatim}
\begin{figure} [h!]
   \centerline{\includegraphics[scale=0.8]{ch7p17.eps} }
	\caption{$\mathbf{nmax = 11}$ Approx. to Two Element Pulse }
\end{figure}   
% eps code for two element pulse plot
% qdraw(xr(-15, 15), yr(-10,10), line(-5,0,5,0,lw(3)),line(0,-10,0,-5,lw(3) ),
%       line(0,5,0,10,lw(3) ), lw(10),
%       ex(  [ f(x), fs(11) ], x, -10, 10 ),
%       pic(eps,"ch7p17",font("Times-Bold",18) ) )$

\noindent Whether you are working with a periodic function or not, the Fourier
  series expansion always represents a periodic function for all \textbf{x}.
For this example, the period is $\mathbf{20}$, and if we make a plot of \textbf{fs(11)}
  over the range \mv|[-10, 30]|, we are including two periods.
\begin{myVerbatim}
(%i14) qdraw(yr(-10, 10),ex( fs(11), x, -10 , 30 ) )$
\end{myVerbatim}
which looks like
\smallskip
\begin{figure} [h]
   \centerline{\includegraphics[scale=0.8]{ch7p18.eps} }
	\caption{$\mathbf{nmax = 11}$ Approx. Drawn for Two Periods }
\end{figure}   
% eps code for two element pulse plot
% qdraw( yr(-10,10), line(-5,0,5,0,lw(3)), line(15,0,25,0,lw(3)),
%      line(0,-10,0,-5,lw(3) ),line(0,5,0,10,lw(3) ), lw(10),
%       ex( fs(11) , x, -10, 30 ),
%       pic(eps,"ch7p18",font("Times-Bold",18) ) )$

\newpage
\subsubsection{Exponential Form of a Fourier Series Expansion} \label{expfs}
The exponential form of a Fourier series can be based on the completeness and
  orthogonality (in the Hermitian sense) of the set of exponential functions
\begin{equation}
\left\{ \mathbf{ exp\left(\frac{2\,\boldsymbol{\pi}\,i\,n\,x}{b-a} \right) } \right\} \qquad 
\mathbf{(n=0,\, \pm 1,\, \pm 2,\, \cdots) }
\end{equation}
on the interval $\mathbf{(a,\,b)}$.
Define the function $\mathbf{\boldsymbol{ \phi}_{n}(x) }$ as
\begin{equation} \label{Eq:phi1}
\mathbf{\boldsymbol{\phi}_{n}(x) =  exp \left(\frac{- 2\,\boldsymbol{\pi}\,i\,n\,x}{b-a} \right) }.
\end{equation}
This function has the properties
\begin{equation} \label{Eq:phi2}
\mathbf{\boldsymbol{ \phi}_{-n}(x) = \boldsymbol{\phi}_{n}(x)^{\ast} ,\qquad 
\boldsymbol{\phi}_{n}(x) \, \boldsymbol{\phi}_{n}(x)^{\ast}  = 1}.
\end{equation}
in which the asterisk indicates the complex conjugate.\\

\noindent The (Hermitian) orthogonality of the
 $\mathbf{\left\{ \boldsymbol{\phi}_{n}(x) \right\}}$ over $\mathbf{(a,\,b)}$ 
  is expressed by
\begin{equation} \label{Eq:phi3}
\mathbf{\int_{a}^{b} \boldsymbol{\phi}_{n}(x) \, \boldsymbol{\phi}_{m}(x)^{\ast}  \, dx =
   (b-a)\,\boldsymbol{\delta}_{n\,m} },
\end{equation}
in which the Kronecker delta symbol is
\begin{equation} \label{Eq:kron}
\mathbf{\boldsymbol{\delta}_{n\,m}} = \left\{ \begin{array}{rcc}
                \mathbf{1} & \text{for} & \mathbf{n = m }\\
				\mathbf{0} & \text{for} & \mathbf{n \neq m}. 
				\end{array} \right\}
\end{equation}
Eq.(\ref{Eq:phi3}) is clearly true for $\mathbf{n = m}$, using Eq.(\ref{Eq:phi2}).
For $\mathbf{m \neq n}$, we can simplify this integral
  by using the exponential product law $\mathbf{e^{A}\,e^{B} = e^{A+B}}$,
  letting $\mathbf{r= m-n \neq 0}$, and changing the variable of integration
  $\mathbf{x \rightarrow y}$ via $\mathbf{x = (b-a)\,y + a}$.
The resulting integral in terms of $\mathbf{y}$ will then be over the interval (0,\,1).
The differential $\mathbf{dx \, \rightarrow (b-a)\,dy}$ and $\mathbf{(b-a)}$ comes outside the integral.
Also, inside the exponential, $\mathbf{x/(b-a) \rightarrow a/(b-a) + y}$, and
  we can take outside an exponential function of a constant. Thus the integral is
  proportional to $\mathbf{\int_{0}^{1} exp(2\,\boldsymbol{\pi}\,i\,r\,y)\,dy}$,
  which is proportional to $\mathbf{exp(2\,\boldsymbol{\pi}\,i\,r) - 1 = exp(2\,\boldsymbol{\pi}\,i)^{r} -1 = 0}$.\\
 
  
\noindent We now write some function of $\mathbf{x}$ as a linear combination 
  of the $\boldsymbol{\phi_{n}(x)}$ with coefficients $\mathbf{C_{n}}$ to be
  determined.
\begin{equation} \label{Eq:fs1}
\mathbf{f(x) = \sum_{n = -\infty}^{\infty} C_{n} \,} \boldsymbol{\phi}_{\mathbf{n}}\mathbf{(x)}.
\end{equation}
To find the $\{\mathbf{C_{n}} \}$, we multiply both sides of Eq.(\ref{Eq:fs1}) by
  $\mathbf{\boldsymbol{\phi}_{m}(x)^{\ast}  \, dx}$, integrate
  both sides over the interval $\mathbf{(a,\,b)}$, and use orthogonality, Eq.(\ref{Eq:phi3}).
\begin{align*}
  \int_{a}^{b} \mathbf{f(x)} \,\boldsymbol{\phi}_{\mathbf{m}}(\mathbf{x})^{\ast}  \, \mathbf{dx} &=
\mathbf{\sum_{n = -\infty}^{\infty} C_{n} } \, \int_{a}^{b} \boldsymbol{\phi}_{\mathbf{n}}\mathbf{(x)}
 \, \boldsymbol{\phi}_{\mathbf{m}} \mathbf{(x)}^{\ast}  \, \mathbf{dx} \\
  &= \mathbf{\sum_{n = -\infty}^{\infty} C_{n}\,(b-a)} \, \boldsymbol{\delta}_{\mathbf{n\,m}} \\
  &= \mathbf{(b - a)\,C_{m}}.
\end{align*}
Hence we have for the coefficients
\begin{equation} \label{Eq:fs2}
\mathbf{C_{n} = \frac{1}{(b-a)}\,
\int_{a}^{b} f(x)} \,\boldsymbol{\phi}_{\mathbf{n}}(\mathbf{x})^{\ast}  \, \mathbf{dx}.
\end{equation}
Inserting these coefficients into Eq.(\ref{Eq:fs1}) we can write
\begin{align*}
\mathbf{f(x)} &= \mathbf{ \sum_{n = -\infty}^{\infty} C_{n} \,} \boldsymbol{\phi}_{\mathbf{n}}\mathbf{(x)} \\
             &= \mathbf{ \sum_{-\infty}^{\infty} exp \left(\frac{- 2\,\boldsymbol{\pi}\,i\,n\,x}{b-a} \right)} \,
\mathbf{ \frac{1}{b - a}\,\int_{a}^{b} f(y)\,exp \left(\frac{ 2\,\boldsymbol{\pi}\,i\,n\,y}{b-a} \right)\,dy } \\
			 &= \mathbf{\frac{1}{b-a}\,	\sum_{-\infty}^{\infty} 
			      \int_{a}^{b} f(y)\,exp \left(\frac{ 2\,\boldsymbol{\pi}\,i\,n\,(y-x)}{b-a} \right)\,dy. }
\end{align*}
We now separate out the $\mathbf{n=0}$ term and combine the $\mathbf{n = \pm m}$ terms into
 a sum over the positive integers. 
\begin{align*}
\mathbf{f(x)} &= \mathbf{ \frac{1}{b-a} \, \int_{a}^{b} f(y)\,dy} +
\mathbf{\frac{1}{b-a}  \sum_{n=1}^{\infty} \int_{a}^{b} f(y)\,
\left[ exp \left(\frac{ 2\,\boldsymbol{\pi}\,i\,n\,(y-x)}{b-a} \right) +
        exp \left(\frac{ -2\,\boldsymbol{\pi}\,i\,n\,(y-x)}{b-a} \right) \right] \, dy } \\
 &= \mathbf{ \frac{1}{b-a} \, \int_{a}^{b} f(y)\,dy} + 
	\mathbf{ \frac{2}{b-a} \sum_{n=1}^{\infty}  \int_{a}^{b} f(y)} \,\boldsymbol{\cos }
	 \mathbf{ \left( \frac{ 2\,\boldsymbol{\pi}\,n\,(y-x)}{b-a} \right) \, dy }
\end{align*}
Using the trig identity $\mathbf{\boldsymbol{\cos}(A-B)= \boldsymbol{\cos} A \, \boldsymbol{\cos} B +
 \boldsymbol{\sin} A \, \boldsymbol{\sin} B}$, we recover the trigonometric form
 of the Fourier series expansion of a function over the interval (a,\,b), which will
 represent a function which has period equal to (b\,-\,a) for all values of $\mathbf{x}$.
\begin{equation} \label{Eq:ab1}
\mathbf{f(x) = \frac{1}{2} \, a_{0} + \sum_{n=1}^{\infty} \left[ 
   a_{n}\,\boldsymbol{\cos} \left( \frac{ 2\,\boldsymbol{\pi}\,n\,x}{b-a} \right) +
   b_{n} \,\boldsymbol{\sin} \left( \frac{ 2\,\boldsymbol{\pi}\,n\,x}{b-a} \right) \right] }
\end{equation}
The coefficients are given by the integrals
\begin{equation} \label{Eq:ab2}
\mathbf{ a_{n} = \frac{2}{b-a} \, \int_{a}^{b} f(y)\,
   \boldsymbol{\cos} \left( \frac{ 2\,\boldsymbol{\pi}\,n\,y}{b-a} \right) \, dy}, \quad
   \mathbf{ b_{n} = \frac{2}{b-a} \, \int_{a}^{b} f(y)\,
   \boldsymbol{\sin} \left( \frac{ 2\,\boldsymbol{\pi}\,n\,y}{b-a} \right) \, dy}. 
\end{equation}
The expansion starts with the term
\begin{equation}
\mathbf{f(x) = \frac{1}{2} \, a_{0} + \cdots  = \frac{1}{b-a} \, \int_{a}^{b} f(y) + \cdots }
\end{equation}
which is the (unweighted) average of $\mathbf{f(x)}$ over (a,\,b).\\

\noindent If we specialize to a function defined over the interval \textbf{(-p,\,p)}, whose Fourier
  expansion will represent a function which has period \textbf{2\,p} for all \textbf{x}, 
  the above results have the replacements $\mathbf{2/(b-a) \rightarrow 2/(2\,p) \rightarrow 1/p}$, and
the appropriate exspansion equations are
\begin{equation}  \label{Eq:pex1}
\mathbf{f(x) = \frac{1}{2} \, a_{0} + \sum_{n=1}^{\infty} \left[ 
   a_{n}\,\boldsymbol{\cos} \left( \frac{ \boldsymbol{\pi}\,n\,x}{p} \right) +
   b_{n} \,\boldsymbol{\sin} \left( \frac{ \boldsymbol{\pi}\,n\,x}{p} \right) \right] }
\end{equation}
with coefficients
\begin{equation} \label{Eq:pex2}
\mathbf{ a_{n} = \frac{1}{p} \, \int_{-p}^{p} f(y)\,
   \boldsymbol{\cos} \left( \frac{ \boldsymbol{\pi}\,n\,y}{p} \right) \, dy}, \quad
   \mathbf{ b_{n} = \frac{1}{p} \, \int_{-p}^{p} f(y)\,
   \boldsymbol{\sin} \left( \frac{ \boldsymbol{\pi}\,n\,y}{p} \right) \, dy}. 
\end{equation}
If we specialize further to $\mathbf{p = \boldsymbol{\pi}}$, the appropriate equations
  are
\begin{equation}  \label{Eq:piex1}
\mathbf{f(x) = \frac{1}{2} \, a_{0} + \sum_{n=1}^{\infty} \left[ 
   a_{n}\,\boldsymbol{\cos} (n\,x) + b_{n} \,\boldsymbol{\sin} (n\,x) \right] }
\end{equation}
with coefficients
\begin{equation} \label{Eq:piex2}
\mathbf{ a_{n} = \frac{1}{\boldsymbol{\pi}} \, \int_{-\boldsymbol{\pi}}^{\boldsymbol{\pi}} f(y)\,
   \boldsymbol{\cos} (n\,y) \, dy}, \quad
   \mathbf{ b_{n} = \frac{1}{\boldsymbol{\pi}} \, \int_{-\boldsymbol{\pi}}^{\boldsymbol{\pi}} f(y)\,
   \boldsymbol{\sin} (n\,y) \, dy}. 
\end{equation}
\subsection{Fourier Integral Transform Pairs}
\subsubsection{Fourier Cosine Integrals and \textbf{fourintcos(..)}}
Given some function $\mathbf{f(x)}$ defined for $\mathbf{x \geq 0}$,
  we define the Fourier cosine transform  of this function as
\begin{equation}  \label{Eq:fc1}
\mathbf{F_{C}(f,\boldsymbol{\omega}) = \frac{2}{\boldsymbol{\pi}} 
  \int_{0}^{\infty} \boldsymbol{\cos}(\boldsymbol{\omega}\,x)\,f(x)\,dx }
\end{equation}
The given function $\mathbf{f(x)}$ can then be written as an integral
  over positive values of $\boldsymbol{\omega}$:
\begin{equation}  \label{Eq:fc2}
\mathbf{f(x) = \int_{0}^{\infty} F_{C}(f,\boldsymbol{\omega})\,
    \cos(\boldsymbol{\omega}\,x)\,d\boldsymbol{\omega} }
\end{equation}
The two equations (\ref{Eq:fc1}) and (\ref{Eq:fc2}) are an example of
  a "Fourier transform pair", which include conventions about where
  to place the factor of $2/\boldsymbol{\pi}$.\\

\noindent Here is a simple example. 
We let the function be $\mathbf{f(x) = \boldsymbol{\sin}(x)\, e^{-x} }$, defined for 
 $\mathbf{x \geq 0}$.
 Let the letter \mv|w| be used to stand for $\boldsymbol{\omega}$.
We first calculate the Fourier cosine transform
 $\mathbf{F_{C}(f,\boldsymbol{\omega})}$   (we use the symbol \mv|fcw| 
  in our work here) using \textbf{integrate} and then show that the \tcbr{inverse}
  integral over $\boldsymbol{\omega}$ gives back the original function.
\begin{myVerbatim}
(%i1) f:sin(x)*exp(-x)$
(%i2) integrate(f*cos(w*x),x,0,inf);
                                             2
                                 2        2 w
(%o2)                          ------ - --------
                                4          4
                               w  + 4   2 w  + 8
(%i3) fcw : (2/%pi)*ratsimp(%);
                                       2
                                   2 (w  - 2)
(%o3)                           - ------------
                                        4
                                  %pi (w  + 4)
(%i4) integrate(fcw*cos(w*x),w,0,inf);
Is  x  positive, negative, or zero?
p;
                                   - x
(%o4)                            %e    sin(x)
\end{myVerbatim}
\newpage
\noindent We can also use the package \textbf{fourie.mac}, which we explored in the section
  on Fourier series.
This package provides for the calculation of our Fourier cosine transform integral
  via the \textbf{fourintcos(expr,var)} function.
The function \textbf{fourintcos(f,x)} returns an answer with the label $\mathbf{a_{z}}$
  which contains the Fourier cosine integral $\mathbf{F_{C}(f,\,z)}$, with the
  letter \mv|z| being the package convention for what we called \mv|w| ($\boldsymbol{\omega}$),
\begin{myVerbatim}
(%i5) load(fourie);
(%o5) C:/PROGRA~1/MAXIMA~3.1/share/maxima/5.18.1/share/calculus/fourie.mac
(%i6) fourintcos(f,x);
                                               2
                                     2        z
                                2 (------ - ------)
                                    4        4
                                   z  + 4   z  + 4
(%t6)                      a  = -------------------
                            z           %pi

(%o6)                                [%t6]
(%i7) az : ratsimp(rhs(%t6));
                                       2
                                    2 z  - 4
(%o7)                          - --------------
                                      4
                                 %pi z  + 4 %pi
(%i8) (2/%pi)*ratsimp(%pi*az/2);
                                       2
                                   2 (z  - 2)
(%o8)                           - ------------
                                        4
                                  %pi (z  + 4)
\end{myVerbatim}
Thus \textbf{fourintcos(expr,var)} agrees with our definition of the Fourier
  cosine transform.    
\subsubsection{Fourier Sine Integrals and \textbf{fourintsin(..)}}
Given some function $\mathbf{f(x)}$ defined for $\mathbf{x \geq 0}$,
  we define the Fourier sine transform  of this function as
\begin{equation}  \label{Eq:fsine1}
\mathbf{F_{S}(f,\boldsymbol{\omega}) = \frac{2}{\boldsymbol{\pi}} 
  \int_{0}^{\infty} \boldsymbol{\sin}(\boldsymbol{\omega}\,x)\,f(x)\,dx }
\end{equation}
The given function $\mathbf{f(x)}$ can then be written as an integral
  over positive values of $\boldsymbol{\omega}$:
\begin{equation}  \label{Eq:fsine2}
\mathbf{f(x) = \int_{0}^{\infty} F_{S}(f,\boldsymbol{\omega})\,
    \sin(\boldsymbol{\omega}\,x)\,d\boldsymbol{\omega} }
\end{equation}
The two equations (\ref{Eq:fsine1}) and (\ref{Eq:fsine2}) are another 
   "Fourier transform pair", which include conventions about where
  to place the factor of $2/\boldsymbol{\pi}$.  
\newpage
\noindent Here is a simple example. 
We let the function be $\mathbf{f(x) = \boldsymbol{\cos}(x)\, e^{-x} }$, defined for 
 $\mathbf{x \geq 0}$.
Let the letter \mv|w| stand for $\boldsymbol{\omega}$.
We first calculate the Fourier sine transform 
  $\mathbf{F_{S}(f,\boldsymbol{\omega})}$ (we use the symbol \mv|fsw| 
  in our work here) using \textbf{integrate} and then show that the
  \tcbr{inverse} integral over $\boldsymbol{\omega}$ gives back the original function.  
\begin{myVerbatim}
(%i1) f:cos(x)*exp(-x)$
(%i2) assume(w>0)$
(%i3) integrate(f*sin(w*x),x,0,inf);
                                       3
                                      w
(%o3)                               ------
                                     4
                                    w  + 4
(%i4) fsw : (2/%pi)*%;
                                        3
                                     2 w
(%o4)                            ------------
                                       4
                                 %pi (w  + 4)
(%i5) integrate(fsw*sin(w*x),w,0,inf);
Is  x  positive, negative, or zero?

p;
                                   - x
(%o5)                            %e    cos(x)
\end{myVerbatim}
We can also use the package \textbf{fourie.mac}, which we used above.
This package provides for the calculation of our Fourier sine transform integral
  via the \textbf{fourintsin(expr,var)} function.
The function \textbf{fourintsin(f,x)} returns an answer with the label $\mathbf{b_{z}}$
  which contains the Fourier sine integral $\mathbf{F_{S}(f,\,z)}$, with the
  letter \mv|z| being the package convention for what we called \mv|w| ($\boldsymbol{\omega}$).
\begin{myVerbatim1}
(%i6) load(fourie);
(%o6) C:/PROGRA~1/MAXIMA~3.1/share/maxima/5.18.1/share/calculus/fourie.mac
(%i7) facts();
(%o7)                               [w > 0]
(%i8) (forget(w>0),facts());
(%o8)                                 []
(%i9) fourintsin(f,x);
                                           3
                                        2 z
(%t9)                          b  = ------------
                                z         4
                                    %pi (z  + 4)

(%o9)                                [%t9]
(%i10) bz : rhs(%t9);
                                        3
                                     2 z
(%o10)                           ------------
                                       4
                                 %pi (z  + 4)
\end{myVerbatim1}
Thus \textbf{fourintsin(expr,var)} agrees with our definition of the Fourier
  sine transform.
\newpage
\subsubsection{Exponential Fourier Integrals and \textbf{fourint}}  
Given some function $\mathbf{f(x)}$ defined for $\mathbf{-\infty <x < \infty}$,
  we define the exponential Fourier transform  of this function as
\begin{equation}  \label{Eq:fexp1}
\mathbf{F_{Exp}(f,\boldsymbol{\omega}) = \frac{1}{2\, \boldsymbol{\pi}} 
  \int_{-\infty}^{\infty} \,f(x) e^{i\,\boldsymbol{\omega}\,x}\,dx }
\end{equation}
The given function $\mathbf{f(x)}$ can then be written as an integral
  over both positive and negative values of $\boldsymbol{\omega}$:
\begin{equation}  \label{Eq:fexp2}
\mathbf{f(x) = \int_{-\infty}^{\infty} F_{Exp}(f,\boldsymbol{\omega})\,
    e^{-i\,\boldsymbol{\omega}\,x} \,d\boldsymbol{\omega} }
\end{equation}
The two equations (\ref{Eq:fexp1}) and (\ref{Eq:fexp2}) are another 
   "Fourier transform pair", which include conventions about where
  to place the factor of $\mathbf{2\, \boldsymbol{\pi}}$ as well as which member
  has the minus sign in the exponent.\\  

\noindent If the given function is even, $\mathbf{f(-x) = f(x) }$, then the
exponential Fourier transform has the symmetry 
\begin{equation}
\mathbf{F_{Exp}(f,\boldsymbol{-\omega}) = F_{Exp}(f,\boldsymbol{\omega}) }
\end{equation}
and can be expressed in terms of the Fourier cosine transform:
\begin{equation}
\mathbf{F_{Exp}(f,\boldsymbol{\omega}) = \frac{1}{2} \, F_{C}(f,\boldsymbol{\omega})}.
\end{equation}
If the given function is odd, $\mathbf{f(-x) = -f(x) }$, then the
exponential Fourier