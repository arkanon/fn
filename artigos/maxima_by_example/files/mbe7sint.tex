% e. woollett
%   july 08 - april 09
% file sint.tex this is new ch. 7
% make ch. 8 just the quadpack romberg stuff nint.tex
% then make ch 9 fourier integrals and laplace transforms
% and ch. 10 bigfloats and arb prec integration
% and ch. 11 fast fourier transforms
% check with new version of maxima
% 1. bfloat erf(z) is changevar example 4
% what about using fpprintprec:8 for numerical outputs here?
% things to add: 
% mailing list hard integral using expansion of exponential
% in terms of bessel functions
% function to interchange order of integration and summation
% r dodier's ''(integrate...) construct for plots
% edit with Notepad++, then load into LED for latexing
\documentclass[12pt]{article}
\usepackage[dvips,top=1.5cm,left=1.5cm,right=1.5cm,foot=1cm,bottom=1.5cm]{geometry}
\usepackage{times,amsmath,amsbsy,graphicx,fancyvrb,url}
\usepackage[usenames]{color}
%\definecolor{MyDarkBlue}{rgb}{0,0.08,0.45}
\definecolor{mdb}{rgb}{0.1,0,0.55}
\newcommand{\tcdb}{\textcolor{mdb}}
\newcommand{\tcbr}{\textcolor{BrickRed}}
\newcommand{\tcb}{\textcolor{blue}}
\newcommand{\tcr}{\textcolor{red}}
\urldef\tedhome\url{ http://www.csulb.edu/~woollett/  }
\urldef\tedmail\url{ woollett@charter.net}
%1.  this is for maxima code: red framed bold, footnotesize 
\DefineVerbatimEnvironment%
   {myVerbatim}%
   {Verbatim}%
   {fontfamily=courier,fontseries=b,fontsize=\footnotesize ,frame=single,rulecolor=\color{BrickRed}}
\DefineVerbatimEnvironment%
   {myVerbatim1}%
   {Verbatim}%
   {fontfamily=courier,fontseries=b,fontsize=\scriptsize ,frame=single,rulecolor=\color{BrickRed}}
%2.  this is for blue framed bold 
\DefineVerbatimEnvironment%
   {myVerbatim2}%
   {Verbatim}%
   {fontfamily=courier,fontseries=b,frame=single,rulecolor=\color{blue}}
\DefineVerbatimEnvironment%
   {myVerbatim2s}%
   {Verbatim}%
   {fontfamily=courier,fontseries=b,fontsize=\small,frame=single,rulecolor=\color{blue}}
\DefineVerbatimEnvironment%
   {myVerbatim2f}%
   {Verbatim}%
   {fontfamily=courier,fontseries=b,fontsize=\footnotesize,frame=single,rulecolor=\color{blue}}
% 3.  this is for black framed  bold
\DefineVerbatimEnvironment%
   {myVerbatim3}%
   {Verbatim}%
   {fontfamily= courier, fontseries=b, frame=single}
% 4.  this is for no frame bold
\DefineVerbatimEnvironment%
   {myVerbatim4}%
   {Verbatim}%
   {fontfamily=courier, fontseries=b}
% 6.  for defaults use usual verbatim
\newcommand{\mv}{\Verb[fontfamily=courier,fontseries=b]}
\newcommand{\mvs}{\Verb[fontfamily=courier,fontseries=b,fontsize=\small]}
\newcommand{\mvf}{\Verb[fontfamily=courier,fontseries=b,fontsize=\footnotesize]}

\renewcommand{\thefootnote}{\ensuremath{\fnsymbol{footnote}}}
%%%%%%%%%%%%%%%%%%%%%%%%%%%%%%%%%%%%%%%%%%%%%%%%%%%%%%%%%%%%%%%%%%%%%%%%
%   title page
%%%%%%%%%%%%%%%%%%%%%%%%%%%%%%%%%%%%%%%%%%%%%%%%%%%%%%%%%%%%%%%%%%%%%%
%% \title{ \color{mdb}\  Maxima by Examp
\title{  Maxima by Example:\\ Ch.7: Symbolic Integration 
            \thanks{This version uses \textbf{Maxima 5.18.1}. This is a live
            document. Check \; \textbf{ \tedhome } \; for the latest version of these notes. Send comments and
			 suggestions to \textbf{\tedmail} } }

%%  \author{\color{BrickRed}\  Edwin
\author{ Edwin L. Woollett}
\date{\today}
%%%%%%%%%%%%%%%%%%%%%%%%%%%%%%%%%%%%%%%%%
%          document
%%%%%%%%%%%%%%%%%%%%%%%%%%%%%%%%%%%%%%%%%%
\begin{document}
%\footnotesize
%\small
\maketitle
\tableofcontents
\numberwithin{equation}{section}
\newpage
\begin{myVerbatim2} 
COPYING AND DISTRIBUTION POLICY    
This document is part of a series of notes titled
"Maxima by Example" and is made available
via the author's webpage http://www.csulb.edu/~woollett/
to aid new users of the Maxima computer algebra system.	
	
NON-PROFIT PRINTING AND DISTRIBUTION IS PERMITTED.
	
You may make copies of this document and distribute them
to others as long as you charge no more than the costs of printing.	

These notes (with some modifications) will be published in book form
eventually via Lulu.com in an arrangement which will continue
to allow unlimited free download of the pdf files as well as the option
of ordering a low cost paperbound version of these notes.
\end{myVerbatim2}	
\smallskip
\noindent \tcbr{Feedback from readers is the best way for this series of notes
  to become more helpful to new users of Maxima}.
\tcdb{\emph{All} comments and suggestions for improvements will be appreciated and
  carefully considered}.
\smallskip
\begin{myVerbatim2s}
The Maxima session transcripts were generated using the XMaxima 
graphics interface on a Windows XP computer, and copied into 
a fancy verbatim environment in a latex file which uses the
fancyvrb and color packages.

We use qdraw.mac (available on the author's web page with Ch. 5 materials)
for plots.
\end{myVerbatim2s}  
\smallskip
\begin{myVerbatim}
Maxima.sourceforge.net. Maxima, a Computer Algebra System. Version 5.18.1
 (2009). http://maxima.sourceforge.net/
\end{myVerbatim}
\newpage
\setcounter{section}{7}
\subsection{Symbolic Integration with \textbf{integrate}}
Although the process of finding an integral can be viewed as
  the inverse of the process of finding a derivative, in practice
  finding an integral is more difficult.
It turns out that the integral of fairly simple looking functions cannot
  be expressed in terms of well known functions, and this has been
  one motivation for the introduction of definitions for a variety of "special functions",
  as well as efficient methods for numerical integration (quadrature) discussed in Ch. 8.\\    
  
\noindent In the following, we always assume we are dealing with a real function of a real variable
  and that the function is single-valued and continuous over the intervals of interest.\\

\noindent The Maxima manual entry for \textbf{integrate} includes
  (we have made some changes and additions):
\small
\begin{quote}
Function: \textbf{integrate(expr, var)} \\
Function: \textbf{integrate(expr, var, a, b)}\\ 
Attempts to symbolically compute the integral of \textbf{expr} with respect to \textbf{var}.
   \textbf{integrate(expr, var)} is an indefinite integral, 
   while \textbf{integrate(expr, var, a, b)} is a definite integral,
   with limits of integration \textbf{a} and \textbf{b}.
The integral  is returned if \textbf{integrate} succeeds.
Otherwise the return value is the noun form of the integral (the quoted operator \textbf{'integrate})
   or an expression containing one or more noun forms.
The noun form of \textbf{integrate} is displayed with an integral sign if \textbf{display2d} is
  set to \textbf{true} (which is the default).\\

In some circumstances it is useful to construct a noun form by hand, by quoting \textbf{integrate}
   with a single quote, e.g., \textbf{'integrate(expr, var)}.
For example, the integral may depend on some parameters which are not yet computed.
The noun may be applied to its arguments by \textbf{ev(iexp, nouns)} where \textbf{iexp}
 is the noun form of interest. \\
 

The Maxima function \textbf{integrate} is defined by the lisp 
  function \textbf{\$integrate} in the file /src/simp.lisp.
The indefinite integral invocation, \textbf{integrate(expr,var)}, results in a call to the lisp
  function \textbf{sinint}, defined in src/sin.lisp, unless the flag \textbf{risch} is present,
  in which case the lisp function \textbf{rischint}, defined in src/risch.lisp, is called.
The definite integral invocation, \textbf{integrate(expr,var,a,b)},
 causes a call to the lisp function \textbf{\$defint}, defined in src/defint.lisp.
 The lisp function \textbf{\$defint} is available as the Maxima function \textbf{defint} and
  can be used to bypass \textbf{integrate} for a definite integral.\\  

To integrate a Maxima function $\mathbf{f(x)}$, insert $\mathbf{f(x)}$ in
 the \textbf{expr} slot.\\

\textbf{integrate} does not respect implicit dependencies established by the \textbf{depends} function. \\

\textbf{integrate} may need to know some property of the parameters in the integrand.
\textbf{integrate} will first consult the \textbf{assume} database, and, if the
   variable of interest is not there, \textbf{integrate} will ask the user.
Depending on the question, suitable responses are \textbf{yes;} or \textbf{no;},
  or \textbf{pos;}, \textbf{zero;}, or \textbf{neg;}.
Thus, the user can use the \textbf{assume} function to avoid all or some questions.  \\
\end{quote}
\normalsize
\newpage

\subsection{Integration Examples and also \textbf{defint}, \textbf{ldefint}, \textbf{beta},
                 \textbf{gamma}, \textbf{erf}, and \textbf{logabs}}  \label{symbolic}

\subsubsection*{Example 1}
Our first example is the \tcbr{indefinite} integral $\mathbf{\int \boldsymbol{\sin}^{3}x \, dx}$:
\begin{myVerbatim}
(%i1) integrate (sin(x)^3, x);
                                  3
                               cos (x)
(%o1)                          ------- - cos(x)
                                  3
(%i2) ( diff (%,x), trigsimp(%%) );
                                       3
(%o2)                               sin (x)
\end{myVerbatim} 
Notice that the \tcbr{indefinite} integral returned by \textbf{integrate}
  \emph{does not include} the arbitrary
  constant of integration which can always be added.\\
  
\noindent If the returned integral is correct (up to an arbitrary constant), then the first
  derivative of the returned indefinite integral should be the original integrand,
  although we may have to simplify the result manually (as we had to do above).
\subsubsection*{Example 2}
Our second example is another indefinite integral,
    $\mathbf{\int x \, (b^{2}-x^{2})^{-1/2} \, dx}$:
\begin{myVerbatim}
(%i3) integrate (x/ sqrt (b^2 - x^2), x);
                                        2    2
(%o3)                           - sqrt(b  - x )
(%i4) diff(%,x);
                                       x
(%o4)                            -------------
                                       2    2
                                 sqrt(b  - x )
\end{myVerbatim}
\subsubsection*{Example 3}
The \tcbr{definite} integral can be related to the "area under a curve" and is the more
  accessible concept, while the \tcbr{} integral is simply a function whose first derivative
  is the original integrand.\\

\noindent Here is a \tcbr{definite} integral,
     $\mathbf{\int_{0}^{\pi} \boldsymbol{\cos}^2 x \, e^{x} \,dx}$:
\begin{myVerbatim}
(%i5) i3 : integrate (cos(x)^2 * exp(x), x, 0, %pi);

                                      %pi
                                  3 %e      3
(%o5)                            ------- - -
                                     5      5
\end{myVerbatim} 
Instead of using \textbf{integrate} for a definite integral, you 
  can try \textbf{ldefint} (think Limit definite integral), which may
  provide an alternative form of the answer (if successful).
\newpage
From the Maxima manual:
\small
\begin{quote}
Function: \textbf{ldefint(expr, x, a, b)}\\
Attempts to compute the definite integral of \textbf{expr} by using \textbf{limit} to evaluate
  the indefinite integral of \textbf{expr} with respect to $\mathbf{x}$ at the
  upper limit $\mathbf{b}$ and at the lower limit $\mathbf{a}$.
If it fails to compute the definite integral, \textbf{ldefint} returns
  an expression containing limits as noun forms.\\
  
\textbf{ldefint} is not called from \textbf{integrate}, so executing
  \textbf{ldefint(expr, x, a, b)} may yield a different result
   than \textbf{integrate(expr, x, a, b)}.
\textbf{ldefint} always uses the same method to evaluate the
   definite integral, while \textbf{integrate} may employ various
   heuristics and may recognize some special cases.
\end{quote}
\normalsize
\noindent Here is an example of use of \textbf{ldefint}, as well as the direct use
  of \textbf{defint} (which bypasses \textbf{integrate} ):
\begin{myVerbatim}
(%i6) ldefint (cos(x)^2 * exp(x), x, 0, %pi);
                                      %pi
                                  3 %e      3
(%o6)                             ------- - -
                                     5      5
(%i7) defint (cos(x)^2 * exp(x), x, 0, %pi);
                                      %pi
                                  3 %e      3
(%o7)                             ------- - -
                                     5      5
\end{myVerbatim} 
\subsubsection*{Example 4}
Here is an example of a definite integral over an infinite range,
  $\mathbf{\int_{-\infty}^{\infty} x^{2}\, e^{-x^{2}} \, dx}$:  
\begin{myVerbatim}
(%i8) integrate (x^2 * exp(-x^2), x, minf, inf);
                                   sqrt(%pi)
(%o8)                              ---------
                                       2									   
\end{myVerbatim} 
To check this integral, we first ask for the indefinite integral and 
  then check it by differentiation.
\begin{myVerbatim}
(%i9) i1 : integrate(x^2*exp(-x^2),x );
                                                    2
                                                 - x
                          sqrt(%pi) erf(x)   x %e
(%o9)                    ---------------- - --------
                                 4              2
(%i10) diff(i1,x);
                                           2
                                    2   - x
(%o10)                             x  %e
\end{myVerbatim} 
Thus the indefinite integral is correct.
The second term, heavily damped by the factor $\mathbf{e^{-x^2}}$ at $\mathbf{\pm \, \infty}$, does
  not contribute to the definite integral.
The first term is proportional to the (Gauss) error function, $\mathbf{erf(x)}$, in which 
  $\mathbf{x}$ is real.
For (in general) complex $\mathbf{w = u + i\,v}$,
\begin{equation}
\mathbf{Erf(w) = \frac{2}{\sqrt{\boldsymbol{\pi}}} \, \int_{0}^{w} e^{-z^2}\, dz}
\end{equation}
in which we can integrate over any path connecting $\mathbf{0}$ and $\mathbf{w}$ in
  the complex $\mathbf{z = x + i\,y}$ plane, since the integrand is an entire function
  of $\mathbf{z}$ (no singularities in the finite $\mathbf{z}$ plane.
\newpage
\noindent Let's make a plot of \mv|erf(x)|:
\begin{myVerbatim}
(%i11) ( load(draw),load(qdraw) )$
(%i12) qdraw(yr(-2,2),ex1(erf(x),x,-5,5,lw(5),lc(red),lk("erf(x)") ) )$	
\end{myVerbatim}
with the result:
%\smallskip
\begin{figure} [h]
   \centerline{\includegraphics[scale=.6]{ch7p1.eps} }
	\caption{ $\mathbf{erf(x)}$  }
\end{figure} 
%% eps code for ch7p1 plot of erf(x)
%  (%i30) qdraw(yr(-2,2), line(-5,0,5,0,lw(3) ),
%          line(0,-2,0,2,lw(3) ),
%          ex1(erf(x),x,-5,5,lw(5),lc(red),lk("erf(x)") ),
%         pic(eps,"ch7p1",font("Times-Bold",20) ) );
%   (%o30)                 [gr2d(points, points, explicit)]

\noindent Maxima's \textbf{limit} function confirms what the plot indicates:
\begin{myVerbatim}
(%i13) [limit(erf(x),x, inf), limit(erf(x),x, minf)];
(%o13)                             [1, - 1]
\end{myVerbatim} 
Using these limits in \mv|%o9| produces the definite integral desired.
\subsubsection*{Example 5: Use of \textbf{assume} }
We next calculate the definite integral $\mathbf{\int_{0}^{\infty} x^{a}\,(x+1)^{-5/2} \, dx}$.
\begin{myVerbatim}
(%i1) (assume(a>1),facts());
(%o1)                               [a > 1]
(%i2) integrate (x^a/(x+1)^(5/2), x, 0, inf );
   2 a + 2
Is ------- an integer?
      5

no;
Is  2 a - 3  positive, negative, or zero?

neg;
                                   3
(%o2)                         beta(- - a, a + 1)
                                   2
\end{myVerbatim} 
The combination of $\mathbf{assume ( a > 1 )}$ and $\mathbf{2\, a - 3 < 0}$ means that
  we are assuming $\mathbf{1 < a < 3/2}$. \\
These assumptions about $\mathbf{a}$ imply that 
 $\mathbf{4/5 < (2\,a+2)/5 < 1}$. 
To be consistent, we must hence answer \textbf{no} to the first question.
\newpage
\small
Let's tell Maxima to \textbf{forget} about the \textbf{assume} assignment 
  and see what the difference is.
\begin{myVerbatim1}
(%i3) ( forget(a>1), facts() );
(%o3)                                 []
(%i4) is( a>1 );
(%o4)                               unknown
(%i5) integrate (x^a/(x+1)^(5/2), x, 0, inf );
Is  a + 1  positive, negative, or zero?

pos;
Is a an integer?

no;
      7
Is ------- an integer?
   2 a + 4

no;
Is  2 a - 3  positive, negative, or zero?

neg;
                                   3
(%o5)                         beta(- - a, a + 1)
                                   2
(%i6) [is( a>1 ), facts() ];
(%o6)                            [unknown, []]
\end{myVerbatim1}
Thus we see that omitting the initial \mvs|assume(a>1)| statement 
  causes \textbf{integrate} to ask four questions instead of two.
We also see that answering questions posed by the \textbf{integrate} dialogue script
  does \tcbr{not} result in population of the \textbf{facts} list. \\  

\noindent The Maxima \textbf{beta} function has the manual entry:
\begin{myVerbatim2s}
Function: beta (a, b) 
The beta function is defined as gamma(a) gamma(b)/gamma(a+b) (A&S 6.2.1). 
\end{myVerbatim2s}
In the usual mathematics notation, the beta function can be defined in terms of the gamma function as
\begin{equation}
\mathbf{B(r,s) = \frac{\boldsymbol{\Gamma}(r)\,\boldsymbol{\Gamma}(s)}{\boldsymbol{\Gamma}(r + s)} }
\end{equation}
for all $\mathbf{r, s}$ in the complex plane.\\

\noindent The Maxima \textbf{gamma} function has the manual entry
\begin{myVerbatim2f}
Function: gamma (z) 
The basic definition of the gamma function (A&S 6.1.1) is 

                               inf
                              /
                              [     z - 1   - t
                   gamma(z) = I    t      %e    dt
                              ]
                              /
                               0
\end{myVerbatim2f}
The gamma function can be defined for complex $\mathbf{z}$ and $\mathbf{Re(z)  > 0}$ by the integral along
  the real $\mathbf{t}$ axis
\begin{equation}
\mathbf{\boldsymbol{\Gamma}(z) = \int_{0}^{\infty} t^{z-1}\,e^{-t}\,d\,t }
\end{equation}
and  for $\mathbf{Im(z) = 0}$ and $\mathbf{Re(z) = n}$ and $\mathbf{n}$ an integer greater than zero we have
\begin{equation}
\mathbf{\boldsymbol{\Gamma}(n+1) = n\,! }
\end{equation}
\newpage
\noindent How can we check the definite integral Maxima has offered?
If we ask the \textbf{integrate} function for the indefinite integral, we get 
   the "noun form", a signal of failure:
\begin{myVerbatim1}
(%i7) integrate(x^a/(x+1)^(5/2), x );
                                /      a
                                [     x
(%o7)                          I ---------- dx
                                ]        5/2
                                / (x + 1)
(%i8) grind(%)$
'integrate(x^a/(x+1)^(5/2),x)$
\end{myVerbatim1} 
Just for fun, let's include the \textbf{risch} flag and see what we get:
\begin{myVerbatim1}
(%i9) integrate(x^a/(x+1)^(5/2), x ), risch;
                        /           a log(x)
                        [         %e
(%o9)                   I -------------------------- dx
                        ]               2
                        / sqrt(x + 1) (x  + 2 x + 1)
\end{myVerbatim1} 
We again are presented with a noun form, but the integrand has been written in
  a different form, in which the identity
  $$\mathbf{x^A = e^{A \, \boldsymbol{\ln}(x)}}$$ has been used.\\  

\noindent We can at least make a spot check for a value of the parameter $\mathbf{a}$ in the
  middle of the assumed range $\mathbf{(1, 3/2)}$, namely for $\mathbf{a = 5/4}$.
\begin{myVerbatim}
(%i10) float(beta(1/4,9/4));
(%o10)                         3.090124462168955
(%i11) quad_qagi(x^(5/4)/(x+1)^(5/2),x, 0, inf);
(%o11)        [3.090124462010259, 8.6105700347616221E-10, 435, 0]
\end{myVerbatim} 
We have used the quadrature routine, \mv|quad_qagi| (see Ch. 8) for
  a numerial estimate of this integral.
The first element of the returned list is the numerical answer,
  which agrees with the analytic answer.
\normalsize
\subsubsection*{Example 6: Automatic Change of Variable and \textbf{gradef} }
Here is an example which illustrates Maxima's ability to make a change of
  variable to enable the return of an indefinite integral.
The task is to evaluate the indefinite integral
\begin{equation}
\mathbf{\int \frac{\boldsymbol{\sin}(r^2) \, dr(x)/dx}{q(r)} \, dx }
\end{equation}
by telling Maxima that the $\mathbf{\boldsymbol{\sin} (r^{2})}$ in the numerator is
 related to $\mathbf{q(r)}$ in the denominator by the derivative:
\begin{equation}
\mathbf{\frac{d\,q(u)}{d\,u} = \boldsymbol{\sin}(u^2) }.
\end{equation}
We would manually rewrite the integrand (using the chain rule) as
\begin{equation}
\mathbf{\frac{\boldsymbol{\sin}(r^2)\, dr(x)/dx}{q} = \frac{1}{q}\, (dq(r)/dr) \, (dr(x)/dx ) =
   \frac{1}{q}\, \frac{dq}{dx} = \frac{d}{dx}\, \boldsymbol{\ln}(q) }
\end{equation}
and hence obtain the indefinite integral $\mathbf{\boldsymbol{\ln}(q(r(x)))}$.
\newpage
\noindent Here we assign the derivative knowledge using \textbf{gradef} (as discussed in Ch. 6):
\begin{myVerbatim}
(%i1) gradef(q(u),sin(u^2) )$
(%i2) integrand : sin(r(x)^2)* 'diff(r(x),x ) /q( r(x) ) ;
                             d               2
                            (-- (r(x))) sin(r (x))
                             dx
(%o2)                       ----------------------
                                   q(r(x))
(%i3) integrate(integrand,x);
(%o3)                            log(q(r(x)))
(%i4) diff(%,x);
                             d               2
                            (-- (r(x))) sin(r (x))
                             dx
(%o4)                       ----------------------
                                   q(r(x))
\end{myVerbatim} 
Note that \textbf{integrate} pays no attention to \textbf{depend} assignments,
  so the briefer type of notation which \textbf{depend} allows with differentiation
  cannot be used with \textbf{integrate}:
\begin{myVerbatim}
(%i5) depends(r,x,q,r);
(%o5)                            [r(x), q(r)]
(%i6) integrand : sin(r^2)* 'diff(r,x) / q;
                                  dr      2
                                  -- sin(r )
                                  dx
(%o6)                             ----------
                                      q
(%i7) integrate(integrand,x);
                                        /
                                     2  [ dr
                                sin(r ) I -- dx
                                        ] dx
                                        /
(%o7)                           ---------------
                                       q
\end{myVerbatim} 
which is fatally flawed, since Maxima pulled both $\mathbf{\boldsymbol{\sin}(r(x)^2)}$
 and $\mathbf{1/q(r(x))}$ outside the integral.\\
  
\noindent Of course, the above \textbf{depends} assignment will still allow Maxima
  to rewrite the derivative of $\mathbf{q}$ with respect to $\mathbf{x}$ using the chain rule:
\begin{myVerbatim}
(%i8) diff(q,x);
                                     dq dr
(%o8)                                -- --
                                     dr dx
\end{myVerbatim} 
\newpage
\subsubsection*{Example 7: Integration of Rational Algebraic Functions, \textbf{rat},
    \textbf{ratsimp}, and \textbf{partfrac} }
A \it rational algebraic function \rm can be written as a quotient of two polynomials.
Consider the following function of $\mathbf{x}$.
\begin{myVerbatim}
(%i1) e1 : x^2 + 3*x -2/(3*x)+110/(3*(x-3)) + 12;
                         2          2       110
(%o1)                   x  + 3 x - --- + --------- + 12
                                   3 x   3 (x - 3)
\end{myVerbatim} 
We can obviously find the lowest common denominator and write this
  as the ratio of two polynomials, using either \textbf{rat} or \textbf{ratsimp}.
\begin{myVerbatim}
(%i2) e11 : ratsimp(e1);
                                  4      2
                                 x  + 3 x  + 2
(%o2)                            -------------
                                    2
                                   x  - 3 x
\end{myVerbatim} 
Because the polynomial in the numerator is of higher degree than the polynomial in
  the denominator, this is called an \textit{improper rational fraction}.
Any improper rational fraction can be reduced by division to a mixed form, consisting
  of a sum of some polynomial and a sum of proper fractions.
We can recover the "partial fraction" representation in terms of \textit{proper rational fractions}
  (numerator degree less than denominator degree) by using \textbf{partfrac}\verb|(expr, var)|.
\begin{myVerbatim}
(%i3) e12 : partfrac(e11,x);
                         2          2       110
(%o3)                   x  + 3 x - --- + --------- + 12
                                   3 x   3 (x - 3)
\end{myVerbatim} 
With this function of $\mathbf{x}$ expressed in partial fraction form, you are
  able to write down the indefinite integral immediately (ie., by inspection,
  without using Maxima).
But, of course, we want to practice using Maxima!
\begin{myVerbatim}
(%i4) integrate(e11,x);
                                                3      2
                 2 log(x)   110 log(x - 3)   2 x  + 9 x  + 72 x
(%o4)          - -------- + -------------- + ------------------
                    3             3                  6
(%i5) integrate(e12,x);
                                               3      2
                  2 log(x)   110 log(x - 3)   x    3 x
(%o5)           - -------- + -------------- + -- + ---- + 12 x
                     3             3          3     2
\end{myVerbatim} 
Maxima has to do less work if you have already provided the partial fraction
  form as the integrand; otherwise, Maxima internally seeks a partial fraction form in order
  to do the integral.  
\newpage
\subsubsection*{Example 8}
The next example shows that \textbf{integrate} can sometimes split the
  integrand into at least one piece which can be integrated, and leaves the
  remainder as a formal expression (using the noun form of \textbf{integrate}).
This may be possible if the denominator of the integrand is a polynomial
  which Maxima can factor.
\begin{myVerbatim}
(%i6) e2: 1/(x^4 - 4*x^3 + 2*x^2 - 7*x - 4);
                                      1
(%o6)                    --------------------------
                           4      3      2
                          x  - 4 x  + 2 x  - 7 x - 4
(%i7) integrate(e2,x);
                                     /  2
                                     [ x  + 4 x + 18
                                     I ------------- dx
                                     ]  3
                        log(x - 4)   / x  + 2 x + 1
(%o7)                  ---------- - ------------------
                            73               73
(%i8) grind(%)$
log(x-4)/73-('integrate((x^2+4*x+18)/(x^3+2*x+1),x))/73$
(%i9) factor(e2);
                                      1
(%o9)                      ----------------------
                                      3
                            (x - 4) (x  + 2 x + 1)
(%i10) partfrac(e2,x);
                                        2
                            1          x  + 4 x + 18
(%o10)                  ---------- - -----------------
                        73 (x - 4)        3
                                     73 (x  + 2 x + 1)
\end{myVerbatim} 
We have first seen what Maxima can do with this integrand, using the \textbf{grind}
  function to clarify the resulting expression, and then we have used \textbf{factor}
  and \textbf{partfrac} to see how the split-up arises.
Despite a theorem that the integral of every rational function can be expressed in terms
  of algebraic, logarithmic and inverse trigonometric expressions, Maxima
  declines to return a symbolic expression for the second, formal, piece of this
  integral (which is good because the exact symbolic answer is an extremely long
  expression).
\subsubsection*{Example 9: The \textbf{logabs} Parameter and \textbf{log} }
There is a global parameter \textbf{logabs} whose default value is \textbf{false}
  and which affects what is returned with an indefinite integral containing logs.
\begin{myVerbatim}
(%i11) logabs;
(%o11)                                false
(%i12) integrate(1/x,x);
(%o12)                               log(x)
(%i13) diff(%,x);
                                       1
(%o13)                                 -
                                       x
(%i14) logabs:true$
\end{myVerbatim}
\newpage
\begin{myVerbatim}
(%i15) integrate(1/x,x);
(%o15)                            log(abs(x))
(%i16) diff(%,x);
                                       1
(%o16)                                 -
                                       x
(%i17) log(-1);
(%o17)                             log(- 1)
(%i18) float(%);
(%o18)                       3.141592653589793 %i
\end{myVerbatim} 
When we override the default and set \textbf{logabs} to \textbf{true}, the argument
  of the \textbf{log} function is wrapped with \textbf{abs}.
According to the manual
\begin{quote}
When doing indefinite integration where logs are generated, e.g. integrate(1/x,x),
 the answer is given in terms of log(abs(...)) if logabs is true,
  but in terms of log(...) if logabs is false.
For definite integration, the logabs:true setting is used,
  because here "evaluation" of the indefinite integral at
  the endpoints is often needed. 
\end{quote}
\vspace{1cm}
\subsection{Piecewise Defined Functions and \textbf{integrate} }
We can use Maxima's \textbf{if}, \textbf{elseif}, and \textbf{else} construct
  to define piecewise defined functions.
We first define and plot a square wave of unit height which extends
  from $\mathbf{1 \leq x \leq 3}$.
\begin{myVerbatim}
(%i1) u(x) := if x >= 1 and x <= 3 then 1 else 0$
(%i2) map('u,[0.5,1,2,3,3.5]);
(%o2)                           [0, 1, 1, 1, 0]
(%i3) (load(draw),load(qdraw))$
               qdraw(...), qdensity(...), syntax: type qdraw(); 
(%i4) qdraw( yr(-1,2),ex1(u(x),x,0,4,lw(5),lc(blue)) )$
\end{myVerbatim} 
This produces 
\begin{figure} [h]
   \centerline{\includegraphics[scale=.6]{ch7p12.eps} }
	\caption{if $\mathbf{x > 1}$ and $\mathbf{x < 3}$ then $\mathbf{1}$ else $\mathbf{0}$}
\end{figure} 
%% eps code for ch7p12 plot of uu(x,1,3)
%%qdraw(yr(-1,2), line(1,0,3,0,lw(4) ),
%          ex1(u(x ),x,0,4,lw(10),lc(blue)  ),
%         pic(eps,"ch7p12",font("Times-Bold",16) ) )$
%

\newpage
\noindent We next define a function which is $\mathbf{(x-1)}$ for $\mathbf{1 \leq x < 2}$ and
  is $\mathbf{(6 - 2\,x)}$ for $\mathbf{2 \leq x \leq 3}$ and is $\mathbf{0}$ otherwise.
\begin{myVerbatim}
(%i5) g(x):= if x >= 1 and x < 2 then (x-1)
              elseif x >= 2 and x <= 3 then (6 - 2*x) else 0$
(%i6) map('g,[1/2,1,3/2,2,5/2,3,7/2]);
                                    1
(%o6)                        [0, 0, -, 2, 1, 0, 0]
                                    2
(%i7) qdraw( yr(-1,3),ex1(g(x),x,0,4,lw(5),lc(blue)) )$
\end{myVerbatim} 
which produces
\begin{figure} [h]
   \centerline{\includegraphics[scale=.6]{ch7p13.eps} }
	\caption{ Example of a Piecewise Defined Function  }
\end{figure} 
%% eps code for ch7p13 plot of g(x)
% qdraw(yr(-1,3), line(1,0,3,0,lw(4) ),
%          ex1(g(x),x,0,4,lw(10),lc(blue)  ),
%         pic(eps,"ch7p13",font("Times-Bold",16) ) )$
%

\noindent This is not quite the figure we want, since it doesn't really show the
  kind of discontinuity we want to illustrate.
With our definition of $\mathbf{g(x)}$, \textbf{draw2d} draws a blue vertical
  line from $\mathbf{(2,1)}$ to $\mathbf{(2,2)}$.
We can change the way $\mathbf{g(x)}$ is plotted to get the plot we want,
  as in
\begin{myVerbatim}
(%i8) block([small :1.0e-6],
         qdraw( yr(-1,3),ex1(g(x),x,0,2-small,lw(5),lc(blue)),
                  ex1(g(x),x,2+small,4,lw(5),lc(blue)) ) )$
\end{myVerbatim}
which produces
\begin{figure} [h]
   \centerline{\includegraphics[scale=.6]{ch7p14.eps} }
	\caption{Example with More Care with Plot Limits }
\end{figure} 
%% eps code for ch7p12 plot of uu(1,3)
% block([small:1.0e-6],
%        qdraw(yr(-1,3), line(1,0,3,0,lw(4) ),
%          ex1(g(x),x,0,2-small,lw(10),lc(blue)  ),
%            ex1(g(x),x,2+small,4,lw(10),lc(blue)),
%         pic(eps,"ch7p14",font("Times-Bold",16) ) ))$
% 

\newpage
\noindent The Maxima function \textbf{integrate} cannot handle correctly a
  function defined with the \textbf{if}, \textbf{elseif}, and \textbf{else} constructs.
\begin{myVerbatim}
(%i9) integrate(g(x),x,0,4);
        4
       /
       [
(%o9) I  (if (x > 1) and (x < 2) then x - 1 elseif (x > 2) and (x < 3)
       ]
       /
        0
                                                        then 6 - 2 x else 0) dx
\end{myVerbatim} 
This means that for now we will have to break up the integration interval
  into sub-intervals by hand and add up the individual integral results.
\begin{myVerbatim}
(%i10) integrate(x-1,x,1,2) + integrate(6-2*x,x,2,3);
                                       3
(%o10)                                 -
                                       2
\end{myVerbatim}
\subsection{Area Between Curves Examples}
\subsubsection*{Example 1}  
We will start with a very simple example, finding the area between $\mathbf{f_{1}(x) = \sqrt{x}}$
  and $\mathbf{f_{2}(x) = x^{3/2}}$.
A simple plot shows that the curves cross at $\mathbf{x = 0}$ and $\mathbf{x = 1}$.
\begin{myVerbatim}
(%i1) (load(draw),load(qdraw) )$
(%i2) f1(x) := sqrt(x)$
(%i3) f2(x) := x^(3/2)$
(%i4) qdraw(   xr(-.5,1.5), yr(-.5,1.5),
         ex1(f1(x),x,0,1.5,lw(5),lc(blue) ),
         ex1(f2(x),x,0,1.5,lw(5),lc(red) ) )$
\end{myVerbatim}
%\newpage
This produces the plot:
\begin{figure} [h!]
   \centerline{\includegraphics[scale=0.7]{ch7p9.eps} }
	\caption{$\mathbf{\sqrt{x}}$ (blue) and $\mathbf{x^{3/2}}$ (red)}
\end{figure}
% eps code for simple plot of two curves
%qdraw(   xr(-.5,1.5), yr(-.5,1.5),
%         line( -.5,0,1.5,0 ),
%         line(0,-.5,0,1.5 ),
%         ex1(f1(x),x,0,1.5,lw(5),lc(blue) ),
%         ex1(f2(x),x,0,1.5,lw(5),lc(red) ),
%           pic(eps,"ch7p9",font("Times-Bold",20) ) )$
%

\newpage
\noindent Next we redraw this simple plot, but add some shading in color
  to show the area.
The simplest way to do this is to draw some vertical lines
  between the functions in the region of interest $\mathbf{0 \leq x \leq 1}$.
We can use the \textbf{qdraw} package function \textbf{line}.
We first construct a list of $\mathbf{x}$ axis positions for the vertical
  lines, using values $\mathbf{0.01, 0.02, ... 0.99}$.
We then construct a list \verb|vv| of the vertical lines and
  merge that list with a list \verb|qdlist| containing the rest of
  the plot instructions.
We then use \textbf{apply} to pass this list as a set of arguments 
  to \textbf{qdraw}.
\begin{myVerbatim}
(%i5) qdlist : [xr(-.5,1.5), yr(-.5,1.5),
         ex1(f1(x),x,0,1.5,lw(5),lc(blue) ) ,
         ex1(f2(x),x,0,1.5,lw(5),lc(red) ) ]$
(%i6) xv:float(makelist(i,i,1,99)/100)$
(%i7) (vv:[],for x in xv do
        vv:cons(line(x,f2(x),x,f1(x),lw(1),lc(khaki) ),vv),
        vv:reverse(vv) )$
(%i8) qdlist : append(vv,qdlist)$
(%i9) apply('qdraw, qdlist)$
\end{myVerbatim}
which produces the result
\begin{figure} [h]
   \centerline{\includegraphics[scale=0.7]{ch7p10.eps} }
	\caption{$\mathbf{\sqrt{x}}$ (blue) and $\mathbf{x^{3/2}}$ (red)}
\end{figure}
% eps code for plot of two curves with shaded area
%(%i21) qdlist : [xr(-.5,1.5), yr(-.5,1.5),
%        line( -.5,0,1.5,0 ),
%         line(0,-.5,0,1.5 ),
%         ex1(f1(x),x,0,1.5,lw(5),lc(blue) ) ,
%         ex1(f2(x),x,0,1.5,lw(5),lc(red) ),
%         pic(eps,"ch7p10",font("Times-Bold",20) ) ]$
%(%i22) qdlist : append(vv,qdlist)$
%(%i23) apply('qdraw,qdlist)$
%
%\newpage

\noindent If we did not know the intersection location of the two
  curves, we could use \textbf{solve} or \mv|find_root| for example.
\begin{myVerbatim}
(%i10) solve( f1(x) = f2(x),x );
(%o10)                          [x = 0, x = 1]
\end{myVerbatim} 
Once we know the interval to use for adding up the area
  and we know that in this interval $\mathbf{f_{1}(x) > f_{2}(x)}$,
  we simply sum the infinitesimal areas given by
  $\mathbf{( f_{1}(x) - f_{2}(x) )\,dx}$ (base $\mathbf{dx}$ columns) over
  the interval $\mathbf{0 \leq x \leq 1}$.
\begin{myVerbatim}
(%i11) integrate(f1(x) - f2(x),x,0,1);
                                       4
(%o11)                                ----
                                       15
\end{myVerbatim} 
so the area equals $\mathbf{4/15}$.
\begin{equation}
\mathbf{\int_{0}^{1} (\sqrt{x} - x^{3/2})\,dx = 4/15 }
\end{equation}
\newpage
\subsubsection*{Example 2}
As a second example we consider two polynomial functions: \\
 $\mathbf{f_{1}(x) = (3/10)\,x^5 - 3\,x^4 + 11\,x^3 - 18\,x^2 + 12\, x + 1}$ \\
 and $\mathbf{f_{2}(x) = -4\,x^3 + 28\,x^2 - 56\,x + 32}$.
We first make a simple plot for orientation.
\begin{myVerbatim}
(%i1) f1(x) := (3/10)*x^5 -3*x^4 + 11*x^3 -18*x^2 + 12*x + 1$
(%i2) f2(x) := -4*x^3 + 28*x^2 -56*x + 32$
(%i3) (load(draw),load(qdraw) )$
               qdraw(...), qdensity(...), syntax: type qdraw(); 

(%i4) qdraw(yr(-20,20),ex1(f1(x),x,-1,5,lc(blue) ),
               ex1(f2(x),x,-1,5,lc(red)))$
\end{myVerbatim} 
which produces the plot
\begin{figure} [h]
   \centerline{\includegraphics[scale=0.7]{ch7p11.eps} }
	\caption{$\mathbf{f_{1}(x)}$ (blue) and $\mathbf{f_{2}(x)}$ (red)}
\end{figure}   
% qdraw(yr(-20,20),ex1(f1(x),x,-1,5,lw(5),lc(blue) ),
%         ex1(f2(x),x,-1,5,lw(5),lc(red)),
%         line(-1,0,5,0 ),line(0,-20,0,20 ),
%          pic(eps,"ch7p11",font("Times-Bold",20) ))$

\noindent Using the cursor on the plot, and working from left to right,
  $\mathbf{f_{1}}$ becomes larger than $\mathbf{f_{2}}$ at about $\mathbf{(0.76,3.6)}$, 
  becomes less than $\mathbf{f_{2}}$ at about $\mathbf{(2.3, 2.62)}$, and
  becomes greater than $\mathbf{f_{2}}$ at about $\mathbf{(3.86,2.98)}$.  
The \textbf{solve} function is not able to produce an analytic 
  solution, but returns a polynomial whose roots are the solutions
  we want.
\begin{myVerbatim}
(%i5) solve(f1(x) = f2(x),x);
                     5       4        3        2
(%o5)        [0 = 3 x  - 30 x  + 150 x  - 460 x  + 680 x - 310]
(%i6) grind(%)$
[0 = 3*x^5-30*x^4+150*x^3-460*x^2+680*x-310]$
\end{myVerbatim} 
By selecting and copying the \textbf{grind(..)} output, we can use that
  result to paste in the definition of a function $\mathbf{p(x)}$ which we can then use with
  \mv|find_root|.
\begin{myVerbatim}
(%i7) p(x) := 3*x^5-30*x^4+150*x^3-460*x^2+680*x-310$
(%i8) x1 : find_root(p(x),x,.7,1);
(%o8)                         0.77205830452781
(%i9) x2 : find_root(p(x),x,2.2,2.4);
(%o9)                         2.291819210962957
(%i10) x3 : find_root(p(x),x,3.7,3.9);
(%o10)                         3.865127100061791
(%i11) map('p, [x1,x2,x3] );
(%o11)                [0.0, 0.0, 9.0949470177292824E-13]
(%i12) [y1,y2,y3] : map('f1, [x1,x2,x3] );
(%o12)     [3.613992056691179, 2.575784006305792, 2.882949345140702]
\end{myVerbatim} 
We have checked the solutions by seeing how close to zero $\mathbf{p(x)}$ is when $\mathbf{x}$
  is one of the three roots $\mathbf{[x1, x2, x3]}$.
We now split up the integration into the two separate regions where
 one or the other function is larger.
\begin{myVerbatim}
(%i13) ratprint:false$
(%i14) i1 : integrate(f1(x) - f2(x),x,x1,x2);
                                   41875933
(%o14)                             --------
                                   7947418
(%i15) i2 : integrate(f2(x)-f1(x),x,x2,x3);
                                   12061231
(%o15)                             --------
                                   1741444
(%i16) area : i1 + i2;
                                  30432786985
(%o16)                            -----------
                                  2495489252
(%i17) area : float(area);
(%o17)                         12.19511843643877								   
\end{myVerbatim} 
Hence the total area enclosed is about $\mathbf{12.195}$.
Maxima tries to calculate exactly, replacing floating point
  numbers with ratios of integers, and the default is to warn the
  user about these replacements.
Hence we have used \mv|ratprint : false$| to turn off this
  warning.
\subsection{Arc Length of an Ellipse}
Consider an ellipse whose equation is 
\begin{equation}
\mathbf{\frac{x^2}{a^2} + \frac{y^2}{b^2} = 1}.
\end{equation}
  in which we assume $\mathbf{a}$ is the semi-major axis of the ellipse so $\mathbf{a > b}$.
In the first quadrant ($\mathbf{0 \leq x \leq a}$ and $\mathbf{0 \leq y \leq b}$), we can solve
  for $\mathbf{y}$ as a function of $\mathbf{x}$ :
\begin{equation}
\mathbf{y(x) = b\,\sqrt{1 - (x/a)^2}}.
\end{equation}
If $\mathbf{S}$ is the arc length of the ellipse, then, by symmetry, one fourth of the
  arclength can be calculated using the first quadrant integral
\begin{equation}
\mathbf{\frac{S}{4} = \int_{0}^{a} \sqrt{ 1 + \left( \frac{dy}{dx} \right)^{2} } \, dx}
\end{equation}
We will start working with the argument of the square root, making a change of
  variable $\mathbf{x \rightarrow z}$, with $\mathbf{x = a\,z}$, so $\mathbf{dx = a\, dz}$,
  and $\mathbf{a}$ will come outside the integral from the transformation of $\mathbf{dx}$.
The integration limits using the $\mathbf{z}$ variable are $\mathbf{0 \leq z \leq 1}$.\\  

\noindent We will then replace the semi-minor axis $\mathbf{b}$ by an expression depending on
  the ellipse eccentricity $\mathbf{e}$, which has $\mathbf{0 < e \leq 1}$,
  and whose square is given by
\begin{equation}
\mathbf{ e^2 = 1 -  \left( \frac{b}{a} \right)^2  \leq  1 }
\end{equation}
(since $\mathbf{b < a}$), so
\begin{equation}
 \mathbf{b = a\, \sqrt{1 - e^{2}} } 
\end{equation}
\begin{myVerbatim}
(%i1) assume(a>0,b>0,a>b,e>0,e<1 )$
(%i2) y:b*sqrt(1 - (x/a)^2)$
(%i3) dydx : diff(y,x)$
\end{myVerbatim}
\newpage
\begin{myVerbatim}
(%i4) e1 : 1 + dydx^2;
                                    2  2
                                   b  x
(%o4)                           ----------- + 1
                                         2
                                 4      x
                                a  (1 - --)
                                         2
                                        a
(%i5) e2 : ( subst( [x = a*z,b = a*sqrt(1-e^2)],e1 ),ratsimp(%%) );
                                    2  2
                                   e  z  - 1
(%o5)                              ---------
                                     2
                                    z  - 1
(%i6) e3 : (-num(e2))/(-denom(e2));
                                        2  2
                                   1 - e  z
(%o6)                              ---------
                                         2
                                    1 - z
(%i7) e4 : dz*sqrt(num(e3))/sqrt(denom(e3));
                                           2  2
                              dz sqrt(1 - e  z )
(%o7)                         ------------------
                                           2
                                 sqrt(1 - z )
\end{myVerbatim} 
The two substitutions give us expression \mv|e2|, and we use a desperate device
  to multiply the top and bottom by $\mathbf{(-1)}$ to get \mv|e3|.
We then ignore the factor $\mathbf{a}$ which comes outside the integral and consider
  what is now inside the integral sign (with the required square root).\\  

\noindent We now make another change of variables, with $\mathbf{z \rightarrow u}$,
 $\mathbf{z = \boldsymbol{\sin}(u)}$,
  so $\mathbf{dz = \boldsymbol{\cos}(u) \, du}$.
The lower limit of integration $\mathbf{z = 0 = \boldsymbol{\sin}(u)}$
  transforms into $\mathbf{u = 0}$, and the upper limit $\mathbf{z = 1 = \boldsymbol{\sin}(u)}$
  transforms into $\mathbf{u = \boldsymbol{\pi}/2}$.
\begin{myVerbatim}
(%i8) e5 : subst( [z = sin(u), dz = cos(u)*du ],e4 );
                                            2    2
                        du cos(u) sqrt(1 - e  sin (u))
(%o8)                   ------------------------------
                                          2
                              sqrt(1 - sin (u))
\end{myVerbatim} 
We now use \textbf{trigsimp} but help Maxima out with an \textbf{assume} statement
  about $\mathbf{\boldsymbol{\cos}(u)}$ and $\mathbf{\boldsymbol{\sin}(u)}$.
\begin{myVerbatim}
(%i9) assume(cos(u)>0, sin(u) >0)$
(%i10) e6 : trigsimp(e5);
                                         2    2
(%o10)                      du sqrt(1 - e  sin (u))
\end{myVerbatim} 
We then have
\begin{equation}
\mathbf{\frac{S}{4} = a \, \int_{0}^{\boldsymbol{\pi}/2} \sqrt{1 - e^{2} \, \boldsymbol{\sin}^{2}u } \, du }
\end{equation}
Although \textbf{integrate} cannot return an expression for this integral in
  terms of elementary functions, in this form one is able to recognise the standard
  trigonometric form of the complete elliptic integral of the second kind,
  (a function tabulated numerous places and also available via Maxima).\\  

\noindent Let
\begin{equation}
\mathbf{ E(k) = E(\boldsymbol{\phi} = \boldsymbol{\pi}/2,k) = 
   \int_{0}^{\boldsymbol{\pi}/2} \sqrt{1 - k^{2} \, \boldsymbol{\sin}^{2}u } \, du }
\end{equation}
  be the definition of the complete elliptic integral of the second kind.\\

\noindent The "incomplete" elliptic integral of the second kind (with two arguments) is
\begin{equation}
\mathbf{E(\boldsymbol{\phi},k) = \int_{0}^{\boldsymbol{\phi}} \sqrt{1 - k^{2} \, \boldsymbol{\sin}^{2}u } \, du}.
\end{equation}
Hence we have for the arc length of our ellipse
\begin{equation}
\mathbf{S = 4 \, a \, E(e) = 4 \, a \, E(\boldsymbol{\pi}/2,e) = 4 \, a \,
 \int_{0}^{\boldsymbol{\pi}/2} \sqrt{1 - e^{2} \, \boldsymbol{\sin}^{2}u } \, du}
\end{equation}
We can evaluate numerical values using \textbf{elliptic\_ec}, where\\
  $\mathbf{E(k) = elliptic\_ec(k^2)}$, so $\mathbf{S = 4 \, a\, elliptic\_ec(e^{2}) }$. \\
  
\noindent As a numerical example, take $\mathbf{a = 3}$, $\mathbf{ b = 2}$, so $\mathbf{e^2 = 5/9}$, and
\begin{myVerbatim}
(%i11) float(12*elliptic_ec(5/9));
(%o11)                         15.86543958929059
\end{myVerbatim} 
We can check this numerical value using \mv|quad_qags|:
\begin{myVerbatim}
(%i12) first( quad_qags(12*sqrt(1 - (5/9)*sin(u)^2),u,0,%pi/2) );
(%o12)                          15.86543958929059
\end{myVerbatim}  
\subsection{Double Integrals and the Area of an Ellipse} \label{symdbl}
Maxima has no core function which will compute a symbolic double definite integral,
  (although it would be easy to construct one).
Instead of constructing such a homemade function, to evaluate the
  double integral
\begin{equation}
\mathbf{\int_{u1}^{u2} du \, \int_{v1(u)}^{v2(u)} dv\,f(u,v)  \equiv 
  \int_{u1}^{u2} \left( \int_{v1(u)}^{v2(u)} f(u,v) \, dv \right) \, du}
\end{equation}
  we use the Maxima code
\begin{myVerbatim2}
       integrate( integrate( f(u,v),v,v1(u),v2(u) ), u, u1,u2 )
\end{myVerbatim2} 
in which $\mathbf{f(u,v)}$ can either be an expression depending on the
  variables $\mathbf{u}$ and $\mathbf{v}$, or  a Maxima function, and
  likewise $\mathbf{v1(u)}$ and $\mathbf{v2(u)}$ can either be expressions depending
  on $\mathbf{u}$ or Maxima functions. 
Both $\mathbf{u}$ and $\mathbf{v}$ are "dummy variables", since the value of the
  resulting double integral does not depend on our choice of symbols for
  the integration variables; we could just as well use $\mathbf{x}$ and $\mathbf{y}$.
\subsubsection*{Example 1: Area of a Unit Square}
 The area of a unit square (the sides have length $\mathbf{1}$) is:  
\begin{equation}
\mathbf{\int_{0}^{1} dx \,  \int_{0}^{1} dy   \equiv
  \int_{0}^{1}  \left( \int_{0}^{1} dy\,\right) \, dx}
\end{equation}
which is done in Maxima as:
\begin{myVerbatim}
(%i1) integrate( integrate(1,y,0,1), x,0,1 );
(%o1)                                  1
\end{myVerbatim} 
\subsubsection*{Example 2: Area of an Ellipse}
We seek the area of the ellipse such that points $\mathbf{(x,y)}$ on the boundary must satisfy:
\begin{equation}
\mathbf{\frac{x^2}{a^2} + \frac{y^2}{b^2} = 1}.
\end{equation}
The area will be four times the area of the first quadrant.
The "first quadrant" refers to the region $\mathbf{0 \leq x \leq a}$ and
  $\mathbf{0 \leq y \leq b}$.
For a given value of $\mathbf{y > 0}$, the region inside the arc of the ellipse in the
  first quadrant is determined by $\mathbf{0 \leq x \leq x_{max}}$, where
   $\mathbf{x_{max} = (a/b)\, \sqrt{b^2 - y^2}}$.
For a given value of $\mathbf{x > 0}$, the region inside the arc of the ellipse in the
  first quadrant is determined by $\mathbf{0 \leq y \leq y_{max}}$, where  
  $\mathbf{y_{max} = (b/a)\, \sqrt{a^2 - x^2}}$.\\

\noindent One way to find the area of the first quadrant of this ellipse is to sum the values
  of $\mathbf{dA = dx\,dy}$ by "rows", fixing $\mathbf{y}$ and summing over $\mathbf{x}$
  from $\mathbf{0}$ to $\mathbf{x_{max}}$ (which depends on the $\mathbf{y}$ chosen). 
That first sum over the $\mathbf{x}$ coordinate accumulates the area of that row, located at $\mathbf{y}$
  and having width $\mathbf{dy}$.
To get the total area (of the first quadrant) we then sum over rows, by letting
  $\mathbf{y}$ vary from $\mathbf{0}$ to $\mathbf{b}$.\\
This method corresponds to the formula
\begin{equation}
\mathbf{\frac{A}{4} = \int_{0}^{b}  \left( \int_{0}^{ (a/b)\,\sqrt{b^2 - y^2} } dx \, \right) \,dy }.
\end{equation}
Here we calculate the first quadrant area using this
   method of summing over the area of each row.
\begin{myVerbatim}
(%i1) facts();
(%o1)                                 []
(%i2) assume(a > 0, b > 0, x > 0, x < a, y > 0,y < b )$
(%i3) facts();
(%o3)             [a > 0, b > 0, x > 0, a > x, y > 0, b > y]
(%i4) [xmax : (a/b)*sqrt(b^2 - y^2),ymax : (b/a)*sqrt(a^2-x^2)]$
(%i5) integrate( integrate( 1,x,0,xmax), y,0,b );
                                    %pi a b
(%o5)                               -------
                                       4
\end{myVerbatim} 
which implies that the area interior to the complete ellipse is $\mathbf{\boldsymbol{\pi}\,a\,b}$.\\

\noindent Note that we have tried to be "overly helpful" to Maxima's \textbf{integrate}
  function by constructing an "assume list" with everything we can think of
  about the variables and parameters in this problem.
The main advantage of doing this is to reduce the number of questions
  which \textbf{integrate} decides it has to ask the user.\\ 

\noindent You might think that \textbf{integrate} would not ask for the sign of $\mathbf{(y - b)}$,
  or the sign of $\mathbf{(x - a)}$, since it should infer that sign from the integration limits.
However, the \textbf{integrate} algorithm is "super cautious" in trying to never
  present you with a wrong answer.
The general philosophy is that the user should be willing to work with \textbf{integrate}
  to assure a correct answer, and if that involves answering questions, then so be it.\\

\noindent The second way to find the area of the first quadrant of this ellipse is to sum the values
  of $\mathbf{dA = dx\,dy}$ by "columns", fixing $\mathbf{x}$ and summing over
  $\mathbf{y}$ from $\mathbf{0}$ to $\mathbf{y_{max}}$
  (which depends on the $\mathbf{x}$ chosen). 
That first sum over the $\mathbf{y}$ coordinate accumulates the area of that column, located at $\mathbf{x}$
  and having width $\mathbf{dx}$.
To get the total area (of the first quadrant) we then sum over columns, by letting
  $\mathbf{x}$ vary from $\mathbf{0}$ to $\mathbf{a}$.  \\

\noindent This method corresponds to the formula
\begin{equation}
\mathbf{\frac{A}{4} = \int_{0}^{a}  \left( \int_{0}^{ (b/a)\,\sqrt{a^2 - x^2} } dy \, \right) \, dx }
\end{equation}
and is implemented by
\begin{myVerbatim}
(%i6) integrate( integrate( 1,y,0,ymax), x,0,a ); 
                                    %pi a b
(%o6)                               -------
                                       4
\end{myVerbatim}
\subsubsection*{Example 3: Moment of Inertia for Rotation about the x-axis}
We next calculate the moment of inertia for rotation of an elliptical lamina 
  (having semi-axes $\mathbf{a, b}$) about the $\mathbf{x}$ axis.
We will call this quantity $\mathbf{I_x}$.
Each small element of area $\mathbf{dA = dx\,dy}$ has a mass $\mathbf{dm}$ given by
 $\mathbf{\boldsymbol{\sigma} \,dA}$,
  where $\boldsymbol{\sigma}$ is the mass per unit area, which we assume is a constant independent of
  where we are on the lamina, and which we can express in terms of the total mass $\mathbf{m}$
  of the elliptical lamina, and the total area $\mathbf{A = \boldsymbol{\pi}\,a\,b}$ as
\begin{equation}
\mathbf{\boldsymbol{\sigma} = \frac{total\: mass}{total\: area} = \frac{m}{\boldsymbol{\pi}\,a\,b} }
\end{equation}
Each element of mass $\mathbf{dm}$ contributes an amount $\mathbf{y^2\,dm}$ to
 the moment of inertia $\mathbf{I_x}$, where $\mathbf{y}$ is the distance of
 the mass element from the $\mathbf{x}$ axis.
The complete value of $\mathbf{I_x}$ is then found by summing this quantity over the whole elliptical
  laminate, or because of the symmetry, by summing this quantity over the first
  quadrant and multiplying by $\mathbf{4}$.
\begin{equation}
\mathbf{I_x = \int \int_{ellipse} y^2\,dm =
                \int \int_{ellipse} y^2\,\boldsymbol{\sigma} \, dx\,dy =
                4\,\boldsymbol{\sigma} \, \int \int_{first \: quadrant} y^2\,dx\,dy }
\end{equation}
Here we use the method of summing over rows:
\begin{myVerbatim}
(%i7) sigma : m/(%pi*a*b)$
(%i8) 4*sigma*integrate( integrate(y^2,x,0,xmax),y,0,b );
                                      2
                                     b  m
(%o8)                                ----
                                      4
\end{myVerbatim} 
Hence we have derived $\mathbf{I_x = m\,b^2/4}$ for the moment of inertia of an elliptical
  laminate having semi-axes $\mathbf{a, b}$ for rotation about the $\mathbf{x}$ axis.
\newpage
\noindent Finally, let's use \textbf{forget} to remove our list of assumptions,
  to show you the types of questions which can arise.
We try calculating the area of the first quadrant, using the method of
  summing over rows, without any assumptions provided:
\begin{myVerbatim}
(%i9) forget(a > 0, b > 0, x > 0, x < a, y > 0,y < b )$
(%i10) facts();
(%o10)                                []
(%i11) integrate( integrate(1,x,0,xmax),y,0,b);
Is  (y - b) (y + b)  positive, negative, or zero?
n;
     2    2
Is  b  - y   positive or zero?
p;
Is  b  positive, negative, or zero?
p;
                                    %pi a b
(%o11)                              -------
                                       4
\end{myVerbatim} 
and if we just tell Maxima that \mv|b| is positive:
\begin{myVerbatim}
(%i12) assume(b>0)$
(%i13) integrate( integrate(1,x,0,xmax),y,0,b);
Is  (y - b) (y + b)  positive, negative, or zero?
n;
     2    2
Is  b  - y   positive or zero?
p;
                                    %pi a b
(%o13)                              -------
                                       4
\end{myVerbatim} 
This may give you some feeling for the value of providing some help to \textbf{integrate}.
%\newpage
\subsection{Triple Integrals: Volume and Moment of Inertia of a Solid Ellipsoid}
Consider an ellipsoid with semi-axes $\mathbf{(a,b,c)}$ such that $\mathbf{-a \leq x \leq a}$,
  $\mathbf{-b \leq y \leq b}$, and $\mathbf{-c \leq z \leq c}$.
We also assume here that $\mathbf{a > b > c}$.
Points $\mathbf{(x,y,z)}$ which live on the surface of this ellipsoid satisfy the
  equation of the surface  
\begin{equation}
\mathbf{\frac{x^{2}}{a^{2}} + \frac{y^{2}}{b^{2}} + \frac{z^{2}}{c^{2}} = 1 }.
\end{equation}
\subsubsection*{Volume}
The volume of this ellipsoid will be $\mathbf{8}$ times the volume of the first octant,
  defined by $\mathbf{0 \leq x \leq a}$, $\mathbf{0 \leq y \leq b}$, and $\mathbf{0 \leq z \leq c}$.
To determine the volume of the first octant we pick fixed values of $\mathbf{(x,y)}$ somewhere
  in the first octant, and sum the elementary volume $\mathbf{dV(x,y,z) = dx\,dy\,dz}$ over all
  accessible values of $\mathbf{z}$ in this first octant,
  $\mathbf{0 \leq z \leq c\,\sqrt{1-(x/a)^{2} -(y/b)^{2}}}$.  
The result will still be proportional to $\mathbf{dx\,dy}$, and we sum the z-direction cylinders over
  the entire range of $\mathbf{y}$ (accessible in the first octant), again holding $\mathbf{x}$ fixed as before,
  so for given $\mathbf{x}$, we have $\mathbf{0 \leq y \leq b\,\sqrt{1 - (x/a)^{2}}}$.
We now have a result proportional to $\mathbf{dx}$ (a sheet of thickness $\mathbf{dx}$ whose normal is in
  the direction of the $\mathbf{x}$ axis) which we sum over all values of $\mathbf{x}$ accessible in
  the first octant (with no restrictions): $\mathbf{0 \leq x \leq a}$.
Thus we need to evaluate the triple integral
\begin{equation}
\mathbf{\frac{V}{8} =  \int_{0}^{a} dx \, \int_{0}^{ymax(x)} dy \, \int_{0}^{zmax(x,y)} dz \equiv 
  \int_{0}^{a} \left[ \int_{0}^{ymax(x)} \left( \int_{0}^{zmax(x,y)} dz \right)  dy \, \right] dx }
\end{equation}
Here we call on \textbf{integrate} to do these integrals.
\begin{myVerbatim}
(%i1) assume(a>0,b>0,c>0,a>b,a>c,b>c)$
(%i2) assume(x>0,x<a,y>0,y<b,z>0,z<c)$
(%i3) zmax:c*sqrt(1-x^2/a^2-y^2/b^2)$
(%i4) ymax:b*sqrt(1-x^2/a^2)$
(%i5) integrate( integrate( integrate(1,z,0,zmax),y,0,ymax),x,0,a );
     2  2    2  2    2  2
Is  a  y  + b  x  - a  b   positive, negative, or zero?

n;
                                   %pi a b c
(%o5)                              ---------
                                       6
(%i6) vol : 8*%;
                                  4 %pi a b c
(%o6)                             -----------
                                       3
\end{myVerbatim} 
To answer the sign question posed by Maxima, we can look at the special case $\mathbf{y = 0}$ 
  and note that since $\mathbf{x^{2} \leq a^{2}}$, the expression will almost always be negative
  (ie., except for points of zero measure).
Hence the expression is negative for all points interior to the surface of the ellipsoid.
We thus have the result that the volume of an ellipsoid having semi-axes $\mathbf{(a,b,c)}$
  is given by
\begin{equation}
\mathbf{V = \frac{4\, \boldsymbol{\pi}}{3} a \,b \,c }
\end{equation}
We can now remove the assumption $\mathbf{a > b > c}$, since what we call the $\mathbf{x}$ axis is
 up to us and we could have chosen any of the principal axis directions of the
 ellipsoid the $\mathbf{x}$ axis.\\

\noindent Of course we will get the correct answer if we integrate over the total volume:
\begin{myVerbatim}
(%i7) [zmin:-zmax,ymin:-ymax]$
(%i8) integrate( integrate( integrate(1,z,zmin,zmax),y,ymin,ymax),x,-a,a );
     2  2    2  2    2  2
Is  a  y  + b  x  - a  b   positive, negative, or zero?

n;
                                  4 %pi a b c
(%o8)                             -----------
                                       3
\end{myVerbatim} 
%\newpage
\subsubsection*{Moment of Inertia}
\noindent If each small volume element $\mathbf{dV = dx\,dy\,dz}$ has a mass given by
 $\mathbf{dm = \boldsymbol{\rho} \, dV}$,
  where $\boldsymbol{\rho}$ is the mass density, and if $\boldsymbol{\rho}$ is a constant, then it is easy
  to calculate the moment of inertia $\mathbf{I_{3}}$ for rotation of the ellipsoid about
  the $\mathbf{z}$ axis:
\begin{equation}
\mathbf{I_{3} = \int \int \int (x^{2} + y^{2}) \, dm = 
   \boldsymbol{\rho} \, \int \int \int (x^{2} +y^{2})\, dx\,dy\,dz }
\end{equation}
where the integration is over the volume of the ellipsoid.
The constant mass density $\boldsymbol{\rho}$ is the mass of the ellipsoid divided by its volume.
\newpage
\noindent Here we define that constant density in terms of our previously found
  volume \mvs|vol| and the mass \mvs|m|, and proceed to calculate the moment of inertia:
\begin{myVerbatim}
(%i9) rho:m/vol;
                                      3 m
(%o9)                             -----------
                                  4 %pi a b c
(%i10) i3:rho*integrate(integrate(integrate(x^2+y^2,z,zmin,zmax),
                       y,ymin,ymax),x,-a,a );
     2  2    2  2    2  2
Is  a  y  + b  x  - a  b   positive, negative, or zero?

n;
                                 5  3          7
                         (8 %pi a  b  + 8 %pi a  b) m
(%o10)                   ----------------------------
                                         5
                                 40 %pi a  b
(%i11) ratsimp(i3);
                                    2    2
                                  (b  + a ) m
(%o11)                            -----------
                                       5
\end{myVerbatim} 
Hence the moment of inertia for the rotation about the $\mathbf{z}$ axis of a solid ellipsoid 
  having mass $\mathbf{m}$, and semi-axes $\mathbf{(a,b,c)}$ is:  
\begin{equation}
\mathbf{I_{3} = m\,(a^{2} + b^{2})/5}.
\end{equation}
\subsection{Derivative of a Definite Integral with Respect to a Parameter}
\noindent Consider a definite integral in which the dummy integration variable
  is $\mathbf{x}$ and the integrand $\mathbf{f\boldsymbol{(x,y)}}$ is a function of both $\mathbf{x}$
  and some parameter $\mathbf{y}$.
We also assume that the limits of integration $\mathbf{(a,b)}$ are also possibly
  functions of the parameter $\mathbf{y}$.
You can find the proof of the following result (which assumes some mild
 restrictions on the funtions $\mathbf{f(x,y),\, a(y) \; \text{and} \; b(y)}$) in calculus texts:
\begin{equation}
\mathbf{\frac{d}{dy} \int_{a(y)}^{b(y)} f(x,y) \, dx =
   \int_{a}^{b} \frac{\boldsymbol{\partial} \, f(x,y)}{\boldsymbol{\partial} \,y} \,dy  +
   f(b(y),y) \,\frac{d\,b}{d\,y} - f(a(y),y) \, \frac{d\,a}{d\,y} }
\end{equation}
Here we ask Maxima for this result for arbitrary functions:
\begin{myVerbatim}
(%i1) expr : 'integrate(f(x,y),x,a(y),b(y) );
                                b(y)
                               /
                               [
(%o1)                          I     f(x, y) dx
                               ]
                               /
                                a(y)
(%i2) diff(expr,y);
                                                         b(y)
                                                        /
                  d                        d            [      d
(%o2) f(b(y), y) (-- (b(y))) - f(a(y), y) (-- (a(y))) + I     (-- (f(x, y))) dx
                  dy                       dy           ]      dy
                                                        /
                                                         a(y)
\end{myVerbatim} 
and we see that Maxima assumes we have enough smoothness in the functions
  involved to write down the formal answer in the correct form.\\
\subsubsection*{Example 1}
We next display a simple example and begin with the simplest case,
 which is that the limits of integration do not depend on the parameter $\mathbf{y}$.
\begin{myVerbatim}
(%i3) expr : 'integrate(x^2 + 2*x*y, x,a,b);
                               b
                              /
                              [            2
(%o3)                         I  (2 x y + x ) dx
                              ]
                              /
                               a
(%i4) diff(expr,y);
                                      b
                                     /
                                     [
(%o4)                              2 I  x dx
                                     ]
                                     /
                                      a
(%i5) (ev(%,nouns), ratsimp(%%) );
                                     2    2
(%o5)                               b  - a
(%i6) ev(expr,nouns);
                              2      3      2      3
                           3 b  y + b    3 a  y + a
(%o6)                      ----------- - -----------
                                3             3
(%i7) diff(%,y);
                                     2    2
(%o7)                               b  - a
\end{myVerbatim} 
In the last two steps, we have verified the result by first doing the original integral
  and then taking the derivative.
\subsubsection*{Example 2}  
As a second example, we use an arbitrary upper limit $\mathbf{b(y)}$, and then evaluate the
  resulting derivative expression for $\mathbf{b(y) = y^{2}}$.
\begin{myVerbatim}
(%i1) expr : 'integrate(x^2 + 2*x*y, x,a,b(y) );
                              b(y)
                             /
                             [               2
(%o1)                        I     (2 x y + x ) dx
                             ]
                             /
                              a
\end{myVerbatim}
\newpage
\begin{myVerbatim}
(%i2) diff(expr,y);
                                                     b(y)
                                                    /
                   2                 d              [
(%o2)            (b (y) + 2 y b(y)) (-- (b(y))) + 2 I     x dx
                                     dy             ]
                                                    /
                                                     a
(%i3) ( ev(%,nouns,b(y)=y^2 ), expand(%%) );
                                  5      4    2
(%o3)                          2 y  + 5 y  - a
(%i4) ev(expr,nouns);
                         3           2         2      3
                        b (y) + 3 y b (y)   3 a  y + a
(%o4)                   ----------------- - -----------
                                3                3
(%i5) diff(%,y);
              2     d                      d               2
           3 b (y) (-- (b(y))) + 6 y b(y) (-- (b(y))) + 3 b (y)
                    dy                     dy                      2
(%o5)      ---------------------------------------------------- - a
                                    3
(%i6) ( ev(%,nouns,b(y)=y^2), (expand(%%) );
                                  5      4    2
(%o6)                          2 y  + 5 y  - a
\end{myVerbatim} 
We have again done the differentiation two ways as a check on consistency.\\
\subsubsection*{Example 3}
An easy example is to derive 
\begin{equation}
\mathbf{\frac{d}{d\,t} \int_{t}^{t^2} (2\,x+t) \,d\,x = 4\,t^3+3\,t^2 - 4\,t}
\end{equation}
\begin{myVerbatim}
(%i7) expr : 'integrate(2*x + t,x,t,t^2);
                                 2
                                t
                               /
                               [
(%o7)                         I   (2 x + t) dx
                               ]
                               /
                                t
(%i8) (diff(expr,t),expand(%%) );
                                  3      2
(%o8)                         4 t  + 3 t  - 4 t
\end{myVerbatim}
\newpage
\subsubsection*{Example 4}
Here is an example which shows a common use of the differentiation of an 
  integral with respect to a parameter.
The integral 
\begin{equation}
\mathbf{f_{1}(a,w) =  \int_{0}^{\infty} x\,e^{- a\,x} \, \boldsymbol{\cos}(w\,x)\,dx}
\end{equation}
can be done by Maxima with no questions asked, if we tell 
  Maxima that $\mathbf{ a > 0}$ and $\mathbf{ w > 0}$.
\begin{myVerbatim}
(%i1) assume( a > 0, w > 0 )$
(%i2) integrate(x*exp(-a*x)*cos(w*x),x,0,inf);
                                      2    2
                                     w  - a
(%o2)                         - -----------------
                                 4      2  2    4
                                w  + 2 a  w  + a
\end{myVerbatim} 
But if we could not find this result directly as above, we could find the result
  by setting $\mathbf{f_{1}(a,w) = - \partial f_{2}(a,w)/(\partial a) }$, where
\begin{equation}
\mathbf{f_{2}(a,w) = \int_{0}^{\infty} e^{- a\,x} \, \boldsymbol{\cos}(w\,x)\,dx}
\end{equation}
since the differentiation will result in the integrand being
 multiplied by the  factor $\mathbf{(-x)}$ and produce the negative of the integral of interest.
\begin{myVerbatim}
(%i3) i1 : 'integrate(exp(-a*x)*cos(w*x),x,0,inf)$
(%i4) i2 : ev(i1,nouns);
                                       a
(%o4)                               -------
                                     2    2
                                    w  + a
(%i5) di2 : (diff(i2,a), ratsimp(%%) );
                                     2    2
                                    w  - a
(%o5)                          -----------------
                                4      2  2    4
                               w  + 2 a  w  + a
(%i6) result : (-1)*(diff(i1,a) = di2 );
                inf
               /                                    2    2
               [        - a x                      w  - a
(%o6)          I    x %e      cos(w x) dx = - -----------------
               ]                               4      2  2    4
               /                              w  + 2 a  w  + a
                0
\end{myVerbatim} 
which reproduces the original result returned by \textbf{integrate}.
\newpage
\subsection{Integration by Parts}
Suppose $\mathbf{f(x)}$ and $\mathbf{g(x)}$ are two continuously differentiable functions
  in the interval of interest.
Then the \textbf{integration by parts rule} states that given an interval with
  endpoints $\mathbf{(a, b)}$, (and of course assuming the derivatives exist) one has  
\begin{equation}
\mathbf{\int_{a}^{b} f(x)\, g'(x)\, dx = [f(x)\,g(x)]_{a}^{b} - \int_{a}^{b} f'(x)\,g(x)\,dx}
\end{equation}
where the prime indicates differentiation.
This result follows from the product rule of differentiation.
This rule is often stated in the context of indefinite integrals as
\begin{equation}
\mathbf{\int f(x)\, g'(x)\, dx = f(x)\,g(x) - \int f'(x)\,g(x)\,dx}
\end{equation}
or in an even shorter form, with $\mathbf{u = f(x)}$, $\mathbf{du = f'(x)\,dx}$, $\mathbf{v = g(x)}$,
  and $\mathbf{dv = g'(x)\,dx}$, as
\begin{equation}
\mathbf{\int u\,dv = u\,v - \int v\,du}
\end{equation}
In practice, we are confronted with an integral whose integrand can be viewed
 as the product of two factors, which we will call $\mathbf{f(x)}$ and $\mathbf{h(x)}$:
 \begin{equation}
 \mathbf{\int f(x) \, h(x)\,dx}
 \end{equation}
 and we wish to use integration by parts to get an integral involving the
   derivative of the first factor, $\mathbf{f(x)}$, which will hopefully result in a
   simpler integral.
We then identify $\mathbf{h(x) = g'(x)}$ and solve this equation for $\mathbf{g(x)}$ (by integrating;
  this is also a choice based on the ease of integrating the second factor $\mathbf{h(x)}$
  in the given integral).
Having $\mathbf{g(x)}$ in hand we can then write out the result using the indefinite integral
  integration by parts rule above.
We can formalize this process for an \it indefinite integral \rm with the Maxima code:
\begin{myVerbatim}
(%i1) iparts(f,h,var):= block([g ],
         g : integrate(h,var),
         f*g - 'integrate(g*diff(f,var),var ) )$
\end{myVerbatim} 
Let's practice with the integral $\mathbf{\int x^2 \, \boldsymbol{\sin}(x) \,dx}$,
 in which $\mathbf{f(x) = x^2}$ and $\mathbf{h(x) = \boldsymbol{\sin}(x)}$, so
 we need to be able to integrate $\boldsymbol{\sin(x)}$ and want to
  transfer a derivative on to $\mathbf{x^2}$, which will reduce the first factor to $\mathbf{2\,x}$.
Notice that it is usually easier to work with "Maxima expressions" rather than
  with "Maxima functions" in a problem like this.
\begin{myVerbatim}
(%i2) iparts(x^2,sin(x),x);
                            /
                            [                2
(%o2)                    2 I x cos(x) dx - x  cos(x)
                            ]
                            /
(%i3) (ev(%,nouns),expand(%%) );
                                     2
(%o3)                 2 x sin(x) - x  cos(x) + 2 cos(x)
(%i4) collectterms(%,cos(x) );
                                            2
(%o4)                   2 x sin(x) + (2 - x ) cos(x)
\end{myVerbatim} 
If we were not using Maxima, but doing everything by hand, we would use two
  integrations by parts (in succession) to remove the factor $\mathbf{x^{2}}$ entirely,
  reducing the original problem to simply knowing the integrals of $\boldsymbol{\sin(x)}$
  and $\boldsymbol{\cos(x)}$. 
\newpage
\noindent Of course, with an integral as simple as this example, there is no need to
  help Maxima out by integrating by parts.
\begin{myVerbatim}
(%i5) integrate(x^2*sin(x),x);
                                            2
(%o5)                   2 x sin(x) + (2 - x ) cos(x)
\end{myVerbatim} 
%\newpage
We can write a similar Maxima function to transform \tcbr{definite integrals} via
  integration by parts.
\begin{myVerbatim}
(%i6) idefparts(f,h,var,v1,v2):= block([g ],
         g : integrate(h,var),
           'subst(v2,var, f*g)  - 'subst(v1,var, f*g ) -
               'integrate(g*diff(f,var),var,v1,v2 ) )$
(%i7) idefparts(x^2,sin(x),x,0,1);
          1
         /
         [                                    2
(%o7) 2 I  x cos(x) dx + substitute(1, x, - x  cos(x))
         ]
         /
          0
                                                                      2
                                                - substitute(0, x, - x  cos(x))
(%i8) (ev(%,nouns),expand(%%) );
(%o8)                       2 sin(1) + cos(1) - 2
(%i9) integrate(x^2*sin(x),x,0,1);
(%o9)                       2 sin(1) + cos(1) - 2
\end{myVerbatim} 
\vspace{4cm}
\subsection{Change of Variable and \textbf{changevar} }
Many integrals can be evaluated most easily by making a change of variable of
  integration.
A simple example is:
\begin{equation}
\mathbf{\int 2 \, x \, (x^2+1)^3 \, dx = \int (x^2+1)^3\,d\,(x^2+1) = \int u^3\, du = u^4/4 = (x^2+1)^4/4}
\end{equation}
There is a function in Maxima, called \textbf{changevar} which will help you change variables
  in a one-dimensional integral (either indefinite or definite).
However, this function is buggy at present and it is safer to do the
  change of variables "by hand".
\begin{quote}
Function: \textbf{changevar(integral-expression, g(x,u), u, x)}\\ 
Makes the change of variable in a given \it noun form \rm \textbf{integrate}
  expression such that the old variable of integration is $\mathbf{x}$, the
  new variable of integration is $\mathbf{u}$, and $\mathbf{x}$ and $\mathbf{u}$ are
  related by the equation $\mathbf{g(x,u) = 0}$. 
\end{quote}
\newpage
\subsubsection*{Example 1}
Here we use this Maxima function on the simple indefinite integral 
   $\mathbf{\int 2\,x\,(x^{2}+1)^{3} \, dx}$ we have just done "by hand":  
\begin{myVerbatim}
(%i1) expr : 'integrate(2*x*(x^2+1)^3,x);
                                /
                                [     2     3
(%o1)                         2 I x (x  + 1)  dx
                                ]
                                /
(%i2) changevar(expr,x^2+1-u,u,x);
                                    /
                                    [  3
(%o2)                               I u  du
                                    ]
                                    /
(%i3) ev(%, nouns);
                                       4
                                      u
(%o3)                                 --
                                      4
(%i4) ratsubst(x^2+1,u,%);
                           8      6      4      2
                          x  + 4 x  + 6 x  + 4 x  + 1
(%o4)                     ---------------------------
                                       4
(%i5) subst(x^2+1,u,%o3);
                                     2     4
                                   (x  + 1)
(%o5)                              ---------
                                       4
(%i6) subst(u=(x^2+1),%o3);
                                     2     4
                                   (x  + 1)
(%o6)                              ---------
                                       4									  
\end{myVerbatim} 
The original indefinite integral is a function of $\mathbf{x}$, which we obtain
  by replacing $\mathbf{u}$ by its equivalent as a function of $\mathbf{x}$.
We have shown three ways to make this replacement to get a function of $\mathbf{x}$,
 using \textbf{subst}  and \textbf{ratsubst}.
\subsubsection*{Example 2}
As a second example, we use a change of variables
   to find $\mathbf{\int \, \left[ \, (x+2)/\sqrt{x+1}\, \right] \, dx  }$.
\begin{myVerbatim}
(%i7) expr : 'integrate( (x+2)/sqrt(x+1),x);
                               /
                               [    x + 2
(%o7)                          I ----------- dx
                               ] sqrt(x + 1)
                               /
\end{myVerbatim}
\newpage
\begin{myVerbatim}
(%i8) changevar(expr,u - sqrt(x+1),u,x);
Is  u  positive, negative, or zero?

pos;
                                /
                                [     2
(%o8)                           I (2 u  + 2) du
                                ]
                                /
(%i9) ev(%, nouns);
                                     3
                                  2 u
(%o9)                             ---- + 2 u
                                   3
(%i10) subst(u = sqrt(x+1),%);
                                  3/2
                         2 (x + 1)
(%o10)                   ------------ + 2 sqrt(x + 1)
                              3
\end{myVerbatim} 
Of course Maxima can perform the original integral in these two examples
  without any help, as in:
\begin{myVerbatim}
(%i11) integrate((x+2)/sqrt(x+1),x);
                                   3/2
                            (x + 1)
(%o11)                   2 (---------- + sqrt(x + 1))
                                3
\end{myVerbatim} 
However, there are occasionally cases in which you can help Maxima find an
  integral ( for which Maxima can only return the noun form) by
  first making your own change of variables and then letting Maxima try again.
\subsubsection*{Example 3}
Here we use \textbf{changevar} with a \tcbr{definite integral}, using
  the same integrand as in the previous example.
For a definite integral, the variable of integration is a "dummy variable", and
  the result is not a function of that dummy variable, so there is no issue
  about replacing the new integration variable $\mathbf{u}$ by the original variable $\mathbf{x}$
  in the result.
\begin{myVerbatim}
(%i12) expr : 'integrate( (x+2)/sqrt(x+1),x,0,1);
                                1
                               /
                               [     x + 2
(%o12)                         I  ----------- dx
                               ]  sqrt(x + 1)
                               /
                                0
\end{myVerbatim}
\newpage
\begin{myVerbatim}
(%i13) changevar(expr,u - sqrt(x+1),u,x);
Is  u  positive, negative, or zero?

pos;
                             sqrt(2)
                            /
                            [            2
(%o13)                      I        (2 u  + 2) du
                            ]
                            /
                             1
(%i14) ev(%, nouns);
                                10 sqrt(2)   8
(%o14)                          ---------- - -
                                    3        3
\end{myVerbatim} 
%\newpage
\subsubsection*{Example 4}
%\small
We next discuss an example which shows that one needs to pay attention to the possible
  introduction of obvious sign errors when using \textbf{changevar}.
The example is the evaluation of the definite integral $\mathbf{\int_{0}^{1} e^{y^2}\,dy}$,
  in which $\mathbf{y}$ is a real variable.
Since the integrand is a positive (real) number over the interval $\mathbf{0 < y \leq 1}$, the
  definite integral must be a positive (real) number.
The answer returned directly by Maxima's \textbf{integrate} function is:
\begin{myVerbatim}
(%i1) fpprintprec:8$
(%i2) i1:integrate (exp (y^2),y,0,1);
                              sqrt(%pi) %i erf(%i)
(%o2)                       - --------------------
                                       2
\end{myVerbatim}
and \textbf{float} yields
\begin{myVerbatim}
(%i3) float(i1);
(%o3)                        - 0.886227 %i erf(%i)
\end{myVerbatim}
so we see if we can get a numerical value from \mv|erf(%i)| by multiplying
  \mv|%i| by a floating point number:
\begin{myVerbatim}
(%i4) erf(1.0*%i);
(%o4)                            1.6504258 %i
\end{myVerbatim}
 so to get a numerical value for the integral we use the same trick
\begin{myVerbatim}
(%i5) float(subst(1.0*%i,%i,i1));
(%o5)                              1.4626517
\end{myVerbatim}
Maxima's symbol \textbf{erf(z)} represents the error function $\mathbf{Erf(z)}$.
We have discussed the Maxima function \textbf{erf(x)} for real $\mathbf{x}$ in Example 4
  in Sec.(\ref{symbolic}).
Here we have a definite integral result returned in terms of \mv|erf(%i)|, which
  is the error function with a pure imaginary agument and we have just seen that 
  \mv|erf(%i)| is purely imaginary with an approximate value \mv|1.65*%i|.\\

\noindent We can confirm the numerical value of the integral \mv|i1| using 
     the quadrature routine \mv|quad_qags|:
\begin{myVerbatim}
(%i6) quad_qags(exp(y^2),y,0,1);
(%o6)                 [1.4626517, 1.62386965E-14, 21, 0]
\end{myVerbatim}
and we see agreement.\\

\noindent Let's ask Maxima to change variables in this definite integral
  from $\mathbf{y}$ to $\mathbf{u = y^2}$ in the following way:
\begin{myVerbatim}
(%i7) expr : 'integrate(exp(y^2),y,0,1);
                                   1
                                  /     2
                                  [    y
(%o7)                             I  %e   dy
                                  ]
                                  /
                                   0
(%i8) changevar(expr,y^2-u,u,y);
                                   1
                                  /      u
                                  [    %e
                                  I  ------- du
                                  ]  sqrt(u)
                                  /
                                   0
(%o8)                           - -------------
                                        2
(%i9) ev(%, nouns);
                             sqrt(%pi) %i erf(%i)
(%o9)                        --------------------
                                      2
(%i10) float(subst(1.0*%i,%i,%));
(%o10)                            - 1.4626517									
\end{myVerbatim}
   which is the negative of the correct result.
Evidently, Maxima uses \textbf{solve}$\mathbf{(y^2 = u,\,y)}$ to find $\mathbf{y(u)}$ and even though
  there are two solutions $\mathbf{ y = \pm \sqrt{u} }$, Maxima picks the wrong solution
  without asking the user a clarifying question.
\textbf{We need to force Maxima to use the correct relation} between $\mathbf{y}$ and $\mathbf{u}$, as in:
\begin{myVerbatim}
(%i11) changevar(expr,y-sqrt(u),u,y);
Is  y  positive, negative, or zero?

pos;
                                  1
                                 /      u
                                 [    %e
                                 I  ------- du
                                 ]  sqrt(u)
                                 /
                                  0
(%o11)                           -------------
                                       2
(%i12) ev(%, nouns);
                              sqrt(%pi) %i erf(%i)
(%o12)                      - --------------------
                                       2
(%i13) float(subst(1.0*%i,%i,%));
(%o13)                             1.4626517
\end{myVerbatim}
which is now the correct result with the correct sign.
%\normalsize
\subsubsection*{Example 5}
We now discuss an example  of a change of variable in which \textbf{changevar} 
   produces the wrong overall sign, even though we try to be very careful.
We consider the indefinite integral $\mathbf{\int \left( \,x/\sqrt{x^{2}-4} \, \right)\, dx}$, which
  \textbf{integrate} returns as:
\begin{myVerbatim}
(%i1) integrate(x/sqrt(x^2-4),x);
                                       2
(%o1)                            sqrt(x  - 4)
\end{myVerbatim}
Now consider the change of variable $\mathbf{x \rightarrow t}$ with
 $\mathbf{x = 2/\boldsymbol{\cos}(t)}$.\\

\noindent We will show first the \textbf{changevar} route (with its error) and then
  how to do the change of variables "by hand", but with Maxima'a assistance.
Here we begin with assumptions about the variables involved.
\begin{myVerbatim}
(%i2) assume(x > 2, t > 0, t < 1.5, cos(t) > 0, sin(t) > 0 );
(%o2)           [x > 2, t > 0, t < 1.5, cos(t) > 0, sin(t) > 0]
(%i3) nix : 'integrate(x/sqrt(x^2-4),x);
                               /
                               [      x
(%o3)                          I ------------ dx
                               ]       2
                               / sqrt(x  - 4)
(%i4) nixt : changevar(nix,x-2/cos(t), t, x) ;
                    /
                    [                  sin(t)
(%o4)        - 2 %i I ----------------------------------------- dt
                    ]                     2
                    / sqrt(cos(t) - 1) cos (t) sqrt(cos(t) + 1)
(%i5) nixt : rootscontract(nixt);
                            /
                            [          sin(t)
(%o5)                - 2 %i I ------------------------- dt
                            ]    2            2
                            / cos (t) sqrt(cos (t) - 1)
(%i6) nixt : scanmap('trigsimp,nixt);
                                   /
                                   [    1
(%o6)                          - 2 I ------- dt
                                   ]    2
                                   / cos (t)
(%i7) ev(nixt,nouns);
(%o7)                             - 2 tan(t)
\end{myVerbatim}
Since we have assumed $\mathbf{t > 0}$, we have $\mathbf{\boldsymbol{\tan}(t) > 0}$,
 so \textbf{changevar} is telling us the indefinite integral is a negative number
 for the range of $\mathbf{t}$   assumed.\\
  
\noindent Since we are asking for an indefinite integral, and we want the result in terms
  of the original variable $\mathbf{x}$, we would need to do some more work on this answer,
  maintaing the assumptions we have made.
We will do that work after we have repeated this change of variable, doing
  it "by hand".
  
\newpage
\noindent We work on the product $\mathbf{f(x)\,dx}$:
\begin{myVerbatim}
(%i8) ix : subst(x=2/cos(t),x/sqrt(x^2 - 4) )* diff(2/cos(t));
                                4 sin(t) del(t)
(%o8)                      -------------------------
                                   4            3
                           sqrt(------- - 4) cos (t)
                                   2
                                cos (t)
(%i9) ix : trigsimp(ix);
                                    2 del(t)
(%o9)                            - -----------
                                      2
                                   sin (t) - 1
(%i10) ix : ratsubst(1,cos(t)^2+sin(t)^2,ix);
                                   2 del(t)
(%o10)                              --------
                                      2
                                   cos (t)
(%i11) integrate(coeff(ix,del(t) ) ,t);
(%o11)                             2 tan(t)
\end{myVerbatim}
   which is the result \textbf{changevar} should have produced.
%\newpage
Now let's show how we can get back to the indefinite integral produced 
  the direct use of \textbf{integrate}.
\begin{myVerbatim}
(%i12) subst(tan(t)= sqrt(sec(t)^2-1),2*tan(t) );
                                        2
(%o12)                         2 sqrt(sec (t) - 1)
(%i13) subst(sec(t)=1/cos(t),%);
                                        1
(%o13)                         2 sqrt(------- - 1)
                                        2
                                     cos (t)
(%i14) subst(cos(t)=2/x,%);
                                        2
                                       x
(%o14)                           2 sqrt(-- - 1)
                                       4
(%i15) ratsimp(%);
                                       2
(%o15)                            sqrt(x  - 4)
\end{myVerbatim}
\normalsize
This concludes our discussion of a change of variable of integration
  and our discussion of symbolic integration.
\end{document}