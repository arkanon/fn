\documentclass[12pt, a4paper, fleqn]{article}

% Arkanon <arkanon@lsd.org.br>
% 2014/03/22 (Sat) 19:06:35 (BRS)
%
% v1.0 1994

% sudo apt-get install texlive-lang-portuguese	# hifenizacao
% sudo apt-get install texlive-latex-extra	# changepage

\usepackage[top=1cm, bottom=1cm, left=1cm, right=1cm]{geometry}
\usepackage[brazilian]{babel}
\usepackage[utf8]{inputenc}
\usepackage[T1]{fontenc}
\usepackage{textcomp}
\usepackage{hyperref}
\usepackage[table]{xcolor}
\usepackage{enumerate}
\usepackage{changepage}
\usepackage{parskip}
\usepackage{mathdots}
\usepackage{tikz}
\usepackage{graphicx}
\usepackage{lmodern}

\usetikzlibrary{decorations.pathreplacing,calc}

\hypersetup
{
  bookmarks    = true,    % show bookmarks bar?
  unicode      = true,    % non-Latin characters in Acrobat’s bookmarks
  pdftoolbar   = true,    % show Acrobat’s toolbar?
  pdfmenubar   = true,    % show Acrobat’s menu?
  pdfnewwindow = true,    % links in new window
  colorlinks   = true,    % false: boxed links; true: colored links
  linkcolor    = red,     % color of internal links (change box color with linkbordercolor)
  citecolor    = green,   % color of links to bibliography
  filecolor    = magenta, % color of file links
  urlcolor     = blue,    % color of external links
  pdffitwindow = false,   % window fit to page when opened
  pdfstartview = {FitH},  % fits width (Horizontal) or height (Vertical) to the window
  pdftitle     = {Função Exponencial Múltipla},
% pdfsubject   = {},
  pdfauthor    = {Arkanon <arkanon@lsd.org.br>},
  pdfkeywords  = {matemática} {porcentagem}
% pdfproducer  = {},
% pdfcreator   = {},
}

\DeclareUnicodeCharacter{00B7}{\cdot}   % ·
\DeclareUnicodeCharacter{2192}{\to}     % →
\DeclareUnicodeCharacter{2248}{\approx} % ≈

% middle dot      (·) &middot; &#183;  U+00B7
% dot operator    (·) &sdot;   &#8901; U+22C5
% bullet operator (∙)          &#8729; U+2219
% bullet          (•) &bull;   &#8226; U+2022

\newcommand{\mb}    [1]{{\color{black}#1}}

\newcommand{\perc}  [1]{{\color{blue}#1\%}}

\newcommand{\money} [1]{{\color{blue}\mbox{R}\textdollaroldstyle\,#1}}

\newcommand{\email} [1]{<\href{mailto:#1}{#1}>}

\newcommand{\signat}[2]
{
  \par
  \vspace*{\fill}
  \raggedleft{\scriptsize{{\color{#1}#2}}}
}

\newcommand{\blubox}[1]
{
  \begin{adjustwidth}{1.1cm}{1.6cm}
    \fcolorbox
    {mblue}
    {lblue}
    {
      \parbox
      {\linewidth}
      {
        \vskip5pt
        \leftskip4pt
        \rightskip4pt
        \color{mblue}
        #1
        \vskip5pt
      }
    }
  \end{adjustwidth}
}

\newcommand\leftidx[3]
{
  {\vphantom{#2}}#1#2#3
}

\newcommand{\veryhigh}[3]% base, exponent, text
{
  \begin{tikzpicture}[line width=1pt,join=round,cap=round]
    \node (tempnode-0) at (0,0) {$#1$};
    \foreach \mytext [count=\c] in {#2}
    {
      \pgfmathtruncatemacro{\b}{\c-1}
      \node[above right,font=\scriptsize,inner sep=3pt] (tempnode-\c) at (tempnode-\b) {$\mytext$};
      \xdef\maxexp{\c}
    }
    \draw [decoration={brace,amplitude=2pt,mirror,raise=2pt},decorate]
          ($(tempnode-1.south east)+(-0.13,0.13)$) -- node[below right=-.5pt,font=\tiny] {#3} ($(tempnode-\maxexp.south east)+(-0.13,0.13)$);
  \end{tikzpicture}
}

\definecolor{lgray}  {gray}{0.6}
\definecolor{lyellow}{HTML}{FFF2AA}
\definecolor{mblue}  {HTML}{365D95}
\definecolor{lblue}  {HTML}{EAE8E3}

%\everymath={\color{red}}



\begin{document}

\thispagestyle{empty}
\pagestyle{empty}



\begin{center}
  {\LARGE Função Exponencial Múltipla}
\end{center}

\begin{flushright}
  {\sf\small Arkanon \email{arkanon@lsd.org.br}}\\
  {\sf\footnotesize 22/03/2014 19:09:06}
\end{flushright}



\[
   \left(\frac{1}{e}\right)^{\left(\frac{1}{e}\right)} = e^{-e^{-1}} = \frac{1}{\sqrt[e]{e}}
\]

\[
   \sqrt[{e}]{e} = e^{e^{-1}}
\]

\[
   \sqrt[{\sqrt[{e}]{e}}]{e} = e^{e^{-e^{-1}}}
\]

\[
   \sqrt[{\sqrt[{\sqrt[{\sqrt[e]{e}}]{e}}]{e}}]{e} = e^{e^{-e^{-e^{-e^{-1}}}}}
\]

\[
   f_{n}(x) = x^{x^{\iddots^{\raisebox{-.4ex}{$\scriptscriptstyle x$}}}}
\]

\[
   f_{n}{\scriptscriptstyle\left(\frac{1}{x}\right)} = x^{-x^{-x^{\iddots^{-x^{-1}}}}}
\]

\[
   r_{n}(x) = \sqrt[{\sqrt[{\sqrt[{\leftidx{^{\sqrt[x]{x}}}{\ddots}{}}]{x}}]{x}}]{x} = x^{x^{-x^{-x^{\iddots^{-x^{-1}}}}}} = x^{f_{n-1}{\scriptscriptstyle\left(\frac{1}{x}\right)}}
\]



Seja
\[
   f_{n}(x) = \veryhigh{x}{x,{\rotatebox{45}{$\!\!\cdots$}},{\!\scriptscriptstyle x}}{$(n-1)$ vezes}
\]
e
\[
   \begin{tikzpicture}[line width=1pt,join=round,cap=round]
     \node (a) at (0,0) {$r_{n}(x) = \sqrt[{\sqrt[{\sqrt[{\leftidx{^{\sqrt[x]{x}}}{\ddots}{}}]{x}}]{x}}]{x}$};
     \draw [decoration={brace,amplitude=2pt,mirror,raise=2pt},decorate]
           ($(1.35,0)$) -- node[above right=0pt,font=\tiny] {$(n-1)$ vezes} ($(.42,.55)$);
   \end{tikzpicture}
\]
\vspace{20pt}
Então
\[ r_{n}(x) = x^{f_{n-1}{\scriptscriptstyle\left(\frac{1}{x}\right)}} \]



\signat{lgray}{Produzido em \LaTeX}

\end{document}
